\documentclass[fontsize=8pt, a4paper, fleqn, landscape, DIV=calc]{scrartcl}
\usepackage[utf8]{inputenc}
\usepackage[ngerman]{babel}

\usepackage{hyperref}%links in the pdf
\hypersetup{pdfborder = {0 0 0}}

\usepackage{multicol}%layout
\usepackage{geometry}%layout
\usepackage{tcolorbox}%section bars,...
\usepackage{fancyhdr}%header
\usepackage{lastpage}%obvious
\usepackage{dirtytalk}% ""
\usepackage{pdfpages}%Anhang
\usepackage{siunitx}%obvious
\geometry{margin=1cm}
\parindent 0pt
\pagestyle{fancy}
\newlength{\breite}
\setlength{\breite}{0.5pt}
\setlength{\columnseprule}{\breite}
\usepackage{qrcode}%QR code im Titel

\usepackage[table]{xcolor}

%Math stuff
\usepackage{mathtools}
\allowdisplaybreaks %allow display breaks in align

\usepackage{enumitem}%Itemise 


%tikzstuff
\usepackage{tikz,pgfplots}
\pgfplotsset{compat=1.18}
\usetikzlibrary{positioning}
\usetikzlibrary{arrows.meta}
\usepackage[european resistors,american voltages,american currents]{circuitikz}
\ctikzset{bipoles/length=.8cm}%smaller circuit symbols
\ctikzset{voltage/bump b=1.5}%more distance for the voltage arrow 
\ctikzset{bipoles/vsourceam/inner plus={\tiny $+$}}%setting + a bit smaller
\ctikzset{bipoles/vsourceam/inner minus={\tiny $-$}}%setting - a bit smaller
\ctikzset{bipole voltage style/.style={color=blue}, bipole current style/.style={color=red}}%setting colors for I and V
\newcommand{\iarr}[1]{% name
    \node [currarrow, color=red, anchor=center,
    rotate=\ctikzgetdirection{#1-Iarrow}] at (#1-Ipos) {}
}
\newcommand{\varr}[1]{% name
    \draw [color=blue] (#1-Vfrom) .. controls (#1-Vlab)
    .. (#1-Vto) node [currarrow,
    sloped, anchor=tip, allow upside down,pos=1]{};
}
\ctikzset{!vi/.style={no v symbols, no i symbols}}


\newcommand{\todo}[1][todo]{% Todo marker
   \begin{tcolorbox}[colback=orange, beforeafter skip=2pt, boxrule=0pt, arc=2pt, left=0pt, right=0pt, top=0pt, bottom=0pt]%
       {#1}%
   \end{tcolorbox}%
} 

%svg setup
\usepackage{svg}
\svgpath{{svg/}}

\usepackage{wrapfig}

% color for  Titel / sub Titel
\definecolor{sectionbarcolor}{RGB}{148,0,255}
\definecolor{subsectionbarcolor}{RGB}{220,173,255}
\definecolor{sectiontextcolor}{RGB}{255,255,255}
\definecolor{subsectiontextcolor}{RGB}{0,0,0}

%section color box
\setkomafont{section}{\mysection}
\newcommand{\mysection}[1]{%
    \Large%
    \begin{tcolorbox}[colback=sectionbarcolor, coltext=sectiontextcolor, beforeafter skip=2pt, boxrule=0pt, arc=2pt, left=0pt, right=0pt, top=0pt, bottom=0pt]%
        {#1}%
    \end{tcolorbox}%
}

%subsection color box
\setkomafont{subsection}{\mysubsection}
\newcommand{\mysubsection}[1]{%
    \Large%
    \begin{tcolorbox}[colback=subsectionbarcolor, coltext=subsectiontextcolor, beforeafter skip=2pt, boxrule=0pt, arc=2pt, left=0pt, right=0pt, top=0pt, bottom=0pt]%
        {#1}%
    \end{tcolorbox}%
}

%Information for maketitle
\title{\vspace{-1cm}Elektrotechnik 1}
\subtitle{HS 2023, Prof. Dr. Hans-Dieter Lang}
\author{Fabian Steiner, \today}
\date{{\small V1.0.1}}

%Header & footer
\fancyhf{}
\setlength{\footskip}{0.5cm}
\fancyfoot[L]{\thepage{} / \pageref{LastPage}}
\fancyfoot[R]{ELT 1}
\renewcommand{\footrulewidth}{0pt}
\renewcommand{\headrulewidth}{0pt}

\begin{document}
	\begin{multicols*}{2}
        \begin{minipage}{0.75\columnwidth}
		      \maketitle
        \end{minipage}
        \begin{minipage}{0.2\columnwidth}
            \begin{center}
                \quad
                \qrcode[height=1.5cm]{https://github.com/Iceteavanill/Cheatsheet_ELT1_HS_2023_OST}
                \qquad    
            \end{center}
        \end{minipage}
        
        \thispagestyle{fancy}%Pagenumber for first page
        \raggedcolumns
        \section{Grundlagen}

\subsection{Einheiten}

Korrekte Angabe von Werten:\\
\noindent
\begin{minipage}{0.5\columnwidth}
    % LTeX: enabled=false
\begin{tikzpicture}
    \node (value) {12};
    \node [right=0.0cm of value] (prefix) {$\text{\textmu}$};
    \node [right=0.0cm of prefix] (unit) {$\si{\m}$};
    \draw[->] 
    (value.south) -- ++(0,-1.25)
               -- ++(1.4,0)
               node[anchor=west] {Wert};
    \draw[->]
    (prefix.south) -- ++(0,-0.75)
               -- ++(0.9,0)
               node[anchor=west] {Präfix};
    \draw[->]
    (unit.south) -- ++(0,-0.25)
               -- ++(0.5,0)
               node[anchor=west] {Einheit};
    
\end{tikzpicture}
\end{minipage}
\begin{minipage}{0.5\columnwidth}
    \{x\} $\rightarrow$ Zahlenwert = 12\\
    $[x]$ $\rightarrow$ Einheit abrufen = \unit{\kilo\gram\metre\per\square\second}$\rightarrow$ kann präfixe enthalten ist technisch gesehen nicht ganz richtig\\
\end{minipage}

\subsection{Graphen}
\begin{center}
    % LTeX: enabled=false
\begin{tikzpicture}
    \begin{axis}[
        title={Random Voltage over Time},
        xlabel={Time [s] $\rightarrow$ FALSCH},
        ylabel={Spannung in V $\rightarrow$ korrekt},
        xmin=0, xmax=10,
        ymin=-5, ymax=5,
        xtick={0,2,4,6,8,10},
        ytick={-5,-3,...,5},
        grid=both,
        grid style={line width=.1pt, draw=gray!10},
        major grid style={line width=.2pt,draw=gray!50},
        minor tick num=1,
        enlarge x limits={abs=0.5},
        enlarge y limits={abs=0.5},
        axis background/.style={fill=white},
        legend style={at={(0.5,-0.1)},anchor=north},
        no markers,
        every axis plot/.append style={thick}
    ]
    
    \addplot[
        domain=0:10,
        samples=100,
        smooth,
    ] {sin(deg(x)) + rand*0.5};
    
    \end{axis}
\end{tikzpicture}
\end{center}
Für Graphen muss darauf geachtet werden das die Einheit korrekt angegeben wird. 
Die Angabe in Eckigen klammern ist nicht mehr zu verwenden. 

        \section{Grundlagen Elektrotechnik}

\subsection{Ladung}
Wichtigster Grundsatz: \textbf{Ladung ist eine Eigenschaft}.\\
Jedes Elementarteilchen hat eine Elementarladung. 
Daher ist jede gemessene Ladung immer ein vielfache, dieser Elementarladung.\\
Die wichtigsten Eigenschaften sind:
\begin{itemize}
    \item Es gibt 2 Arten. Positiv und negativ
    \item Gleiche Ladung stossen sich ab, ungleiche ziehen sich an
    \item Ladung ist quantisierbar (ein Vielfaches einer Elementarladung)
    \item Ladung bleibt insgesamt erhalten
    \item Ladung ist an Masse gebunden
\end{itemize}
Ladung wird in \unit{\coulomb}  $\widehat{=}$ \unit{\ampere\second} quantisiert.

\subsection{Ladungstransport}

Elektrische Ladung wird unterschiedlich gut in verschiedenen Medien transportiert.\\

\begin{center}
    \begin{tabular}{|l|l|l|} \hline  
        \textbf{Medium} & \textbf{Beweglicher Ladungsträger} & \textbf{Beispiele}\\ \hline\hline 
            Metalle & freie Elektronen & Cu, Ag, Au,.... \\\hline
            Elektrolyte& freie Ionen&Salzwasser, Säuren, Laugen, ....\\\hline
            Gase, Plasma& freie Elektronen, freie Ionen&FL-Röhren, Sonnenplasma\\\hline
            Halbleiter& Freie Elektronen, Löcher&Si, Ge, Se, GaN, ...\\\hline
            Vakuum& Keine(\say{nichts})&CRT-Bildschirme\\\hline
            Isolatoren& \say{nichts}&Keramik, Kunststoffe, ...\\\hline
    \end{tabular}
\end{center}

Luft kann dabei zu den Gasen und zu den Isolatoren gezählt werden, je nach Spannung welche herrscht.\\

\subsection{Elektrischer Strom}

Elektrischer Strom ist der Transport von Ladung.
\[ \text{Einfache definition: I} = \dfrac{\Delta\text{Q}}{\Delta\textbf{t}} = \dfrac{\text{N}_\text{e}}{\Delta\textbf{t}}\]
\[\text{Masseinheit: }[\text{I}] = \dfrac{[\text{Q}]}{[\text{t}]} = \dfrac{\unit{\ampere\second}}{\unit{\second}} = \dfrac{\unit{\coulomb}}{\unit{\second}} = \unit{\ampere}\]
Elektrischer Strom braucht immer eine Bezugsrichtung:\\
\begin{wrapfigure}{l}{0.3\linewidth}
    \includesvg[width = \linewidth, inkscapelatex=false]{svg/Elektrischer Strom.svg}
\end{wrapfigure}
Die Richtung des elektrischen Stromes sagt aus in welche Richtung sich die Elektronen bewegen. 
Es wird zwischen der technischen und der physikalischen Richtung unterschieden, welche entgegen einander verlaufen. 
Die Elementarladung eines Elektrons ist negativ.\\

\subsection{Stomdichte(rechtwinklig zur Fläche)}

Stomdichte sagt aus wie viel Strom über einen Querschnitt fliesst. 
In einem Leiter ohne konstanten Querschnitt bleibt die Stromdichte proportional zur Fläche.
\[\text{Masseinheit: }[\text{J}] = \dfrac{[\text{I}]}{[\text{A}]} = \dfrac{\unit{\ampere}}{\unit{\square\meter}}\]\\


\begin{center}
    \includesvg[width = 0.8\linewidth, inkscapelatex=false]{svg/Stromdichte.svg}
\end{center}

\subsubsection{Schmelzstromdichte}

Die Schmelzstromdichte sagt aus wie viel Strom durch einen Leiter fliessen muss bis dieser anfängt zu schmelzen (typischerweise tausende \unit{\ampere}).

\subsection{Elektrisches Potenzial}

Elektrisches Potenzial ist Arbeit, die für den Spannungstransport zur Verfügung steht oder die Freigesetzt wird bei Transport der Ladung. 

\[ \text{Einfache Definition:} [\varphi] = \frac{\Delta[\text{W}]}{[\text{Q}]} = \frac{\text{J}}{\text{C}} = \unit{\volt} = \frac{\unit{\square\meter\kilo\gram}}{\unit{\cubic\second\ampere}}\]

\subsection{Elektrische Spannung}

Elektrische Spannung ist ein Potenzialunterschied. 
Dh. Elektrische Spannung ist eine Differenz von Energie, die zum Ladungstransport zur Verfügung steht.

\[ \text{Einfache Definition :}[\text{U}] = \varphi_a - \varphi_b = \unit{\volt}   \]

\subsection{Kirchhoff current law / Kirchhoff 1 / KCL}

Das Kirchhoff current law sagt aus das die Summe aller Ströme in einem geschlossenen Knoten 0 sein müssen.

\[ \sum_{n} \text{I}_n = 0\]

Die einzige Bedingung ist das alle Ströme dieselbe Bezugsrichtung haben.

\begin{center}
    % LTeX: enabled=false
\begin{tikzpicture}
    % Define the style for the currents
    \tikzset{current/.style={-{Latex[length=2.5mm]},thick}}
  
    % Draw the node for the junction
    \node[draw, circle, minimum size=0.3cm] (Junction) at (0,0) {};
  
    % Draw the currents entering and leaving the junction
    \draw[current] (-2,0) -- (Junction) node[midway, above] {\(I_2\)};
    \draw[current] (Junction) -- (2,0) node[midway, above] {\(I_1\)};
    \draw[current] (0,-2) -- (Junction) node[midway, right] {\(I_3\)};
  
    % Optionally, add plus and minus signs
    \node at (-0.7,0.3) {+};
    \node at (1.4,0.3) {-};
    \node at (0.5,-1) {+};
  \end{tikzpicture}
\end{center}

\subsection{Kirchhoff voltage law / Kirchhoff 2 / KCV}

Die Summe aller Teilspannungen in einer Masche ist 0. 

\[ \sum_{n} \text{U}_n = 0\]

Die Bedingung ist das alle Spannungen dasselbe Referenzpotential haben.

\subsection{Elektrische Leistung}

Elektrische Leistung ist Arbeit pro Zeit.

\[ \text{Definition :} [\text{P}] = \frac{\Delta[\text{W}]}{\Delta[\text{t}]} = \frac{\unit{\joule}}{\unit{\second}} = \frac{\unit{\volt\ampere\second}}{\unit{\second}} = \unit{\volt\ampere} = \unit{\watt} \]

\subsection{Verbraucher oder Quelle} \label{source_sink}

Um ein Verbraucher und eine Quelle auseinander zu halten kann man sich die Richtungen der Spannung und des Stromes überprüfen. Ist die Spannung  und der der Strom parallel zu einander, handelt es sich um ein Verbraucher(Wie ein Widerstand). Ist die Spannung und der Strom antiparalell, handelt es sich um eine Quelle (aus einer Spannungsquelle fliesst ein Strom hinaus).\\

\begin{center}
    % LTeX: enabled=false
\begin{tikzpicture}
    \draw[thick] (0,0) rectangle (2,5);
    \node at (1,4.5) {Verbraucher};
    \foreach \y in {1,4}
        \draw[thick] (2,\y) -- ++(0.5,0) circle [radius=0.1];

    \draw[<-,red,very thick] (2,1) .. controls (0.5,2.5) .. (2,4) node[midway,left] {I};
    \draw[<-,blue,very thick] (3,1) .. controls (3.5,2.5) .. (3,4) node[midway,right] {U};

    \draw[thick] (5,0) rectangle (7,5);
    \node at (6,4.5) {Quelle};
    \foreach \y in {1,4}
        \draw[thick] (7,\y) -- ++(0.5,0) circle [radius=0.1];

    \draw[->,red,very thick] (7,1) .. controls (5.5,2.5) .. (7,4) node[midway,left] {I};
    \draw[<-,blue,very thick] (8,1) .. controls (8.5,2.5) .. (8,4) node[midway,right] {U};      
\end{tikzpicture}  

\end{center}    

\subsection{Elektrischer Widerstand / Leitwert}

Elektrischer Widerstand / Leitwert ist eine Konsequenz des Zusammenhangs von Elektrischen Strom und Spannung. 
\[\text{I}   \underbrace{\text{ist Proportional zu}}_\propto{}   \text{U} \\ \text{I} = \text{G}\cdot\text{U}\]\\
G ist der so genannte Elektrische Leitwert und von oben abgeleitet: $\text{G} = \dfrac{\text{I}}{\text{U}}$ in der Masseinheit \unit{\siemens}(Siemens).\\
Der Elektrische Widerstand entsteht durch den Kehrwert von dem Leitwert $[\text{R}] = \dfrac{1}{[\text{G}]} = \dfrac{\text{[\text{U}]}}{[\text{I}]} = \unit{\ohm} $ (Ohm).\\

\subsection{Leitfähigkeit}

Elektrische Leitfähigkeit ist eine Materialeigenschaft und hängt von der Mobilität, Anzahl freier Ladungsträger und der Ladung der freien Ladungsträger ab.\\
\begin{align*}
    [\text{Leitfähigkeit}]& = \text{n}\cdot\text{q}\cdot\text{\textmu} = \sigma = \unit[per-mode=fraction]{\ampere\per\volt\per\meter} = \unit[per-mode=fraction]{\siemens\per\meter}\\
    \text{n}& = \text{Anzahl freier Ladungsträger}\\
    \text{q}& = \text{Ladung pro Ladungsträger}\\
    \text{\textmu}& = \text{Mobilität / Freiheit der Elektronen sich zu bewegen}\\
\end{align*}


Die Faktoren \say{n} und \say{q} sind dabei Material konstanten. 
Die Variable \say{\textmu} ist allerdings temperaturabhängig. 

\subsubsection{Mobilität}

Die Mobilität (\textmu) eines Ladungsträgers setzt sich wie folgt zusammen:
\begin{align*}
    \text{\textmu}& =  \frac{\text{q}\tau}{\text{m}}\\
    \text{q}& = \text{Ladung des Ladungsträger}\\
    \tau& = \text{Mittlere Stoszeit zwischen den Ladungsträger}\\
    \text{m}& = \text{Masse des Ladungsträger}\\
    \newline
    [\text{\textmu}]& = \unit[per-mode=fraction]{\coulomb\second\per\kilo\gram} = \unit[per-mode=fraction]{\square\meter\per\volt\per\second}
\end{align*}


\subsubsection{Spezifischer Widerstand}

Der Kehrwert der Leitfähigkeit wird oftmals auch verwendet:\\

\[ [\text{Spezifischer Widerstand}] = \left[ \frac{1}{\sigma} \right] = [\varrho] = \unit[per-mode=fraction]{\volt\per\meter\per\ampere}\]

\subsubsection{Widerstandsberechnungen}

Um aus Leitfähigkeit und spez. Widerstand einen Widerstand berechnen zu können können folgende Formeln verwendet werden.\\
\[
\unit{\ohm} = \frac{\varrho\cdot[\text{Leiterlänge}]}{[\text{Leiterfläche}]} = \frac{\varrho \cdot \unit{\meter}}{\unit{\square\meter}}\\
\unit{\ohm} = \frac{[\text{Leiterlänge}]}{\sigma\cdot[\text{Leiterfläche}]} = \frac{\unit{\meter}}{\sigma\cdot\unit{\square\meter}}\\
\]

Allerdings sind diese Formeln nicht allgemeingültig und können nur dann verwendet werden, wenn die Feldlinien senkrecht zu der Fläche stehen(und sich das nicht verändert). 
Manchmal wird die Einheit auch gekürzt dargestellt (machen aber nur evtl fiese Dozenten).

\subsection{Temperaturabhängigkeit}

Nicht ideale Widerstände ändern ihren Wert, unter anderem, abhängig mit der Temperatur. 
In der Regel ändern sich der Widerstand nicht linear und ist von Widerstand zu Widerstand etwas anders. 
Um trotzdem Widerstandswerte annähern zu können wird die Funktion des Widerstands angenähert. 
Dies erfolgt je nach Anwendung mit einem, zwei oder noch mehr Termen.
\[
\begin{aligned}
    \Delta\text{T}& = \alpha\cdot\Delta\text{T}\cdot\text{R}_\text{t20} &\rightarrow \text{Lineare Annäherung}\\
    \Delta\text{T}& = (\alpha\cdot\Delta\text{T} + \beta\cdot\Delta\text{T})\cdot\text{R}_\text{t20} &\rightarrow \text{Quadratische Annäherung}\\
    &\text{ ...}\\
    \text{R}_{(\text{T})}& = (1 + \alpha\cdot\Delta\text{T})\cdot\text{R}_\text{t20} &\rightarrow \text{Lineare Annäherung}\\
    \text{R}_{(\text{T})}& = (1 + \alpha\cdot\Delta\text{T} + \beta\cdot\Delta\text{T})\cdot\text{R}_\text{t20} &\rightarrow \text{Quadratische Annäherung}\\
\end{aligned}
\]

Der Widerstand $\text{R}_\text{t20}$ setzt hier jeweils den \say{Nullpunkt} oder den Ausgangspunkt, von welcher aus die Temperatur berechnet wird.

\subsubsection{PPM's}

Eine andere, in der Praxis häufig verwendete Art, Temperaturkoeffizienten anzugeben, sind die so genannten PPM's oder auch parts per milion (in der Praxis auch TCR / temperature coefficient). 
Diese sagen aus um wie viel sich ein Widerstands Wert pro Grad Kelvin ändert daher handelt es sich hier nur um ein lineares Modell mit beschränkter Genauigkeit. 
Sie haben meist keine Potenz oder sind ganzzahlig da, wie der Name es schon andeutet, sie mit $10^{-6}$ multipliziert werden. 
Die korespondierende Formel dazu lautet:\\
\[
    \frac{\Delta\text{R}(\Delta\text{T})}{\text{R}_{\text{T0}}} = \frac{\Delta\text{R}}{10^{6}}
\]

        \section{Gleichstromnetzwerke}

Generell wird im folgenden Abschnitt nur lineare Gleichstromnetzwerke behandelt. 
Ein Netzwerk besteht in der Regel aus mindestens 2 verknüpften Komponenten. 
Alle Verbindungen werden als ideale Verbindung zu betrachten.

\subsection{Ideale Quellen}

\subsubsection{Spannungsquelle}

Eine Ideale Spannungsquelle legt fest das über den Anschlüssen die festgelegte Spannung herrscht. 
Es ist egal, wie gross die Last ist. 
Die Spannung bleibt gleich.\\
Es gibt verschiedene Symbole:\\
\begin{center}
    \input{circuits/spannungsquellen.tex}
    \begin{itemize}
        \item \textbf{Typ 1} stellt eine Batterie dar. Durch das Symbol ist klar welche Polarität herrscht. Dh kann nur eine Spannung ohne Spannungspfeil gezeichnet werden.
        \item \textbf{Typ 2} ist die Europäische Darstellungsart. Sie ist weniger gebräuchlich da zwingend immer ein Pfeil mit der Spannung gezeichnet werden muss.
        \item \textbf{Typ 3} ist die Amerikanische Variante. Sie ist die gebräuchlichste Variante da die Polarität auch klargestellt ist.
    \end{itemize}
\end{center}

\subsubsection{Stromquelle}

Eine Ideale Stromquelle legt fest das durch das Symbol der festgelegte Stom fliesst. 
Es ist egal wie gross die Last ist. 
Der Strom fliesst durch.\\
Es gibt verschiedene Symbole:\\
\begin{center}
    % LTeX: enabled=false
\begin{circuitikz}[european]
    \draw (0,0) to[isource, *-*, i=$I_q$, bipole/is voltage=false, !vi, name=I1] (0,2);
    \iarr{I1};
    \node at(0,-0.5) {Typ 1};
    \draw (2,0) to[ioosource, *-*, i=$I_q$, bipole/is voltage=false, !vi, name=I2] (2,2);
    \iarr{I2};
    \node at(2,-0.5) {Typ 2};
\end{circuitikz}
\begin{circuitikz}
    \node at(0,-0.5) {};
    \draw (1,0) to[isource, *-*, i=$I_q$, bipole/is voltage=false, !vi, name=I1] (1,2);
    \iarr{I1};
    \node at(1,-0.5) {Typ 3};
\end{circuitikz}
\end{center}

\begin{itemize}
    \item \textbf{Typ 1} ist die Europäische Darstellungsart. Sie braucht ein Strompfeil da die Richtung des Stromes nicht definiert ist.
    \item \textbf{Typ 2} braucht auch ein Strompfeil, aber nicht sehr gebräuchlich.
    \item \textbf{Typ 3} ist die amerikanische Art und am gebräuchlichsten da die Stromrichtung schon definiert ist.
\end{itemize}

\subsubsection{Aussage ob Quelle oder keine Quelle}

Die obigen Quellen sagen lediglich aus, ob eine Spannung oder ein Strom an einem Punkt existiert. 
Ob die Quelle schlussendlich wirklich eine Quelle ist, daher Energie ins System liefert, muss durch abgleich mit der korrespondierenden Spannung / Strom über der Quelle (siehe Kapitel \ref{source_sink}).\\

\subsubsection{Widersprüche}

Es gibt Widersprüche welche durch die obigen Quellen entstehen können:\\
\begin{center}
    % LTeX: enabled=false
\begin{circuitikz}[american]
    \draw (0,2) to[vsource, o-o, l=$u_q>0$, bipole/is voltage=false] (0,0);
    \draw (0,0) -- (2,0) -- (2,2) -- (0,2) [color=red]; 
    \draw (4,0) to[isource, o-o, l=$I_q>0$, bipole/is voltage=false] (4,2); 
    \draw (4,0) -- (6,0) -- (6,2) -- (4,2) [color=red,dashed];
    \node at(1,2.5) {Schaltung 1};
    \node at(5,2.5) {Schaltung 2};
\end{circuitikz} 
\end{center}

Die beiden obigen Schaltungen sind Widersprüchlich, hervorgerufen durch (oder fehlenen) roten idealen Verbindungen.\\ 
In Schaltung 1 sind die beiden Pole einer Idealen Spannungsquelle verbunden. 
Allerdings ist sie Spannung grösser als 0. 
Die Spannung kann nicht grösser als 0 sein, aber an beiden Polen gleich. 
Ein Widerspruch.\\
In Schaltung 2 erzeugt die ideale Stromquelle ein Strom der grösser ist als 0, dieser kann allerdings nicht fliessen da keine Verbindung vorhanden ist. 
Ein Widerspruch.\\

\subsection{Gesteuerte Quellen}

In der Elektrotechnik viel gebrauchte Elemente sind sogenannte gesteuerte Quellen. 
Sie haben eine Messgrösse, welcher dann mit dem Steuerparameter multipliziert wird und dann als Ausgangsgrösse ausgegeben wird. 
Die Messgrösse und Ausgangsgrösse kann entweder Spannung oder Strom sein. 
Daher gibt es 4 verschiedene Arten Gesteuerte Quellen.\\
\begin{itemize}
    \item Spannungsgesteuerte Spannungsquelle (VCVS)
    \item Spannungsgesteuerte Stromquelle (VCCS)
    \item Stromgesteuerte Spannungsquelle (CCVS)
    \item Stromgesteuerte Stromquelle (CCCS)
\end{itemize}
Der Steuerparameter der VCVS und CCCS sind einheitenlose Grössen, welche meist mit Symbolen wie $g, v, r  \text{ oder } \gamma$ bezeichnet. 
Die Steuerparameter der VCCS und CCVS müssen die Messgrösse in die Ausgangsgrösse umwandeln und haben dadurch eine Einheit.\\
$\text{VCCS} \xrightarrow{} \unit{\ohm}$\\
$\text{CCVS} \xrightarrow{} \unit{\siemens}$\\
Symbole und dessen Messklemmen:\\

\begin{center}
\begin{tabular}{|c||c|}\hline
    VCVS & VCCS\\\hline
    % LTeX: enabled=false
\begin{circuitikz}
    \draw (0,2) -- (1,2) to[open,o-o] (1,0);
    \draw (0,0) -- (1,0);
    \draw[-Latex,blue] (1,1.8) to[bend left, blue]  node[midway, right] {$U_m$}(1,0.2);
\end{circuitikz} 
\begin{circuitikz}[american]
    \node at(0,2.5){};
    \draw (0.5,2) to[cvsource,o-o,l=$\gamma\cdot I_m$] (0.5,0);
\end{circuitikz} & % LTeX: enabled=false
\begin{circuitikz}
    \draw (0,2) -- (1,2) to[open,o-o] (1,0);
    \draw (0,0) -- (1,0);
    \draw[-Latex,blue] (1,1.8) to[bend left, blue]  node[midway, right] {$U_m$}(1,0.2);
\end{circuitikz} 
\begin{circuitikz}[american]
    \node at(0,2.5){};
    \draw (0.5,0) to[cisource,o-o,l_=$\unit{\siemens}\cdot u_m$] (0.5,2);
\end{circuitikz}
     \\\hline\hline
    CCCS & CCVS\\\hline
    % LTeX: enabled=false
\begin{circuitikz}
    \draw (1,2) to[short,o-o,i=$I_m$, !vi, name=I1] (1,0);
    \iarr{I1};
\end{circuitikz} 
\begin{circuitikz}[american]
    \node at(0,2.5){};
    \draw (0.5,0) to[cisource,o-o,l_=$\unit{\siemens}\cdot I_m$] (0.5,2);
\end{circuitikz} & % LTeX: enabled=false
\begin{circuitikz}
    \draw  (1,2) to[short,o-o,i=$I_m$, !vi, name=I1] (1,0);
     \iarr{I1};
\end{circuitikz} 
\begin{circuitikz}[american]
    \node at(0,2.5){};
    \draw (0.5,2) to[cvsource,o-o,l=$\gamma\cdot I_m$] (0.5,0);
\end{circuitikz} 
     \\\hline
\end{tabular}
\end{center}

\subsection{Lineare Quellen}

Eine lineare Quelle besteht aus einer Idealen Quelle und einem (linearen) Widerstand. 
Die Spannung der Quelle verhält sich linear zu dem Strom, der aus der Quelle fliesst. 
Mann kann eine lineare Quelle mit einer Spannungs- oder Stromquelle realisieren. 
Man unterscheidet dabei zwischen einer Thévenin oder einer Norton Quelle. 
Lineare Quellen haben 3 verschiedene Eigenschaften. 
\begin{itemize}
    \item \textbf{Leerlaufspannung} ist die Spannung welche an der Ausgangsklemme anliegt wenn keine last anliegt
    \item \textbf{Kurzschlussstom} ist der maximale Strom welche durch einen kurzschluss an der Ausgangsklemme angelegt erzeugt werden kann
    \item \textbf{Innenwiderstand} ist der innere Widerstand der Quelle
\end{itemize}

Wenn 2 dieser Eigenschaften bekannt sind, kann die letzte berechnet werden da diese linear abhängig ist.

\subsubsection{Thévenin Quelle}

Eine Thévenin Quelle besteht aus einer Idealen Spannungsquelle und einem Widerstand in der folgenden Konstellation:

\begin{center}
    % LTeX: enabled=false
\begin{circuitikz}
    \draw (0,2) to[vsource,l_=$U_q$] (0,0)
     to[short,-o] (2,0);
    \draw (0,2) to[R,l=$R_i$,-o] (2,2);
    \draw (2,0) to[open,name=opn,v] (2,2);
    \draw[-Latex,blue] (opn-Vto) to[bend left, blue]  node[midway, right] {$U_{\text{out}}$}(opn-Vfrom);
\end{circuitikz}
\end{center}

\subsubsection{Norton Quelle}

Eine Nortonquelle besteht aus einer Idealen Stromquelle und einem Widerstand in der folgenden Konstellation:

\begin{center}
    % LTeX: enabled=false
\begin{circuitikz}
    \draw (0,0) to[I,l=$I_q$] (0,2)
    -- (2,2) to[R,l=$R_i$,*-*] (2,0) -- (0,0);
    \draw (2,2)to[short, -o] (4,2)
    to[open,name=opn,v] (4,0)
    to[short, o-] (2,0);
    \draw[-Latex,blue] (opn-Vfrom) to[bend left, blue]  node[midway, right] {$U_{\text{out}}$}(opn-Vto);
\end{circuitikz}
\end{center}

\subsection{Helmholz, Thévenin, Norton - Theorem}

Das Helmholz- / Thévenin- und das Nortontheorem sagen etwas sehr Ähnliches aus. 
Helmholz / Thévenin stellen fest das alle Kombinationen aus linearen Widerständen und idealen Spannungsquellen auf eine Thévenin Quelle reduziert werden können. 
Norton hat das erweitert das man auch auf eine Norton Quelle reduzieren kann. 
Daher, wenn man eine Thévenin quelle hat, kann man diese zu einer Norton Quelle umwandeln und umgekehrt. 
Dies kann sehr nützlich sein, um lineare Netzwerke zu lösen.

\subsection{Superposition / Überlagerungssatz}

Superposition ermöglicht das schrittweise Erkunden von Schaltungen durch Ausschalten von idealen Quellen bis auf eine und das anschliessende überlagern(summieren der strömen). 
Voraussetzung ist eine Lineare Schaltung. 
Alle passiven Bauteile bleiben immer gleich. 
Es werden nur die Quellen verändert. 
Wenn abhängige Quellen, welche wirklich abhängig sind, im Netzwerk vorhanden sind dann dürfen diese nicht ausgeschaltet werden und müssen immer eingeschalten bleiben. 
Thévenin und Norton Quellen werden verschieden abgeschaltet. 
Eine Thévenin Quellen wird zu einer idealen Verbindung (nicht kurzgeschlossen!). 
Eine Norton Quelle wird zu einem idealen Unterbruch.

\begin{center}
    % LTeX: enabled=false
\begin{circuitikz}
    \draw (1,2) to[V,v,*-*,name=v1, bipole/is voltage=false] (1,0);
    \draw (0.5,0) -- (1.5,0);
    \draw (0.5,2) -- (1.5,2);
    \draw[-Latex,blue] (v1-Vfrom) to[bend left=55, blue]  node[midway, right] {$U_q$}(v1-Vto);

    \draw[-Latex,orange] (2.5,1) -- node[midway, above] {ausschalten} (3.5,1);

    \draw (5,2) to[short,*-*] (5,0);
    \draw (4.5,0) -- (5.5,0);
    \draw (4.5,2) -- (5.5,2);
\end{circuitikz}
\begin{circuitikz}
    \draw (1,0) to[I,i=$I_q$,*-*,name=i1, bipole/is voltage=false] (1,2);
    \draw (0.5,0) -- (1.5,0);
    \draw (0.5,2) -- (1.5,2);
    \iarr{i1};
    \draw[-Latex,orange] (2.5,1) -- node[midway, above] {ausschalten} (3.5,1);

    \draw (5,0.5) to[short,-*] (5,0);
    \draw (5,1.5) to[short,-*](5,2);
    \draw (4.5,0) -- (5.5,0);
    \draw (4.5,2) -- (5.5,2);
\end{circuitikz}
\end{center}

\subsubsection{Leistung bei Superposition}

Leistung darf während dem Superpositionieren nicht separat berechnet werden. 
Es müssen alle Quellen superponiert werden, summiert und dann darf erst Leistung berechnet werden. 
Ansonsten stimmt das Resultat nicht.

\subsubsection{Beispiel}

\begin{center}
    % LTeX: enabled=false
\begin{circuitikz}
    \tikzstyle{every node}=[font=\normalsize]
    \draw (0,0) to[american current source] (0,3);
    \draw (9,0) to[american current source] (9,3);
    \draw (3,3) to[american voltage source] (3,0);
    \draw (3,3) to[R,l={ \normalsize $3\Omega$}] (6,3);
    \draw (3,0) to[R,l={ \normalsize 3$\Omega$}] (6,0);
    \draw (6,0) to[R] (6,3);
    \draw (6,3) to[short] (9,3);
    \draw (0,3) to[short] (3,3);
    \draw (0,0) to[short] (3,0);
    \draw (6,0) to[short] (9,0);
    \node at (3,0) [circ] {};
    \node at (3,3) [circ] {};
    \node at (6,3) [circ] {};
    \node at (6,0) [circ] {};
    \node [font=\normalsize] at (6.5,1.5) {-3$\Omega$};
    \node [font=\normalsize] at (-1,1.5) {1A};
    \node [font=\normalsize] at (3.75,1.5) {3V};
    \draw [ color={rgb,255:red,255; green,0; blue,0}, short] (0.25,2) -- (0.25,2.75);
    \draw [ color={rgb,255:red,255; green,0; blue,0}, short] (1.5,2.75) -- (2.75,2.75);
    \draw [ color={rgb,255:red,255; green,0; blue,0}, short] (2.75,1) -- (2.75,0.25);
    \draw [ color={rgb,255:red,255; green,0; blue,0}, short] (1.25,0.25) -- (0.25,0.25);
    \draw [ color={rgb,255:red,255; green,0; blue,0}, ->, >=Stealth] (2.75,0.25) -- (1.25,0.25);
    \draw [ color={rgb,255:red,255; green,0; blue,0}, ->, >=Stealth] (0.25,0.25) -- (0.25,1);
    \draw [ color={rgb,255:red,255; green,0; blue,0}, ->, >=Stealth] (0.25,2.75) -- (1.5,2.75);
    \draw [ color={rgb,255:red,255; green,0; blue,0}, ->, >=Stealth] (2.75,2.75) -- (2.75,2);
    \node [font=\normalsize, color={rgb,255:red,255; green,0; blue,0}] at (1.25,3.5) {1A};
    \node [font=\normalsize] at (9.75,1.5) {1A};
    \draw [ color={rgb,255:red,185; green,0; blue,185}, short] (3.25,2) -- (3.25,3.25);
    \draw [ color={rgb,255:red,185; green,0; blue,185}, short] (3.25,3.25) -- (3.75,3.25);
    \draw [ color={rgb,255:red,185; green,0; blue,185}, short] (9.25,2) -- (9.25,3.25);
    \draw [ color={rgb,255:red,185; green,0; blue,185}, ->, >=Stealth] (5,3.25) -- (5.75,3.25);
    \draw [ color={rgb,255:red,185; green,0; blue,185}, ->, >=Stealth] (6.25,3.25) -- (6.25,2.25);
    \draw [ color={rgb,255:red,185; green,0; blue,185}, short] (6.25,0.75) -- (6.25,-0.25);
    \draw [ color={rgb,255:red,185; green,0; blue,185}, ->, >=Stealth] (6.25,-0.25) -- (5.25,-0.25);
    \draw [ color={rgb,255:red,185; green,0; blue,185}, short] (3.75,-0.25) -- (3.25,-0.25);
    \draw [ color={rgb,255:red,185; green,0; blue,185}, short] (3.25,-0.25) -- (3.25,1);
    \draw [ color={rgb,255:red,185; green,0; blue,185}, ->, >=Stealth] (6.25,-0.25) -- (7.25,-0.25);
    \draw [ color={rgb,255:red,185; green,0; blue,185}, short] (7.25,-0.25) -- (9.25,-0.25);
    \draw [ color={rgb,255:red,185; green,0; blue,185}, ->, >=Stealth] (9.25,-0.25) -- (9.25,1);
    \draw [ color={rgb,255:red,185; green,0; blue,185}, ->, >=Stealth] (9.25,3.25) -- (7.5,3.25);
    \draw [ color={rgb,255:red,185; green,0; blue,185}, short] (7.5,3.25) -- (5.75,3.25);
    \draw [ color={rgb,255:red,0; green,106; blue,0}, ->, >=Stealth] (3.5,2.75) -- (3.75,2.75);
    \draw [ color={rgb,255:red,0; green,106; blue,0}, ->, >=Stealth] (5.75,2.75) -- (5.75,2.25);
    \draw [ color={rgb,255:red,0; green,106; blue,0}, ->, >=Stealth] (5.75,0.25) -- (5.25,0.25);
    \draw [ color={rgb,255:red,0; green,106; blue,0}, ->, >=Stealth] (3.5,0.25) -- (3.5,1);
    \draw [ color={rgb,255:red,0; green,106; blue,0}, short] (5.75,0.25) -- (5.75,0.75);
    \draw [ color={rgb,255:red,0; green,106; blue,0}, short] (3.75,0.25) -- (3.5,0.25);
    \draw [ color={rgb,255:red,0; green,106; blue,0}, short] (3.5,2.75) -- (3.5,1.75);
    \draw [ color={rgb,255:red,0; green,106; blue,0}, short] (5.75,2.75) -- (5.25,2.75);
    \node [font=\normalsize, color={rgb,255:red,0; green,106; blue,0}] at (4,2.5) {1A};
    \node [font=\normalsize, color={rgb,255:red,0; green,106; blue,0}] at (5.25,0.5) {1A};
    \node [font=\normalsize, color={rgb,255:red,185; green,0; blue,185}] at (5.5,3.75) {1A};
    \node [font=\normalsize, color={rgb,255:red,185; green,0; blue,185}] at (6.75,2.5) {2A};
    \node [font=\normalsize, color={rgb,255:red,185; green,0; blue,185}] at (7.25,3.75) {1A};
    \node [font=\normalsize, color={rgb,255:red,185; green,0; blue,185}] at (5.5,-0.75) {1A};
\end{circuitikz}
\end{center}

In diesem Beispiel wurden alle Quellen nacheinander ausgeschaltet und die Ströme eingezeichnet. 
Nun können die Ströme summiert und die Leistungen bestimmt werden.
\subsection{Reziprozität /Kirchhoffscher Umkehrungssatz}


In einem Netzwerk mit nur einer realen Idealen Quelle kann die Quelle und eine Messgrösse ausgetauscht werden. 
Das Verhältnis bleibt gleich. 
Die Quellgrösse, welche im veränderten Netzwerk angelegt wird, ist aber nicht gleich.

\subsubsection{Herleitung mit Spannungsquelle}

\begin{multicols*}{2}
    \begin{center}
        % LTeX: enabled=false
\resizebox{\columnwidth}{!}{%
\begin{circuitikz}
    \tikzstyle{every node}=[font=\normalsize]
    \draw (0,3) to[american voltage source] (0,0);
    \draw (0,3) to[R] (3,3);
    \draw (3,3) to[R] (3,0);
    \draw (3,3) to[R] (6,3);
    \draw (6,3) to[R] (6,0);
    \draw (6,0) to[short] (0,0);
    \node at (3,3) [circ] {};
    \node at (3,0) [circ] {};
    \node [font=\normalsize] at (1.5,3.5) {$R_1$};
    \node [font=\normalsize] at (0.75,1.5) {$U_q$};
    \node [font=\normalsize] at (3.5,1.5) {$R_2$};
    \node [font=\normalsize] at (4.5,3.5) {$R_3$};
    \node [font=\normalsize] at (6.5,1.5) {$R_4$};
    \draw [ color={rgb,255:red,255; green,0; blue,0}, ->, >=Stealth] (6,0.5) -- (6,0.25);
    \node [font=\normalsize, color={rgb,255:red,255; green,0; blue,0}] at (6.25,0.25) {$I_1$};
\end{circuitikz}
}%
    \end{center}
    \begin{align*}
        I_1 &= \frac{U_q}{R_1 + (R_2 \parallel (R_3 + R_4))} \cdot \frac{R_2}{R_1 + R_3 + R_4}\\
        I_1 &= \frac{U_q R_2}{\left(R_1 + \frac{R_2(R_3 + R_4)}{R_1 + R_3 + R_4}\right)\left(R_2 + R_3 + R_4\right)}\\
        I_1 &= \frac{U_q R_2}{R_1(R_1 + R_3 + R_4) + R_2(R_3 + R_4)}
    \end{align*}
    \newcolumn

    \begin{center}
        % LTeX: enabled=false
\resizebox{\columnwidth}{!}{%
\begin{circuitikz}        
    \tikzstyle{every node}=[font=\LARGE]
    \draw (0,3) to[R] (3,3);
    \draw (3,3) to[R] (3,0);
    \draw (3,3) to[R] (6,3);
    \draw (6,3) to[R] (6,1.5);
    \draw (6,0) to[short] (0,0);
    \node at (3,3) [circ] {};
    \node at (3,0) [circ] {};
    \node [font=\normalsize] at (1.5,3.5) {$R_1$};
    \node [font=\normalsize] at (6.75,0.75) {$U_q$};
    \node [font=\normalsize] at (3.5,1.5) {$R_2$};
    \node [font=\normalsize] at (4.5,3.5) {$R_3$};
    \node [font=\normalsize] at (6.5,2.5) {$R_4$};
    \node [font=\normalsize, color={rgb,255:red,255; green,0; blue,0}] at (0.5,1.75) {$I_2$};
    \draw (0,0) to[short] (0,3);
    \draw [ color={rgb,255:red,255; green,0; blue,0}, ->, >=Stealth] (0,1.5) .. controls (0,1.5) and (0,1.5) .. (0,1.75) ;
    \draw (6,0) to[american voltage source] (6,1.5);
\end{circuitikz}
}%
    \end{center}
    \begin{align*}
        I_2 &= \frac{U_q}{(R_1 \parallel R_2) + R_3 + R_4 }\frac{R_2}{R_1 + R_2}\\
        I_2 &= \frac{U_q R_2}{\left(\frac{R_1 R_2}{R_1 + R_2} + R_3 + R_4 \right)(R_1 + R_2)}\\
        I_2 &= \frac{U_q R_2}{R_1 R_2 + R_3 (R_1 + R_2) + R_4(R_1 + R_2)}
    \end{align*}

\end{multicols*}
\begin{center}
    $R_1 R_2 + R_1 R_3 + R_1 R_4 + R_2 R_3 + R_2 R_4 \mathcolor{red}{=} R_1 R_2 + R_1 R_3 + R_1 R_4 + R_2 R_3 + R_2 R_4$
\end{center}

\subsubsection{Herleitung mit Stromquelle}

\begin{multicols*}{2}
    \begin{center}
        % LTeX: enabled=false
\resizebox{\columnwidth}{!}{%
\begin{circuitikz}
    \tikzstyle{every node}=[font=\normalsize]
    \draw (0,3) to[R] (3,3);
    \draw (3,3) to[R] (3,0);
    \draw (3,3) to[R] (6,3);
    \draw (6,3) to[R] (6,0);
    \draw (6,0) to[short] (0,0);
    \node at (3,3) [circ] {};
    \node at (3,0) [circ] {};
    \node [font=\normalsize] at (1.5,3.5) {$R_1$};
    \node [font=\normalsize] at (0.75,1.5) {$I_q$};
    \node [font=\normalsize] at (3.5,1.5) {$R_2$};
    \node [font=\normalsize] at (4.5,3.5) {$R_3$};
    \node [font=\normalsize] at (6.5,1.5) {$R_4$};
    \draw (0,0) to[american current source] (0,3);
    \draw [ color={rgb,255:red,162; green,162; blue,162}, line width=0.6pt, ->, >=Stealth] (1,2.5) .. controls (1.5,2.25) and (1.5,0.75) .. (1,0.5) node[pos=0.5, fill=white]{$U_q$};
    \draw (6,3) to[short, -o] (6.25,3) ;
    \draw (6,0) to[short, -o] (6.25,0) ;
    \draw [ color={rgb,255:red,0; green,128; blue,255}, line width=0.6pt, ->, >=Stealth] (6.75,2.5) .. controls (7.25,2.25) and (7.25,0.75) .. (6.75,0.5) node[pos=0.5, fill=white]{$U_1$};
\end{circuitikz}
}%
    \end{center}
    \[U_1 = I_q \frac{R_2}{R_2 + R_3 + R_4} R_4\]
    \newcolumn

    % LTeX: enabled=false
\resizebox{\columnwidth}{!}{%
\begin{circuitikz}
    \tikzstyle{every node}=[font=\normalsize]
    \draw (0,3) to[R] (3,3);
    \draw (3,3) to[R] (3,0);
    \draw (3,3) to[R] (6,3);
    \draw (6,3) to[R] (6,0);
    \draw (6,0) to[short] (0,0);
    \node at (3,3) [circ] {};
    \node at (3,0) [circ] {};
    \node [font=\normalsize] at (1.5,3.5) {$R_1$};
    \node [font=\normalsize] at (9.75,1.5) {$I_q$};
    \node [font=\normalsize] at (3.5,1.5) {$R_2$};
    \node [font=\normalsize] at (4.5,3.5) {$R_3$};
    \node [font=\normalsize] at (6.5,1.5) {$R_4$};
    \draw [ color={rgb,255:red,162; green,162; blue,162}, line width=0.6pt, ->, >=Stealth] (10,2.5) .. controls (10.5,2.25) and (10.5,0.75) .. (10,0.5) node[pos=0.5, fill=white]{$U_q'$};
    \draw [ color={rgb,255:red,0; green,128; blue,255}, line width=0.6pt, ->, >=Stealth] (0,0.5) .. controls (-0.5,0.75) and (-0.5,2.25) .. (0,2.5) node[pos=0.5, fill=white]{$U_2$};
    \draw (9,3) to[american current source] (9,0);
    \draw (6,3) to[short] (9,3);
    \draw (9,0) to[short] (6,0);
    \node at (6,0) [circ] {};
    \node at (6,3) [circ] {};
\end{circuitikz}
}%
    \[U_2 = I_q \frac{R_4}{R_2 + R_3 + R_4} R_2\]

\end{multicols*}
\begin{center}
    $U_1 = U_2$\\
    \begin{tcolorbox}[colframe=red , colback=white, arc is curved, hbox]
        $U_q \neq U_q'$
    \end{tcolorbox}
\end{center}


\subsection{Ideale Quellen verschieben}\label{Quelln verschieben}

Ideale Quellen Lassen sich verschieben. 
Für das Systematische lösen von Netzwerken kann das von Vorteil sein. 
Allerdings ist es immer ein tradeof. 
Man verfälscht und verliert immer Potenziale oder Ströme. 
Mann kann sich jedoch zu Beginn eine Formel zurechtlegen, falls diese wieder interessant werden. 
Das Verfahren für Strom und Spannungsquelle ist prinzipiell das gleiche.


\subsubsection{Ideale Stromquellen}

Ein virtueller Strom wird über die vorhandene Quelle gelegt. 
Der Quellstrom wird in der Summe null. 
Da der Strom jedoch nicht einfach so verschwindet müssen neue Stromquellen hinzugefügt werden, welche den Strom im Kreis führen:

\begin{center}
    % LTeX: enabled=false
\begin{circuitikz}[scale=0.75]
    \tikzstyle{every node}=[font=\normalsize]
    \draw (0,-0.75) to[R] (3,-0.75);
    \draw (3,-0.75) to[R] (3,2.25);
    \draw (3,2.25) to[R] (0,2.25);
    \draw (0,2.25) to[short, -o] (-0.25,2.25) ;
    \draw (0,2.25) to[short, -o] (0,2.5) ;
    \draw (3,2.25) to[short, -o] (3,2.5) ;
    \draw (3,2.25) to[short, -o] (3.25,2.25) ;
    \draw (3,-0.75) to[short, -o] (3.25,-0.75) ;
    \draw (3,-1) to[short, o-] (3,-0.75) ;
    \draw (0,-1) to[short, o-] (0,-0.75) ;
    \draw (0,-0.75) to[short, -o] (-0.25,-0.75) ;
    \draw [ color={rgb,255:red,255; green,0; blue,0} ](0,2.25) to[american current source,l={ \normalsize I}] (0,-0.75);
    \draw [->, >=Stealth] (4,0.75) -- (6,0.75);
    \draw (8,-0.75) to[R] (11,-0.75);
    \draw (11,-0.75) to[R] (11,2.25);
    \draw (11,2.25) to[R] (8,2.25);
    \draw (8,2.25) to[short, -o] (7.75,2.25) ;
    \draw (8,2.25) to[short, -o] (8,2.5) ;
    \draw (11,2.25) to[short, -o] (11,2.5) ;
    \draw (11,2.25) to[short, -o] (11.25,2.25) ;
    \draw (11,-0.75) to[short, -o] (11.25,-0.75) ;
    \draw (11,-1) to[short, o-] (11,-0.75) ;
    \draw (8,-1) to[short, o-] (8,-0.75) ;
    \draw (8,-0.75) to[short, -o] (7.75,-0.75) ;
    \draw [ color={rgb,255:red,255; green,0; blue,0} ](8,2.25) to[american current source,l={ \normalsize I}] (8,-0.75);
    \node at (3,2.25) [circ] {};
    \node at (0,2.25) [circ] {};
    \node at (3,-0.75) [circ] {};
    \node at (0,-0.75) [circ] {};
    \node at (8,2.25) [circ] {};
    \node at (11,2.25) [circ] {};
    \node at (11,-0.75) [circ] {};
    \node at (8,-0.75) [circ] {};
    \draw [ color={rgb,255:red,200; green,0; blue,150} ](12,1.75) to[american current source,l={ \normalsize I}] (12,-0.25);
    \draw [ color={rgb,255:red,200; green,0; blue,150} ](10.5,-1.75) to[american current source,l={ \normalsize I}] (8.5,-1.75);
    \draw [ color={rgb,255:red,200; green,0; blue,150} ](7,-0.25) to[american current source,l={ \normalsize I}] (7,1.75);
    \draw [ color={rgb,255:red,200; green,0; blue,150} ](8.5,3.25) to[american current source,l={ \normalsize I}] (10.5,3.25);
    \draw [ color={rgb,255:red,200; green,0; blue,150}, ](7,1.75) to[short] (8,1.75);
    \draw [ color={rgb,255:red,200; green,0; blue,150}, ](8.5,3.25) to[short] (8.5,2.25);
    \draw [ color={rgb,255:red,200; green,0; blue,150}, ](10.5,3.25) to[short] (10.5,2.25);
    \draw[ color={rgb,255:red,200; green,0; blue,150}, ] (12,1.75) to[short] (11,1.75);
    \draw[ color={rgb,255:red,200; green,0; blue,150}, ] (12,-0.25) to[short] (11,-0.25);
    \draw [ color={rgb,255:red,200; green,0; blue,150}, ](10.5,-1.75) to[short] (10.5,-0.75);
    \draw [ color={rgb,255:red,200; green,0; blue,150}, ](8.5,-1.75) to[short] (8.5,-0.75);
    \draw [ color={rgb,255:red,200; green,0; blue,150}, ](7,-0.25) to[short] (8,-0.25);
    \node at (8,1.75) [circ] {};
    \node at (8.5,2.25) [circ] {};
    \node at (10.5,2.25) [circ] {};
    \node at (11,1.75) [circ] {};
    \node at (10.5,-0.75) [circ] {};
    \node at (8.5,-0.75) [circ] {};
    \node at (8,-0.25) [circ] {};
    \draw [ color={rgb,255:red,255; green,128; blue,64} , dashed] (6,2) rectangle  (9,-0.5);
    \node [font=\normalsize] at (6.5,2.5) {$\sum = 0$};
    \node [font=\normalsize] at (4.75,1) {Verschieben};
\end{circuitikz}
\end{center}


Das Netzwerk hat nun eine möglicherweise mühsame Stromquelle weniger, allerdings generell mehr. 
Am besten schiebt man diese auf eine ideale Spannungsquelle, um diese ebenso zu eliminieren.


\subsubsection{Ideale Spannungsquellen}

Bei Idealen Spannungsquellen gilt dasselbe wie für ideale Stromquellen. 
Sie werden über einen Knoten hinweg geschoben. 
Mann kann sich das auch als zwingenden Ausgleich durch das Entfernen der ersten Spannungsquelle vorstellen.

\begin{center}
    % LTeX: enabled=false
\begin{circuitikz}[scale=0.6]
    \tikzstyle{every node}=[font=\LARGE]
    \draw (1,3.25) to[R] (4,3.25);
    \draw (6,3.25) to[R] (9,3.25);
    \draw (5,4.25) to[R] (5,7.25);
    \draw (5,4.25) to[short] (5,2.25);
    \draw (6,3.25) to[short] (4,3.25);
    \draw (1,3.25) to[short, -o] (0.75,3.25) ;
    \draw (5,7.25) to[short, -o] (5,7.5) ;
    \draw (9,3.25) to[short, -o] (9.25,3.25) ;
    \draw [ color={rgb,255:red,0; green,128; blue,0}, ](5,-1) to[short, o-] (5,-0.75) ;
    \draw [ color={rgb,255:red,0; green,0; blue,255} ](5,2.25) to[american voltage source] (5,-0.75);
    \draw [ color={rgb,255:red,0; green,128; blue,0}, ](5,-0.75) to[short] (5,0.25);
    \node at (5,3.25) [circ] {};
    \draw (12.75,3.25) to[R] (14.25,3.25);
    \draw (17.75,3.25) to[R] (19,3.25);
    \draw (16,5) to[R] (16,6.5);
    \draw (16,3.5) to[short] (16,3);
    \draw[ color={rgb,255:red,0; green,128; blue,0}, ] (16.25,3.25) to[short] (15.75,3.25);
    \draw (12,3.25) to[short, -o] (11.75,3.25) ;
    \draw (16,7.25) to[short, -o] (16,7.5) ;
    \draw (20,3.25) to[short, -o] (20.25,3.25) ;
    \draw [ color={rgb,255:red,0; green,128; blue,0}, ](16,-1) to[short, o-] (16,-0.75) ;
    \draw [ color={rgb,255:red,0; green,0; blue,255} ](16,1.5) to[american voltage source] (16,0);
    \draw [ color={rgb,255:red,0; green,128; blue,0}, ](16,-0.75) to[short] (16,0.25);
    \node at (16,3.25) [circ, color={rgb,255:red,0; green,128; blue,0}] {};
    \draw [ color={rgb,255:red,128; green,0; blue,64} ](16,5) to[american voltage source] (16,3.5);
    \draw (16,6.5) to[short] (16,7.25);
    \draw (12,3.25) to[short] (12.75,3.25);
    \draw (19,3.25) to[short] (20,3.25);
    \draw [ color={rgb,255:red,128; green,0; blue,64} ](17.75,3.25) to[american voltage source] (16.25,3.25);
    \draw [ color={rgb,255:red,128; green,0; blue,64} ](14.25,3.25) to[american voltage source] (15.75,3.25);
    \draw [ color={rgb,255:red,128; green,0; blue,64} ](16,1.5) to[american voltage source] (16,3);
    \draw [->, >=Stealth] (9.75,3.25) -- (11.25,3.25);
    \node [font=\normalsize] at (10.5,3.75) {Verschieben};
    \node [font=\normalsize] at (5.25,-0.5) {$\varphi$};
    \node [font=\normalsize] at (16.25,-0.5) {$\varphi$};
    \node [font=\normalsize] at (16.25,3) {$\varphi$};
    \draw [ dashed] (15.5,2.75) rectangle  (16.5,0.25);
    \node [font=\LARGE] at (17.5,2) {$\sum = 0 $};
\end{circuitikz}
\end{center}

Note: Das Potenzial $\varphi$ hat sich nach oben verschoben.

\subsection{Widerstände}
\begin{wrapfigure}{l}{0.1\linewidth}
    \begin{circuitikz}
        \draw(0,0) to[R,o-o,l=R] (0,1.5);
    \end{circuitikz}
    \begin{circuitikz}[american resistors]
        \draw(0,0) to[R,o-o,l=R] (0,1.5);
    \end{circuitikz}
\end{wrapfigure}

Fixe, lineare Widerstände haben typischerweise die links stehenden Schaltungssymbole.\\
Links das Europäische(mehr gebräuchlich), rechts das Amerikanische.\\
Es gibt verschieden elementare Schaltungen welche einfacher separat analysiert werden können.\\

\subsubsection{Serienschaltung}

Wenn Widerstände in Serie zueinander stehen, können sie summiert werden.\\
\begin{center}
    % LTeX: enabled=false
\begin{circuitikz}
    \draw (0,0) to[R,l=R1,o-] (2,0)
    to[R,l=R2] (4,0)
    to[R,l=R3] (6,0)
    to[R,l=R4,-o] (8,0);
\end{circuitikz}
\end{center}
    
\[ \text{R}_\text{tot} = \sum_{\text{i}=1}^{\text{n}} \text{R}_\text{i}\]

\subsubsection{Paralellschaltung}

Wenn Widerstände parallel zueinander stehen, können ihre Leitwerte zusammengezählt werden.\\

\begin{center}
    % LTeX: enabled=false
\begin{circuitikz}
    \draw (0.5,2) to[R,*-*,l=R1] (0.5,0);
    \draw (1.5,2) to[R,*-*,l=R2] (1.5,0);
    \draw (2.5,2) to[R,*-*,l=R3] (2.5,0);
    \draw (3.5,2) to[R,*-*,l=R4] (3.5,0);
    \draw (0,0) to[short,o-o] (4,0);
    \draw (0,2) to[short,o-o] (4,2);
\end{circuitikz}
\end{center}

\[ \text{R}_\text{tot} = \frac{1}{\sum_{\text{i}=1}^{\text{n}} \frac{1}{\text{R}_\text{i}}} \]

Wenn \textbf{nur 2} Widerstände paralell stehen, kann folgende vereinfachte Formel verwendet werden: $\frac{\text{R1}\cdot\text{R2}}{\text{R1}+\text{R2}}$\\

\subsubsection{Stern - dreieck umwandlung}

Wenn Widerstände in eine Dreieckskonstellation sind, kann man sie in einen Stern und umgekehrt umwandeln.\\

\begin{minipage}[b]{0.55\linewidth}
    \centering

    % LTeX: enabled=false
\begin{circuitikz}[scale=0.75]
    \draw (0,0) to[R,l=$\text{R}_\text{B}$,o-*] (1.5,1.299);
    \draw (3,0) to[R,l=$\text{R}_\text{C}$,o-*] (1.5,1.299);
    \draw (1.5,2.598) to[R,l=$\text{R}_\text{A}$,o-*] (1.5,1.299);
    \node at(0,0) [left]{B};
    \node at(1.5,2.598) [above]{A};
    \node at(3,0) [right]{C};
    \draw[-Latex] (3,1.299) -- node[midway, above] {umformen} (4,1.299);
\end{circuitikz}
\begin{circuitikz}[scale=0.75]
    \draw (0,0) to[R,l=$\text{R}_\text{AB}$,o-o] (1.5,2.598);
    \draw (0,0) to[R,l=$\text{R}_\text{BC}$,o-o] (3,0);
    \draw (1.5,2.598) to[R,l=$\text{R}_\text{BC}$,o-o] (3,0);
    \node at(0,0) [left]{B};
    \node at(1.5,2.598) [above]{A};
    \node at(3,0) [right]{C};
\end{circuitikz}

\end{minipage}%%
\begin{minipage}[b]{0.45\linewidth}
    \centering
    \[
    \begin{aligned}
        \text{R}_\text{AB} &= \text{R}_\text{A} + \text{R}_\text{B} + \frac{\text{R}_\text{A}\cdot\text{R}_\text{B}}{\text{R}_\text{C}}\\
        \text{R}_\text{Ac} &= \text{R}_\text{A} + \text{R}_\text{C} + \frac{\text{R}_\text{A}\cdot\text{R}_\text{C}}{\text{R}_\text{B}}\\
        \text{R}_\text{cB} &= \text{R}_\text{B} + \text{R}_\text{C} + \frac{\text{R}_\text{B}\cdot\text{R}_\text{C}}{\text{R}_\text{A}}
    \end{aligned}
    \]

    \vspace{4ex}
\end{minipage} 
\begin{minipage}[b]{0.55\linewidth}
    \centering

    \input{circuits/stern zu dreiek.tex}

    \vspace{4ex}
\end{minipage}%% 
\begin{minipage}[b]{0.45\linewidth}
    \centering
    \[
    \begin{aligned}
        \text{R}_\text{A} &= \frac{\text{R}_\text{AB}\cdot\text{R}_\text{AC}}{\text{R}_\text{AB}+\text{R}_\text{BC}+\text{R}_\text{AC}}\\
        \text{R}_\text{B} &= \frac{\text{R}_\text{AB}\cdot\text{R}_\text{BC}}{\text{R}_\text{AB}+\text{R}_\text{BC}+\text{R}_\text{AC}}\\
        \text{R}_\text{C} &= \frac{\text{R}_\text{AC}\cdot\text{R}_\text{BC}}{\text{R}_\text{AB}+\text{R}_\text{BC}+\text{R}_\text{AC}}
    \end{aligned}
    \]

    \vspace{4ex}
\end{minipage} 


\subsection{Symetrien in Widerstandsnetzwerken}

Symmetrien können beim Lösen von Widerstandsnetzwerken nützlich sein da komplette Netzwerke nur einmal zusammengefasst werden müssen. 
Dabei ist das Äquipotentialprinzip sehr wichtig da ein Widerstand, der auf der Symmetrielinie liegt, aufgeteilt werden darf. 
Dabei ist es egal ob in paralell oder in serie. 
Voraussetzung ist das Gleiche Widerstandswerte vorligen.

% LTeX: enabled=false
\begin{circuitikz}[scale=0.75]
    \tikzstyle{every node}=[font=\normalsize]
    \draw (0,2.25) to[R,l={ \normalsize R3}] (0,0.25);
    \draw (4,2.25) to[R,l={ \normalsize R4}] (4,0.25);
    \draw (2,4.25) to[R,l={ \normalsize R5}] (2,2.25);
    \draw (0.5,6.25) to[R,l={ \normalsize R1}] (0.5,4.25);
    \draw (3.5,6.25) to[R,l={ \normalsize R2}] (3.5,4.25);
    \draw (0.5,4.25) to[short] (3.5,4.25);
    \draw (0.5,6.25) to[short] (3.5,6.25);
    \draw (4,2.25) to[short] (0,2.25);
    \draw (0,0.25) to[short] (4,0.25);
    \draw [ color={rgb,255:red,255; green,128; blue,0}, dashed] (2,7.25) -- (2,-0.75);
    \node [font=\large] at (2.5,7.5) {symetrie Linie};
    \node [font=\LARGE] at (2.5,7.5) {};
    \draw (0.5,6.25) to[short, -o] (-0.5,6.25) ;
    \draw (0,0.25) to[short, -o] (-1,0.25) ;
    \node at (0.5,6.25) [circ] {};
    \node at (3.5,6.25) [circ] {};
    \node at (0,0.25) [circ] {};
    \node at (2,2.25) [circ] {};
    \draw [->, >=Stealth] (5.25,3.25) -- (7,3.25);
    \node at (2,4.25) [circ] {};
    \draw (7.5,2.25) to[R,l={ \normalsize R6}] (7.5,0.25);
    \draw (11.5,2.25) to[R,l={ \normalsize R7}] (11.5,0.25);
    \draw [ color={rgb,255:red,255; green,0; blue,0} ](8.5,4.25) to[R,l={ \normalsize R5,5}] (8.5,2.25);
    \draw (8,6.25) to[R,l={ \normalsize R1}] (8,4.25);
    \draw (11,6.25) to[R,l={ \normalsize R2}] (11,4.25);
    \draw (8,4.25) to[short] (11,4.25);
    \draw (8,6.25) to[short] (11,6.25);
    \draw (11.5,2.25) to[short] (7.5,2.25);
    \draw (7.5,0.25) to[short] (11.5,0.25);
    \draw [ color={rgb,255:red,255; green,128; blue,0}, dashed] (9.5,7.25) -- (9.5,-0.75);
    \node [font=\large] at (10,7.5) {symetrie Linie};
    \node [font=\LARGE] at (10,7.5) {};
    \draw (8,6.25) to[short, -o] (7,6.25) ;
    \draw (7.5,0.25) to[short, -o] (6.5,0.25) ;
    \node at (8,6.25) [circ] {};
    \node at (7.5,0.25) [circ] {};
    \node at (8.5,2.25) [circ] {};
    \node at (8.5,4.25) [circ] {};
    \draw [ color={rgb,255:red,255; green,0; blue,0} ](10.5,4.25) to[R,l={ \normalsize R6,5}] (10.5,2.25);
    \node at (10.5,4.25) [circ] {};
    \node at (10.5,2.25) [circ] {};
    \node [font=\normalsize] at (13,3.25) {(2 x R Wert)};
\end{circuitikz}

Nun können beide Seiten separat angeschaut werden. 
Da beide Seiten identisch sind, kann nur eine Seite berechnet und der resultierende Wert halbiert werden. 
        \section{Nichtlineare Netzwerke / Leistungsanpassung}



\begin{wrapfigure}{l}{0.1\linewidth}
    \begin{circuitikz}
        \draw (0,0) to[R] (0,1.5);
        \draw [short] (0.5,1.5) -- (0.5,1);
        \draw [short] (0.5,1) -- (-0.5,0.5);
        \draw [short] (-0.5,0.5) -- (-0.5,0);   
    \end{circuitikz}     
\end{wrapfigure}

Viele Netzwerke sind in der Praxis nicht linear.\\
Meistens wird dabei ein nicht linearen Widerstand verwendet, um diesen Effekt zu modellieren.\\
Das links stehende Schemasymbol wird dabei verwendet.\\
Die Änderung des Widerstandswertes durch einen Effekt wie zum Beispiel Temperatur, licht oder mechanische Einwirkung herbeigeführt wird können Symbole diesen Effekt signalisieren. 
Symbole wie z.B. Pfeile oder ein Temperatursymbol.\\ 


\subsection{Leistungsanpassung}

Leistungsanpassung beschreibt das Verfahren ein Widerstand an einer realen Quelle anzupassen, sodass die maximal mögliche Energie aus der Quelle fliesst. 
Im Lineraren Fall entspricht der Lastwiderstand dem Innenwiderstand der Quelle. 

\begin{multicols*}{2}
    
    % LTeX: enabled=false
\begin{circuitikz}
    \tikzstyle{every node}=[font=\normalsize]
    \draw (0,2) to[american voltage source,l={ \normalsize $U_0$}] (0,0);
    \draw (0,2) to[R,l={ \normalsize $R_i$}] (2,2);
    \draw (2,2) to[short, -o] (2.5,2) ;
    \draw (2,0) to[short, -o] (2.5,0) ;
    \draw (2,0) to[short] (0,0);
    \draw [ color={rgb,255:red,115; green,0; blue,230} ](3.25,2) to[R,l={ \normalsize Rl}] (3.25,0);
    \draw[ color={rgb,255:red,115; green,0; blue,230}, ] (3.25,2) to[short] (2.5,2);
    \draw [ color={rgb,255:red,115; green,0; blue,230}, ](2.5,0) to[short] (3.25,0);
    \draw [ color={rgb,255:red,115; green,0; blue,230}, ->, >=Stealth] (2.75,0.5) -- (3.75,1.5);
    \draw [ color={rgb,255:red,0; green,0; blue,255}, short] (2.5,1.75) -- (4.5,1.5);
    \draw [ color={rgb,255:red,0; green,0; blue,255}, short] (2.5,1.75) -- (2.5,0.25);
    \draw [ color={rgb,255:red,0; green,0; blue,255}, short] (2.5,0.25) -- (4.5,0.5);
    \draw [ color={rgb,255:red,0; green,0; blue,255}, short] (4.5,0.5) -- (4.5,0.25);
    \draw [ color={rgb,255:red,0; green,0; blue,255}, short] (4.5,1.5) -- (4.5,1.75);
    \draw [ color={rgb,255:red,0; green,0; blue,255}, short] (4.5,1.75) -- (5.25,1);
    \draw [ color={rgb,255:red,0; green,0; blue,255}, short] (5.25,1) -- (4.5,0.25);
    \node [font=\normalsize, color={rgb,255:red,0; green,0; blue,255}] at (4.75,1) {Pl};
\end{circuitikz}  

    \begin{center}
        \begin{tcolorbox}[colframe=violet , colback=white, arc is curved, hbox]
            $R_l^\ast = R_i$
        \end{tcolorbox}
        Maximale leistung, $\eta=50\%$\\
        Der Stern am $R_l$ signalisiert, dass Leistungsanpassung gemacht wurde. 
    \end{center}


    \newcolumn
    % LTeX: enabled=false
\begin{tikzpicture}
    \draw [->, >=Stealth] (0,0) -- (0,5.5);
    \draw [->, >=Stealth] (0,0) -- (5.5,0);
    \node [font=\normalsize] at (-0.5,3.75) {$\frac{3U_0}{4}$};
    \node [font=\normalsize] at (-0.5,2.5) {$\frac{U_0}{2}$};
    \node [font=\normalsize] at (-0.5,1.25) {$\frac{U_0}{4}$};
    \draw [ color={rgb,255:red,0; green,128; blue,0}, short] (0,0) .. controls (0.75,5) and (4.25,5) .. (5,0);
    \draw [short] (0,2.5) -- (-0.25,2.5);
    \draw [short] (0,1.25) -- (-0.25,1.25);
    \draw [short] (0,3.75) -- (-0.25,3.75);
    \draw [short] (0,5) -- (-0.25,5);
    \draw [short] (1.25,0) -- (1.25,-0.25);
    \draw [short] (2.5,0) -- (2.5,-0.25);
    \draw [short] (3.75,0) -- (3.75,-0.25);
    \draw [short] (5,0) -- (5,-0.25);
    \draw [dashed] (0,5) -- (5,0);
    \node [font=\normalsize] at (2.5,-0.5) {$\frac{I_0}{2}$};
    \node [font=\normalsize] at (1.25,-0.5) {$\frac{I_0}{4}$};
    \node [font=\normalsize] at (3.75,-0.5) {$\frac{3I_0}{4}$};
    \node [font=\normalsize] at (5,-0.5) {$I_0$};
    \node [font=\normalsize, color={rgb,255:red,255; green,0; blue,0}] at (5.75,0) {$I$};
    \node [font=\normalsize, color={rgb,255:red,0; green,0; blue,255}] at (-0.25,5.75) {$U$};
    \node [font=\normalsize] at (-0.5,5) {$U_0$};
    \draw [ color={rgb,255:red,255; green,128; blue,0}, dashed] (0,3.75) -- (1.25,3.75);
    \draw [ color={rgb,255:red,255; green,128; blue,0}, dashed] (1.25,3.75) -- (1.25,0);
    \draw [ color={rgb,255:red,255; green,0; blue,0}, dashed] (0,2.5) -- (2.5,2.5);
    \draw [ color={rgb,255:red,255; green,0; blue,0}, dashed] (2.5,2.5) -- (2.5,0);
    \draw [ color={rgb,255:red,162; green,162; blue,162}, dashed] (0,1.25) -- (3.75,1.25);
    \draw [ color={rgb,255:red,162; green,162; blue,162}, dashed] (3.75,1.25) -- (3.75,0);
    \draw [short] (2.5,2.5) -- (2.5,3.75);
    \node [font=\normalsize, color={rgb,255:red,128; green,0; blue,64}] at (3.5,4) {Maximum!};
    \draw [ color={rgb,255:red,128; green,0; blue,128} , dashed] (2.5,3.75) circle (0.25cm);
\end{tikzpicture}
\end{multicols*}

\subsubsection{Herleitung des optimalen Lastwiderstands einer linearen Stromquelle}

\begin{center}
    % LTeX: enabled=false
\begin{circuitikz}
    \tikzstyle{every node}=[font=\normalsize]
    \draw (2,0) to[R,l={ \normalsize $R_i$}] (2,2);
    \draw (2,2) to[short, -o] (2.5,2) ;
    \draw (2,0) to[short, -o] (2.5,0) ;
    \draw (2,0) to[short] (0,0);
    \draw [ color={rgb,255:red,115; green,0; blue,230} ](3.25,2) to[R,l={ \normalsize Rl}] (3.25,0);
    \draw[ color={rgb,255:red,115; green,0; blue,230}, ] (3.25,2) to[short] (2.5,2);
    \draw [ color={rgb,255:red,115; green,0; blue,230}, ](2.5,0) to[short] (3.25,0);
    \draw [ color={rgb,255:red,115; green,0; blue,230}, ->, >=Stealth] (2.75,0.5) -- (3.75,1.5);
    \draw (0,0) to[american current source, l = $I$] (0,2);
    \draw (0,2) to[short] (2,2);
    \node at (2,0) [circ] {};
    \node at (2,2) [circ] {};
\end{circuitikz}
\end{center}

\begin{align*}
    U_{RL} & = \frac{R_i R_l I}{R_i + R_l}\\
    P_{R_l} & = \frac{1}{R_L}(U(R_l))^2\\
    0 & = \frac{d P(R_l)}{d R_l}\\
    0 & = R_l' (U(R_l))^2 + 2\frac{1}{R_L}(U(R_l)')\\
    0 & = -\frac{1}{R_l^2}\frac{R_i^2 R_l^2 I^2}{(R_i + R_l)^2 } + 2\frac{1}{R_l} \frac{R_i R_l I}{R_i + R_l}\left(\frac{R_i I(R_i + R_L)- R_i R_lI}{(R_i + R_l)^2}\right)\\
    0 & = 2 \frac{R_i^3 I^2}{(R_i + R_l)^3} - \frac{R_i^2  I^2}{(R_i + R_l)^2 }\\
    \frac{R_i^2  I^2}{(R_i + R_l)^2 } & = 2 \frac{R_i^3 I^2}{(R_i + R_l)^3}\\
    R_i^2 I^2 (R_i + R_l) & = 2 R_i^3 I^2\\
    2R_I & = R_i + R_l\\
    R_i &= R_l
\end{align*}

Im falle eines nicht linearen Widerstands muss eine Formel für die Leistung über dem Lastwiderstand gefunden werden. 
Die gefundene Formel muss dann abgeleitet und null gesetzt werden, ähnlich wie in der Herleitung.

        \section{Systematische Netzwerkanalyse}

{\color{red}Wichtiger Grundsatz}\\
Das zu analysierende Netzwerk \textbf{MUSS} linear sein. 
Ansonsten kann das Netzwerk nicht mit den folgenden Methoden analysiert werden.
Grundsätzlich solle das zu analysierende Netzwerk immer zuerst maximal vereinfacht werden da dies meist zu weniger Aufwand führt. 
Das schlussendliche Ziel ist es eine Matrixgleichung zu erhalten, welche man zuletzt invertieren muss.

\subsection{Auswahl einer geeigneten Methode}

Meisten ist die Wahl der Methode frei. 
Um die optimale zu wählen, sollte immer der Aufwand von allen Methoden bestimmt werden und aufgrund dessen entschieden werden.

\subsection{Graphentheorie}

Um elektrische Netzwerke zu analysieren wird ein mathematisches Teilgebiet namens Graphentheorie angewendet.
Für systematische Netzwerkanalyse müssen ein paar Grundbegriffe geläufig sein:

\begin{itemize}
    \item \textbf{Knoten} : Ein Punkt in welchem Zweige enden
    \item \textbf{Zweig/Kante} : Verbindet Knoten untereinander
    \item \textbf{Weg} : Eine zusammenhängende Reihe von Knoten und zweigen (keine Kante doppelt)
    \item \textbf{Baum} : Ein Weg welcher nirgends in einem Kreis geschlossen ist (sehr bildlich)
    \item \textbf{Spannbaum} : Ein weg welcher alle Knoten miteinander verbindet
    \item \textbf{Zyklus} : Ein weg welcher im gleichen Knoten beginnt und endet
    \item \textbf{Kreis} : Ein Zyklus welcher keinen Knoten zweimal enthält
    \item \textbf{Masche} : Ein Kreis welche Keine inneren oder äussere Zweige aufweisst.
\end{itemize}

\begin{center}
    % LTeX: enabled=false
\begin{circuitikz}
    \tikzstyle{every node}=[font=\normalsize]
    \draw (6.25,10.75) to[short] (6.25,11.25);
    \node at (1.75,10.75) [circ] {};
    \node at (2.75,10.5) [circ] {};
    \node at (2,10.25) [circ] {};
    \node at (2.5,11.25) [circ] {};
    \node [font=\normalsize] at (2,12) {Knoten};
    \node [font=\normalsize] at (4.5,12) {Zweig /Kante};
    \node [font=\normalsize] at (7,12) {Weg (rot)};
    \node [font=\normalsize] at (1.75,8.5) {};
    \node [font=\normalsize] at (9.5,12) {spannbaum (rot)};
    \draw (4.25,11) to[short] (4.25,9.75);
    \node at (4.25,11) [circ] {};
    \node at (4.25,9.75) [circ] {};
    \draw (7.75,10.75) to[short] (7.75,9.75);
    \draw[ color={rgb,255:red,255; green,0; blue,0}, ] (7.75,9.75) to[short] (6.25,9.75);
    \draw (6.25,10.75) to[short] (6.25,9.75);
    \draw [ color={rgb,255:red,255; green,0; blue,0}, ](6.25,10.75) to[short] (7.75,10.75);
    \draw [ color={rgb,255:red,255; green,0; blue,0}, ](7.75,10.75) to[short] (7.75,9.75);
    \node at (6.25,10.75) [circ, color={rgb,255:red,255; green,0; blue,0}] {};
    \node at (7.75,10.75) [circ, color={rgb,255:red,255; green,0; blue,0}] {};
    \draw (6.25,11.25) to[short] (7.75,11.25);
    \draw (7.75,11.25) to[short] (7.75,10.75);
    \node at (6.25,11.25) [circ] {};
    \draw (6.25,11.25) to[short] (5.75,11.25);
    \draw (5.75,11.25) to[short] (5.75,9.75);
    \draw (5.75,9.75) to[short] (6.25,9.75);
    \node at (6.25,9.75) [circ, color={rgb,255:red,255; green,0; blue,0}] {};
    \draw (10.5,10.75) to[short] (10.5,9.75);
    \draw (10.5,9.75) to[short] (9,9.75);
    \draw [ color={rgb,255:red,255; green,0; blue,0}, ](9,10.75) to[short] (9,9.75);
    \draw [ color={rgb,255:red,255; green,0; blue,0}, ](9,10.75) to[short] (10.5,10.75);
    \draw (10.5,10.75) to[short] (10.5,9.75);
    \node at (9,10.75) [circ, color={rgb,255:red,255; green,0; blue,0}] {};
    \node at (10.5,10.75) [circ, color={rgb,255:red,255; green,0; blue,0}] {};
    \draw [ color={rgb,255:red,255; green,0; blue,0}, ](9,10.75) to[short] (9,11.25);
    \draw (9,11.25) to[short] (10.5,11.25);
    \draw (10.5,11.25) to[short] (10.5,10.75);
    \node at (9,11.25) [circ] {};
    \draw (9,11.25) to[short] (8.5,11.25);
    \draw (8.5,11.25) to[short] (8.5,9.75);
    \draw (8.5,9.75) to[short] (9,9.75);
    \node at (9,9.75) [circ, color={rgb,255:red,255; green,0; blue,0}] {};
\end{circuitikz}  
\end{center}

Anzahl an Elementen wird wie folgt angegeben:
\begin{itemize}
    \item Zyklen : z
    \item Knoten : k
    \item Maschen M = z - (k - 1) = z - k + 1
    \item Ideale (nicht lineare) Spannungsquellen : v
    \item Ideale (nicht lineare) Stromquellen : i
\end{itemize}

\subsection{Zweigstrommethode}

Bei der Zweigstrommethode wird das Ohmsche und die beiden krichhoffschen Gesetze systematisch angewendet. 
3 Gruppen an Gleichungen entstehen.

\subsubsection{Zweiggleichungen}

Anzahl: z - i - v\\
Ohmsches Gesetz anwenden. 
Darauf achten das keine Brüche vorkommen

\subsubsection{Stromgleichungen}

Anzahl: k - 1\\
Kirchhoff current law umsetzen. 
Bekannte quellen nach links nehmen.

\subsubsection{Spannungsgleichungen}

Anzahl: z - k + 1\\
Kirchhoff voltage law anwenden. 
Bekannte Quellen auf die linke Seite nehmen.


\subsubsection{Zusammenführen}

Zweiggleichungen in die Strom- und Spannungsgleichungen einsetzen. 
Gleichungstablo aufstellen mit unbekannten als Vektor.

\subsubsection{Beispiel}

In diesem Beispiel wird die maximale Anzahl Knoten verwendet.
Ein Knoten könnte man sparen, allerdings bleibt der Rechenaufwand gleich.

\begin{center}
    % LTeX: enabled=false
\begin{circuitikz}
    \tikzstyle{every node}=[font=\small]
    \draw (3,11.75) to[american voltage source] (3,9.5);
    \draw (3,11.75) to[R] (5.25,11.75);
    \draw (5.25,11.75) to[R] (5.25,9.5);
    \draw (5.25,11.75) to[R] (7.5,11.75);
    \draw (7.5,9.5) to[american current source] (7.5,11.75);
    \draw (3,9.5) to[short] (7.5,9.5);
    \node at (5.25,11.75) [circ] {};
    \node at (7.5,11.75) [circ] {};
    \node at (5.25,9.5) [circ] {};
    \draw [ color={rgb,255:red,255; green,0; blue,0}, ->, >=Stealth] (3.25,11.75) -- (3.5,11.75);
    \node at (3,11.75) [circ] {};
    \node [font=\small, color={rgb,255:red,255; green,0; blue,0}] at (3.75,9.25) {};
    \node [font=\small, color={rgb,255:red,255; green,0; blue,0}] at (3.75,9.25) {};
    \draw [ color={rgb,255:red,255; green,0; blue,0}, ->, >=Stealth] (4.75,11.75) -- (5,11.75);
    \draw [ color={rgb,255:red,255; green,0; blue,0}, ->, >=Stealth] (5.25,10) -- (5.25,9.75);
    \draw [ color={rgb,255:red,255; green,0; blue,0}, ->, >=Stealth] (5.75,11.75) -- (5.5,11.75);
    \draw [ color={rgb,255:red,0; green,0; blue,255}, ->, >=Stealth] (5,11.25) .. controls (4.75,11) and (4.75,10.5) .. (5,10) ;
    \draw [ color={rgb,255:red,0; green,0; blue,255}, ->, >=Stealth] (3.5,11.5) .. controls (3.75,11.25) and (4.5,11.25) .. (4.75,11.5) ;
    \draw [ color={rgb,255:red,0; green,0; blue,255}, ->, >=Stealth] (7,11.5) .. controls (6.5,11.25) and (6.25,11.25) .. (5.75,11.5) ;
    \node [font=\small, color={rgb,255:red,0; green,0; blue,255}] at (2,10.5) {$U_q$};
    \node [font=\small, color={rgb,255:red,0; green,0; blue,255}] at (4,11) {$U_1$};
    \node [font=\small, color={rgb,255:red,0; green,0; blue,255}] at (4.5,10.75) {$U_2$};
    \node [font=\small, color={rgb,255:red,0; green,0; blue,255}] at (6.5,11) {$U_3$};
    \node [font=\small, color={rgb,255:red,255; green,0; blue,0}] at (3.5,12) {$I_0$};
    \node [font=\small, color={rgb,255:red,255; green,0; blue,0}] at (5,12) {$I_1$};
    \node [font=\small, color={rgb,255:red,255; green,0; blue,0}] at (5.5,12) {$I_3$};
    \node [font=\small, color={rgb,255:red,255; green,0; blue,0}] at (5.5,9.75) {$I_2$};
    \node [font=\small, color={rgb,255:red,255; green,0; blue,0}] at (8.25,10.75) {$I_q$};
    \node [font=\small] at (4.25,12.25) {R1};
    \node [font=\small] at (6.25,12.25) {R3};
    \node [font=\small] at (5.75,10.5) {R2};
    \draw [ color={rgb,255:red,0; green,0; blue,255}, ->, >=Stealth] (2.75,11.25) .. controls (2.25,11) and (2.25,10.25) .. (2.75,10) ;
    \draw [ color={rgb,255:red,0; green,0; blue,255}, ->, >=Stealth] (7.25,11.25) .. controls (7,11) and (7,10.5) .. (7.25,10) ;
    \node [font=\small, color={rgb,255:red,0; green,0; blue,255}] at (6.75,10.5) {$U_4$};
\end{circuitikz}
\end{center}
$U_q$ und $I_q$ gegeben\\
Knoten : 4, Zweige : 5\\

\textbf{Zweiggleichungen:}\\
z - v - i = 5 - 1 - 1 = 3

\[
U_1 = R_1 \cdot I_1\\
U_2 = R_2 \cdot I_2\\
U_3 = R_3 \cdot I_3
\]

\textbf{Stromgleichungen:}\\
k - 1 = 4 - 1 = 3 

\[
0 = I_1 + I3 - I_2\\
I_q = I_3\\
0 = I_1 - I_0
\]

\textbf{Spannungsgleichungen:}\\

z - k + 1 = 5 - 4 + 1 = 2

\[
U_q = U_1 + U_2\\
0 = U_2 + U_3 + U_4
\]

\textbf{Einsetze der Zweiggleichungen:}

\[
U_q = R_1 \cdot I_1 + R_2 \cdot I_2\\
0 = R_2 \cdot I_2 + R_3 \cdot I_3 + - U_4\\
0 = I_1 - I_2 + I_3\\
I_q = I_3\\
0 = I_1 - I_0
\]

\textbf{Tablo erstellen:}

\[
\begin{bmatrix}
    \cdot & R_1 & R_2 & \cdot  & \cdot\\
    \cdot & R_2 & R_3 & \cdot  & -1\\
    \cdot & 1 & -1 & 1 & \cdot \\
    \cdot & \cdot & 1 & \cdot  & \cdot\\
    -1 & 1 & \cdot & \cdot  & \cdot\\
\end{bmatrix}
\begin{pmatrix}
    I_0\\
    I_1\\
    I_2\\
    I_3\\
    U_4\\
\end{pmatrix}
=
\begin{pmatrix}
    U_q\\
    \cdot\\
    \cdot\\
    I_q\\
    \cdot\\
\end{pmatrix}
\]    

\subsection{Maschenstrom-/Kreisstrommethode}

\textbf{Aufwand} : M - 1 - i\\

Bei der Kreisstrommethode werden verschiedene Kreisströme als unbekannte definiert. 
Liegen diese auf Maschen spricht man von der Maschentrommethode. 
KVL wird dabei stark genutzt. 
Falls Ströme oder Spannungen gesucht werden, können diese durch einen einzelnen Kreisstrom am einfachsten bestimmt werden. \\
Das Verfahren kann nur bei Planaren Schaltungen verwendet werden. 
Das darf keine Überschneidungen der Zweige geben.    
\textbf{Achtung!} nicht planar wirkende Schaltungen können planar sein, wenn sie etwas umgezeichnet werden:

\begin{center}
    % LTeX: enabled=false
\begin{circuitikz}
    \tikzstyle{every node}=[font=\small]
    \node at (3,0) [circ, color={rgb,255:red,255; green,128; blue,0}] {};
    \draw (0,3.5) to[crossing] (0,4);
    \draw (0,3) to[short] (0,3.5);
    \draw (0,4) to[short] (0,4.25);
    \draw (0,0) to[short] (-0.5,0);
    \draw (-0.5,0) to[short] (-0.5,3.75);
    \draw (-0.5,3.75) to[short] (3,3.75);
    \draw (3,3.75) to[short] (3,3);
    \draw (3,0) to[short] (3.5,0);
    \draw (3.5,0) to[short] (3.5,4.25);
    \draw (3.5,4.25) to[short] (0,4.25);
    \draw (7,3) to[short] (7,0);
    \draw (10,3) to[short] (10,0);
    \draw [short] (7,3) -- (10,0);
    \draw [short] (10,3) -- (7,0);
    \node at (0,3) [circ, color={rgb,255:red,128; green,0; blue,128}] {};
    \node at (3,3) [circ, color={rgb,255:red,0; green,128; blue,255}] {};
    \node at (0,0) [circ, color={rgb,255:red,0; green,255; blue,0}] {};
    \draw [short] (0,0) -- (3,3);
    \draw [short] (0,3) -- (3,0);
    \node at (1.5,1.5) [circ, color={rgb,255:red,255; green,0; blue,0}] {};
    \node at (10,3) [circ, color={rgb,255:red,128; green,0; blue,128}] {};
    \node at (10,0) [circ, color={rgb,255:red,255; green,128; blue,0}] {};
    \node at (7,0) [circ, color={rgb,255:red,0; green,255; blue,0}] {};
    \node at (7,3) [circ, color={rgb,255:red,0; green,128; blue,255}] {};
    \node at (8.5,1.5) [circ, color={rgb,255:red,255; green,0; blue,0}] {};
    \draw [->, >=Stealth] (4.25,1.5) -- (6.5,1.5);
    \node [font=\small] at (5.25,1.75) {umzeichnen};
\end{circuitikz}
\end{center}

\subsubsection{Vorgehen}

\begin{enumerate}
    \item Knoten definieren
    \item Stammbaum definieren\\
    {\color{red}\textbf{Es darf keine Ideale Stromquelle im Pfad des Stammbaums sein}}
    \item Matrix einfüllen:
\end{enumerate}

\begin{tabular}{ccc}
    Widerstände & Kreisstöme & Quellspannungen\\
    $\begin{bmatrix} 
        \mathcolor{orange}{R + R} & R \mathcolor{violet}{+ v}\\ 
        R & \mathcolor{orange}{R + R}\\ 
    \end{bmatrix}$
    &
    $\begin{bmatrix}
        J_1\\
        J_2\\
    \end{bmatrix}$
    &
    $=
    \begin{bmatrix}
        U_{q_1}\\
        U_{q_2}\\
    \end{bmatrix}$
    \\
\end{tabular}
\\
Die Matrix ist grundsätzlich Symmetrisch an der diagonalen. 
In der Diagonalen sind die Summen aller Widerstände im Pfad der Kreisströme. 
Alle Überlagerungen welche durch verschiedene Kreisströme durch denselben Widerstand entstehen, werden jeweils oberhalb und unterhalb davon notiert. 
Diese sind \textbf{negativ}. 
Entgegengesetzte Richtung \say{-} sonst \say{+}. 
Gesteuerte Quellen brechen diese Symetrie.
Wenn Kreisströme nicht unabhängig definiert sind müssen sie trotzdem als Störfaktoren mit einbezogen werden. 
Diese können in den Quellenvektor genommen werden. 
Falls basierend auf vorherige Grössen können sie in die Matrix genommen werden.

\subsubsection{Eifaches Beispiel ohne gesteuerte Quellen}

\begin{center}
    % LTeX: enabled=false
\begin{circuitikz}
    \tikzstyle{every node}=[font=\normalsize]
    \draw (3.75,13.25) to[R] (10.75,13.25);
    \draw (4.75,10) to[R] (4.75,7.5);
    \draw (3.75,11.5) to[short] (3.75,13.25);
    \draw (2.75,7.5) to[american current source] (2.75,10);
    \draw (2.75,10) to[short] (4.75,10);
    \draw (4.75,7.5) to[short] (2.75,7.5);
    \draw (3.75,10) to[short] (3.75,11.5);
    \draw (3.75,11.75) to[R] (7.25,11.75);
    \draw (7.25,11.75) to[R] (10.75,11.75);
    \draw (3.75,6.75) to[short] (3.75,7.5);
    \draw (10.75,11.75) to[short] (10.75,13.25);
    \draw (10.75,11.75) to[R] (10.75,9.75);
    \draw (9,9.75) to[R] (9,8);
    \draw (10.75,9.75) to[american voltage source] (10.75,8.25);
    \draw (9,9.75) to[short] (9,10.25);
    \draw (9,10.25) to[short] (7.25,10.25);
    \draw (9,8) to[short] (9,7.5);
    \draw (9,7.5) to[short] (10.75,7.5);
    \draw (10.75,8.25) to[short] (10.75,7.5);
    \draw (7.25,6.75) to[R] (10.75,6.75);
    \draw (10.75,7.5) to[short] (10.75,6.75);
    \draw (10.75,11.75) to[short] (12.75,11.75);
    \draw (12.75,11.75) to[R] (12.75,5.5);
    \draw (12.75,5.5) to[short] (3.75,5.5);
    \draw (3.75,5.5) to[short] (3.75,6.75);
    \draw (3.75,6.75) to[R] (7.25,6.75);
    \draw (7.25,10.25) to[american voltage source] (7.25,6.75);
    \draw (7.25,10.25) to[short] (7.25,11.75);
    \node at (3.75,7.5) [circ] {};
    \node at (3.75,6.75) [circ] {};
    \node at (3.75,10) [circ] {};
    \node at (3.75,11.75) [circ] {};
    \node at (7.25,10.25) [circ] {};
    \node at (7.25,11.75) [circ] {};
    \node at (10.75,11.75) [circ] {};
    \node at (7.25,6.75) [circ] {};
    \node at (10.75,7.5) [circ] {};
    \node [font=\normalsize] at (2,8.75) {$I_c$};
    \node [font=\normalsize] at (5.25,8.75) {$R_1$};
    \node [font=\normalsize] at (5.5,12.25) {$R_2$};
    \node [font=\normalsize] at (7.25,13.75) {$R_4$};
    \node [font=\normalsize] at (9,12.25) {$R_7$};
    \node [font=\normalsize] at (11.25,10.75) {$R_{10}$};
    \node [font=\normalsize] at (11.5,9) {$U_{10}$};
    \node [font=\normalsize] at (13.25,8.5) {$R_5$};
    \node [font=\normalsize] at (9.5,9) {$R_8$};
    \node [font=\normalsize] at (6.5,8.5) {$U_6$};
    \node [font=\normalsize] at (9,7.25) {$R_9$};
    \node [font=\normalsize] at (5.5,7.25) {$R_3$};
    \draw [ color={rgb,255:red,0; green,128; blue,255}, ->, >=Stealth] (7,12.25) .. controls (6.25,13) and (8.25,13) .. (7.5,12.25) ;
    \node [font=\normalsize, color={rgb,255:red,0; green,128; blue,255}] at (7.25,12.5) {$J_1$};
    \draw [ color={rgb,255:red,0; green,128; blue,255}, ->, >=Stealth] (9.5,10.5) .. controls (8.75,11.25) and (10.75,11.25) .. (10,10.5) ;
    \node [font=\normalsize, color={rgb,255:red,0; green,128; blue,255}] at (9.75,10.75) {$J_4$};
    \draw [ color={rgb,255:red,0; green,128; blue,255}, ->, >=Stealth] (7.75,7.25) .. controls (7,8) and (9,8) .. (8.25,7.25) ;
    \node [font=\normalsize, color={rgb,255:red,0; green,128; blue,255}] at (8,7.5) {$J_3$};
    \draw [ color={rgb,255:red,0; green,128; blue,255}, ->, >=Stealth] (11.25,6.25) .. controls (10.5,7) and (12.5,7) .. (11.75,6.25) ;
    \node [font=\normalsize, color={rgb,255:red,0; green,128; blue,255}] at (11.5,6.5) {$J_5$};
    \draw [ color={rgb,255:red,0; green,128; blue,255}, ->, >=Stealth] (5.5,10.25) .. controls (4.75,11) and (6.75,11) .. (6,10.25) ;
    \node [font=\normalsize, color={rgb,255:red,0; green,128; blue,255}] at (5.75,10.5) {$J_2$};
\end{circuitikz}
\end{center}

\[
\begin{bmatrix}
    R_2 + R_4 + R_7 & - R_2 & \cdot & -R_7 & \cdot \\
    -R_2 & R_1 + R_2 + R_3 & \cdot & \cdot & - R_3 \\
    \cdot & \cdot & R_8 + R_9 & -R_8 & -R_9 \\
    -R_7 & \cdot & -R_8 & R_7 + R_8 + R_{10} & -R_{10} \\
    \cdot & -R_3 & -R_9 & -R_{10} & R_3 + R_5 + R_9 + R_{10} \\
\end{bmatrix}
\begin{bmatrix}
    J_1\\
    J_2\\
    J_3\\
    J_4\\
    J_5\\
\end{bmatrix}
=
\]
\[
\begin{bmatrix}
    \cdot\\
    R_1\cdot I_c - U_6\\
    U_6\\
    -U_{10}\\
    U_{10}\\
\end{bmatrix}
\]

\subsubsection{2. Beispiel mit gesteuerten Quellen}

\begin{center}
    % LTeX: enabled=false
\begin{circuitikz}
    \tikzstyle{every node}=[font=\normalsize]
    \draw (2.25,14) to[american current source] (2.25,11);
    \draw (5.5,11) to[R] (5.5,14);
    \draw (8.75,11) to[R] (8.75,14);
    \draw (2.25,14) to[short] (8.75,14);
    \draw (2.25,11) to[R] (5.5,11);
    \draw (5.5,8) to[R] (5.5,11);
    \draw (2.25,11) to[american controlled voltage source] (2.25,8);
    \draw (8.75,8) to[american controlled current source] (8.75,11);
    \draw (5.5,11) to[american voltage source] (8.75,11);
    \draw (2.25,8) to[short] (5.5,8);
    \draw (5.5,8) to[short] (8.75,8);
    \node [font=\normalsize] at (1.5,12.5) {$I_q$};
    \node [font=\normalsize] at (4,11.5) {$R_1$};
    \node [font=\normalsize] at (6,12.5) {$R_2$};
    \node [font=\normalsize] at (9.25,12.5) {$R_3$};
    \node [font=\normalsize] at (9.5,9) {$g\cdot U_s$};
    \node [font=\normalsize] at (6,9.5) {$R_4$};
    \node [font=\normalsize] at (1.25,9.5) {$r\cdot I_s$};
    \node [font=\normalsize] at (7.25,11.5) {$U_q$};
    \draw [ color={rgb,255:red,255; green,0; blue,0}, ->, >=Stealth] (7,14) -- (7.25,14);
    \draw [ color={rgb,255:red,0; green,128; blue,255}, ->, >=Stealth] (5.25,13.5) .. controls (4.75,13.25) and (4.75,12) .. (5.25,11.5) ;
    \node [font=\normalsize, color={rgb,255:red,0; green,128; blue,255}] at (4.5,12.5) {$U_s$};
    \node [font=\normalsize, color={rgb,255:red,255; green,0; blue,0}] at (7.25,14.5) {$I_s$};
    \draw [ color={rgb,255:red,217; green,0; blue,108}, ->, >=Stealth] (7,9.25) .. controls (6.25,9.75) and (8,9.75) .. (7.25,9.25) node[pos=0.5, fill=white]{$J_1$};
    \draw [ color={rgb,255:red,217; green,0; blue,108}, ->, >=Stealth] (3.5,12.5) .. controls (2.75,13) and (4.5,13) .. (3.75,12.5) node[pos=0.5, fill=white]{$J_1$};
    \draw [ color={rgb,255:red,217; green,0; blue,108}, ->, >=Stealth] (7,12.5) .. controls (6.25,13) and (8,13) .. (7.25,12.5) node[pos=0.5, fill=white]{$J_1$};
    \draw [ color={rgb,255:red,217; green,0; blue,108}, ->, >=Stealth] (3.5,9.25) .. controls (2.75,9.75) and (4.5,9.75) .. (3.75,9.25) node[pos=0.5, fill=white]{$J_1$};
    \node at (2.25,11) [circ] {};
    \node at (5.5,11) [circ] {};
    \node at (5.5,8) [circ] {};
    \node at (8.75,11) [circ] {};
    \node at (5.5,14) [circ] {};
\end{circuitikz}
\end{center}

Vorbereitungsgleichungen:\\
\begin{align*}
    J_1 &= -I_q\\
    J_2 &= I_s\\
    J_4 &= -g\cdot U_s = -g\cdot R_2 \cdot (J_1 - J_2) = g\cdot R_2\cdot (I_q + J_2)\\ 
\end{align*}


gesucht sind $_J2$ \& $J_3$

\[
\begin{bmatrix}
    R_2 + R_3 & \cdot\\
    \cdot & R_1 + R_4\\
\end{bmatrix}
\begin{pmatrix}
    J_2\\
    J_3\\
\end{pmatrix}
=
\begin{pmatrix}
    U_q + \overbrace{J_1 \cdot R_2}^{-I_q \cdot R_2}  \\
    r\cdot \underbrace{I_s}_{J_2} + \underbrace{J_1 \cdot R_1}_{-I_q\cdot R_1} + \underbrace{J_4 \cdot R_4}_{g \cdot R_2 \cdot R_4 \cdot I_q + g \cdot R_2 \cdot R_4 \cdot J_2} \\
\end{pmatrix}
\]
Vereinfachen:

\[
\begin{bmatrix}
    R_2 + R_3 & \cdot\\
    -g \cdot R_2 \cdot R_4 - r & R_1 + R_4\\
\end{bmatrix}
\begin{pmatrix}
    J_2\\
    J_3\\
\end{pmatrix}
=
\begin{pmatrix}
    U_q - I_q \cdot R_2 \\
    -I_q \cdot R_1 + g \cdot R_2 \cdot R_4 \cdot I_q \\
\end{pmatrix}
\]
Überprüfen ob in der Matrix nur Widerstände stehen und im Quellvektor nur Spannunen. 

\subsection{Knotenpotentialmethode}

\textbf{Aufwand:} k - 1 - v\\

Bei der Knotenpotentialmethode wird implizit KCL verwendet. 
Es werden Knoten definiert, welche in den Unekanntenvektor eingefüllt werden. 
Ein Knoten wird als \say{0} definiert. 
Dieser wird nicht in den Vektor eingefügt.
Die Matrix enthält nur Leitwerte.
Die Diagonale sind die Leitwerte, die an den knoten angeschlossen sind. 
Die Positionen neben der diagonalen sind \textbf{immer Negativ}. 
Sie sind die Verbindungen zwischen den jeweiligen knoten. 
Wenn keine gesteuerte Quellen vorhanden sind, treten Widerstände entweder 1 oder 4-mal in der Matrix auf. 
Bis auf die gesteuerten Quellen ist die Matrix symmetrisch. 
Spannungsquellen sind problematisch da sie weder in die Matrix noch in den Quellvektor eingefügt werden können. 
Sie müssen verschoben werden wie im Kapitel \ref{Quelln verschieben} gezeigt.

\subsubsection{Beispiel}

\begin{center}
    \input{circuits/Beispiel Knotenpotentialmethode Schritt 1.tex}
\end{center}

Vorbereitung:

\begin{align*}
    U_3 &= -U_q\\
    U_4 &= U_1 - r\cdot I_s\\
    I_s &= \frac{1}{R_3} \cdot (U_2 - U_3)\\
    U_s &= U_2\\
\end{align*}

Verschieben der Spannungsquellen:

\begin{center}
    % LTeX: enabled=false
\begin{circuitikz}
    \tikzstyle{every node}=[font=\normalsize]
    \draw (2.25,14) to[american current source] (2.25,11);
    \draw (5.5,11) to[R] (5.5,14);
    \draw (8.75,12.5) to[R] (8.75,14);
    \draw (2.25,14) to[short] (8.75,14);
    \draw (2.25,11) to[R] (5.5,11);
    \draw (5.5,9.5) to[R] (5.5,11);
    \draw (8.75,8) to[american controlled current source] (8.75,9.5);
    \draw (2.25,8) to[short] (5.5,8);
    \node [font=\normalsize] at (1.5,12.5) {$I_q$};
    \node [font=\normalsize] at (4,11.5) {$R_1$};
    \node [font=\normalsize] at (6,12.5) {$R_2$};
    \node [font=\normalsize] at (9.25,13.25) {$R_3$};
    \node [font=\normalsize] at (9.75,8.75) {$g \cdot U_s$};
    \node [font=\normalsize] at (6,10.25) {$R_4$};
    \node [font=\normalsize] at (4.5,8.75) {$r \cdot I_s$};
    \node [font=\normalsize] at (7.25,11.5) {$U_q$};
    \draw [ color={rgb,255:red,255; green,0; blue,0}, ->, >=Stealth] (7,14) -- (7.25,14);
    \draw [ color={rgb,255:red,0; green,128; blue,255}, ->, >=Stealth] (5.25,13.5) .. controls (4.75,13.25) and (4.75,12) .. (5.25,11.5) ;
    \node [font=\normalsize, color={rgb,255:red,0; green,128; blue,255}] at (4.5,12.5) {$U_s$};
    \node [font=\normalsize, color={rgb,255:red,255; green,0; blue,0}] at (7.25,14.5) {$I_s$};
    \node at (2.25,11) [circ] {};
    \node at (5.5,11) [circ] {};
    \node at (5.5,8) [circ] {};
    \node at (8.75,11) [circ] {};
    \node at (5.5,14) [circ] {};
    \node [font=\normalsize, color={rgb,255:red,0; green,128; blue,255}] at (1.75,11) {$U_1$};
    \node [font=\normalsize, color={rgb,255:red,0; green,128; blue,255}] at (5.5,14.5) {$U_2$};
    \node [font=\normalsize, color={rgb,255:red,0; green,128; blue,255}] at (9.25,11) {$U_3$};
    \node [font=\normalsize, color={rgb,255:red,0; green,128; blue,255}] at (5.5,7.5) {$U_4$};
    \draw (4.75,10.25) to (4.75,10) node[ground]{};
    \draw [short] (4.75,10.25) -- (5.5,11);
    \node [font=\normalsize] at (4.25,10.25) {$"0"$};
    \draw (8.75,11) to[american voltage source] (8.75,9.5);
    \draw (5.5,11) to[short] (8.75,11);
    \draw (8.75,11) to[american voltage source] (8.75,12.5);
    \draw (5.5,9.5) to[american controlled voltage source] (5.5,8);
    \draw (8.75,8) to[american controlled voltage source] (5.5,8);
    \node [font=\normalsize] at (7.25,8.75) {$r \cdot I_s$};
    \draw [ color={rgb,255:red,255; green,128; blue,0}, ](6.5,8) to[short] (7.75,8);
    \draw [ color={rgb,255:red,255; green,128; blue,0}, ](8.75,9.75) to[short] (8.75,10.75);
    \draw (2.25,11) to[short] (2.25,8);
\end{circuitikz}
\end{center}

\[
\begin{bmatrix}
    \frac{1}{R_1} + \frac{1}{R_4} & \cdot \\
    \cdot & \frac{1}{R_2} + \frac{1}{R_3} \\
\end{bmatrix}
\begin{pmatrix}
    U_1\\
    U_2\\
\end{pmatrix}
=
\begin{pmatrix}
    I_q - \overbrace{-\frac{r\cdot I_s}{R_4}}^{\frac{r \cdot \frac{1}{R_3}(U_2 - U_3)}{R_3\cdot R_4}} - \overbrace{g\cdot U_s}^{g \cdot U_2}\\
    -I_q - \frac{U_q}{R_3}\\
\end{pmatrix}
\]
vereinfachen:
\[
\begin{bmatrix}
    \frac{1}{R_1} + \frac{1}{R_4} & -\frac{r}{R_3\cdot R_4}-g \\
    \cdot & \frac{1}{R_2} + \frac{1}{R_3} \\
\end{bmatrix}
\begin{pmatrix}
    U_1\\
    U_2\\
\end{pmatrix}
=
\begin{pmatrix}
    I_q\\
    -I_q - \frac{U_q}{R_3}\\
\end{pmatrix}
\]

Kontrolle: Matrix enthält nur Leitwert und Quellvektor nur Ströme.

\subsection{Modifizierte Knotenpotentialmethode}

\textbf{Aufwand:} K - 1 + v + s (s sind gesuchte grössen)\\

Die modifizierte Knotenpotentialmethode ist die allgemeinere Form der kpm. 
Mit dieser kann jedes Netzwerk ohne vorherige Veränderung analysiert werden. 
Die Matrix der kpm erhält lediglich mehr Variablen. 
Zu der aufgestellten Matrix der kpm werden für die Ströme in idealen Spannungsquellen, gesuchte Ströme und Ströme in Spulen eine zusätzliche Variable ermittelt.  
Für diese muss dann auch eine zusätzliche Gleichung aufgestellt werden. 
Für Ideale Spannungsquellen wird der Strom generell hinei ndefiniert.
Im Allgemeinen kann die Matrix auch mit Elementstempel aufgestellt werden.
Diese sind für alle Elemente definiert, die in einer Schaltung vorkommen. 

\subsubsection{elementstempel}

Jedes Bauteil in einer zu analysierenden Schaltung fügt bestimmte Teile in eine Matrix / Quellvektor hinzu. 
Da diese immer gleich sind, können diese nach folgendem Muster 1:1 hinzugefügt werden. 
Für Abhängige Grössen wird jeweils eine weitere Gleichung gebraucht welche mit a (auxiliary) benannt wird.  

\begin{multicols*}{2}

    \subsubsection{Widerstand}

    \begin{center}

        % LTeX: enabled=false
\begin{circuitikz}
    \tikzstyle{every node}=[font=\LARGE]
    \draw (0,0) to[R, l={ \Large R}] (2,0);
    \draw (0,0) to[short, -o] (-0.25,0) ;
    \draw (2,0) to[short, -o] (2.25,0) ;
    \node [font=\LARGE] at (-0.25,0.5) {m};
    \node [font=\LARGE] at (2.25,0.5) {n};
\end{circuitikz}

        \begin{tabular}{c|c|c}
            & m & n \\\hline
            m & $+ \frac{1}{R}$ & $- \frac{1}{R}$\\\hline
            n & $- \frac{1}{R}$ & $+ \frac{1}{R}$\\
        \end{tabular}
        \begin{tabular}{|c|}
            q vec.\\\hline
            $\cdot$\\\hline
            $\cdot$\\
        \end{tabular}
    \end{center}

   \subsubsection{Unabhängige Stromquelle}

   \begin{center}

        \input{circuits/Elementstempel unabhängige Stromquelle.tex}

        \begin{tabular}{c|c|c}
           & m & n \\\hline
            m & $\cdot$ & $\cdot$\\\hline
            n & $\cdot$ & $\cdot$\\
        \end{tabular}
        \begin{tabular}{|c|}
            q vec.\\\hline
            $-I_q$\\\hline
            $+I_q$\\
        \end{tabular}   
    \end{center}

    \subsubsection{Widerstand mit gesuchtem Strom}

    \begin{center}

        % LTeX: enabled=false
\begin{circuitikz}
    \tikzstyle{every node}=[font=\LARGE]
    \draw (0,0) to[R, l={ \Large R}] (2,0);
    \draw (0,0) to[short, -o] (-0.25,0) ;
    \draw (2,0) to[short, -o] (2.25,0) ;
    \node [font=\LARGE] at (-0.25,0.5) {m};
    \node [font=\LARGE] at (2.25,0.5) {n};
    \draw [ color={rgb,255:red,255; green,0; blue,0}, ->, >=Stealth] (0,0) -- (0.25,0);
    \node [font=\large, color={rgb,255:red,255; green,0; blue,0}] at (0.25,-0.5) {$I_s$};
\end{circuitikz}

        \[a: U_m - U_n = -R  I_s = 0\]

        \begin{tabular}{c|c|c|c}
            & m & n & $I_s$\\\hline
            m & $\cdot$ & $\cdot$ & 1 \\\hline
            n & $\cdot$ & $\cdot$ & -1\\\hline
            a & 1 & -1 & -R
        \end{tabular}
        \begin{tabular}{|c|}
            q vec.\\\hline
            $\cdot$\\\hline
            $\cdot$\\\hline
            0
        \end{tabular}
    \end{center}

    \subsubsection{Unabhängigeige Spannungsquelle}

    \begin{center}

        \input{circuits/Elementstempel unabhängige Spannungsquelle.tex}

        \[a: U_m - U_n = U_q \]

        \begin{tabular}{c|c|c|c}
            & m & n & $I_s$\\\hline
            m & $\cdot$ & $\cdot$ & 1 \\\hline
            n & $\cdot$ & $\cdot$ & -1\\\hline
            a & 1 & -1 & $\cdot$
        \end{tabular}
        \begin{tabular}{|c|}
            q vec.\\\hline
            $\cdot$\\\hline
            $\cdot$\\\hline
            $U_q$
        \end{tabular}
    \end{center}
    \newcolumn

    \subsubsection{Spannungsgesteuerte Spannungsquelle}
    \begin{center}

        % LTeX: enabled=false
\begin{circuitikz}
    \tikzstyle{every node}=[font=\large]
    \node [font=\LARGE] at (-0.25,0.5) {k};
    \node [font=\LARGE] at (2.25,0.5) {l};
    \draw (3.75,0) to[short, -o] (3.5,0) ;
    \draw (5.75,0) to[short, -o] (6,0) ;
    \node [font=\LARGE] at (3.5,0.5) {m};
    \node [font=\LARGE] at (6,0.5) {n};
    \node at (-0.25,0) [circ] {};
    \node at (2.25,0) [circ] {};
    \draw (3.75,0) to[american controlled voltage source, l = $v_u U_S$] (5.75,0);
    \node [font=\large, color={rgb,255:red,255; green,0; blue,0}] at (4,-0.5) {$I_{qu}$};
    \draw [ color={rgb,255:red,255; green,0; blue,0}, ->, >=Stealth] (3.75,0) -- (4,0);
    \draw [ color={rgb,255:red,0; green,128; blue,255}, ->, >=Stealth] (0,0.25) .. controls (0.25,0.75) and (1.75,0.75) .. (2,0.25) ;
    \node [font=\large, color={rgb,255:red,0; green,128; blue,255}] at (1,1) {$U_s$};
\end{circuitikz}

        \begin{align*}
            m - n &= v_u U_s\\
            U_s &= k-l\\
            0 &= m - n - k v_u \\            
        \end{align*}

        \begin{tabular}{c|c|c|c|c|c}
              & k & l & m & n & $I_q$ \\\hline
            k & $\cdot$ & $\cdot$ & $\cdot$ & $\cdot$ & $\cdot$ \\\hline
            l & $\cdot$ & $\cdot$ & $\cdot$ & $\cdot$ & $\cdot$ \\\hline
            m & $\cdot$ & $\cdot$ & $\cdot$ & $\cdot$ & 1 \\\hline
            n & $\cdot$ & $\cdot$ & $\cdot$ & $\cdot$ & -1 \\\hline
            a & $-v_u$ & $v_u$ & 1 & -1 & $\cdot$ \\
        \end{tabular}
        \begin{tabular}{|c|}
            q vec. \\\hline
            $\cdot$ \\\hline
            $\cdot$ \\\hline
            $\cdot$ \\\hline
            $\cdot$ \\\hline
            $\cdot$ \\
        \end{tabular}

    \end{center}

    \subsubsection{Spannungsgesteuerte Stromquelle}
    \begin{center}

        % LTeX: enabled=false
\begin{circuitikz}
    \tikzstyle{every node}=[font=\large]
    \node [font=\LARGE] at (-0.25,0.5) {k};
    \node [font=\LARGE] at (2.25,0.5) {l};
    \draw (3.75,0) to[short, -o] (3.5,0) ;
    \draw (5.75,0) to[short, -o] (6,0) ;
    \node [font=\LARGE] at (3.5,0.5) {m};
    \node [font=\LARGE] at (6,0.5) {n};
    \node at (-0.25,0) [circ] {};
    \node at (2.25,0) [circ] {};
    \draw (3.75,0) to[american controlled current source, l = $g U_S$] (5.75,0);
    \draw [ color={rgb,255:red,0; green,128; blue,255}, ->, >=Stealth] (0,0.25) .. controls (0.25,0.75) and (1.75,0.75) .. (2,0.25) ;
    \node [font=\large, color={rgb,255:red,0; green,128; blue,255}] at (1,1) {$U_s$};
\end{circuitikz}        

        \begin{align*}
            U_s = k - l\\
            I_{mn} = g(k-l)\\
            I_{mn} = gk - gl
        \end{align*}

        \begin{tabular}{c|c|c|c|c}
              & k & l & m & n \\\hline
            k & $\cdot$ & $\cdot$ & $\cdot$ & $\cdot$  \\\hline
            l & $\cdot$ & $\cdot$ & $\cdot$ & $\cdot$ \\\hline
            m & g & -g & $\cdot$ & $\cdot$ \\\hline
            n & -g & g & $\cdot$ & $\cdot$ \\
        \end{tabular}
        \begin{tabular}{|c|}
            q vec. \\\hline
            $\cdot$ \\\hline
            $\cdot$ \\\hline
            $\cdot$ \\\hline
            $\cdot$ \\
        \end{tabular}

    \end{center}
\end{multicols*}

\begin{multicols*}{2}
    
    \subsubsection{Stromgesteuerte Spannungsquelle}
    \begin{center}

        % LTeX: enabled=false
\begin{circuitikz}
    \tikzstyle{every node}=[font=\large]
    \node [font=\LARGE] at (-0.25,0.5) {k};
    \node [font=\LARGE] at (2.25,0.5) {l};
    \draw (3.75,0) to[short, -o] (3.5,0) ;
    \draw (5.75,0) to[short, -o] (6,0) ;
    \node [font=\LARGE] at (3.5,0.5) {m};
    \node [font=\LARGE] at (6,0.5) {n};
    \node at (-0.25,0) [circ] {};
    \node at (2.25,0) [circ] {};
    \draw (3.75,0) to[american controlled current source,l={ \large $v_i I_s$}] (5.75,0);
    \draw (-0.25,0) to[short] (2.25,0);
    \node [font=\Large, color={rgb,255:red,255; green,0; blue,0}] at (4.5,-0.5) {};
    \draw [ color={rgb,255:red,255; green,0; blue,0}, ->, >=Stealth] (1,0) -- (1.25,0);
    \node [font=\large, color={rgb,255:red,255; green,0; blue,0}] at (1,0.5) {$I_s$};
\end{circuitikz}

        \[U_k - U_l = 0\]

        \begin{tabular}{c|c|c|c|c|c}
              & k & l & m & n & $I_s$  \\\hline
            k & $\cdot$ & $\cdot$ & $\cdot$ & $\cdot$ & 1 \\\hline
            l & $\cdot$ & $\cdot$ & $\cdot$ & $\cdot$ & -1 \\\hline
            m & $\cdot$ & $\cdot$ & $\cdot$ & $\cdot$ & $v_i$ \\\hline
            n & $\cdot$ & $\cdot$ & $\cdot$ & $\cdot$ & $-v_i$  \\\hline
            $a_1$ & 1 & -1 & $\cdot$ & $\cdot$ &$\cdot$ \\
        \end{tabular}
        \begin{tabular}{|c|}
            q vec. \\\hline
            $\cdot$ \\\hline
            $\cdot$ \\\hline
            $\cdot$ \\\hline
            $\cdot$ \\\hline
            $\cdot$ \\
        \end{tabular}
    \end{center}

    \newcolumn

    \subsubsection{Stromgesteuerte Stromquelle}
    \begin{center}

        \input{circuits/Elementstempel Stromgesteuerte Stromquelle.tex}
        \begin{align*}
            a_1&: 0 = k - l\\
            a_2&: r I_s = m - n\\
        \end{align*}
        \begin{tabular}{c|c|c|c|c|c|c}
              & k & l & m & n & $I_s$ & $I_{qu}$ \\\hline
            k & $\cdot$ & $\cdot$ & $\cdot$ & $\cdot$ & 1 & $\cdot$  \\\hline
            l & $\cdot$ & $\cdot$ & $\cdot$ & $\cdot$ & -1 & $\cdot$ \\\hline
            m & $\cdot$ & $\cdot$ & $\cdot$ & $\cdot$ & $\cdot$ & 1 \\\hline
            n & $\cdot$ & $\cdot$ & $\cdot$ & $\cdot$ & $\cdot$ & -1 \\\hline
            $a_1$ & 1 & -1 & $\cdot$ & $\cdot$ & $\cdot$ & $\cdot$ \\\hline
            $a_2$ & $\cdot$ & $\cdot$ & 1 & -1 & -r & $\cdot$ \\
        \end{tabular}
        \begin{tabular}{|c|}
            q vec. \\\hline
            $\cdot$ \\\hline
            $\cdot$ \\\hline
            $\cdot$ \\\hline
            $\cdot$ \\\hline
            $\cdot$ \\\hline
            $\cdot$ \\
        \end{tabular}
    \end{center}

\end{multicols*}

\subsubsection{Beispiel}

\begin{center}
    % LTeX: enabled=false
\begin{circuitikz}
    \tikzstyle{every node}=[font=\normalsize]
    \draw (1.75,10.5) to[american controlled voltage source] (1.75,13.25);
    \draw (1.75,13.25) to[R] (4.5,13.25);
    \draw (4.5,13.25) to[R] (4.5,10.5);
    \draw (4.5,13.25) to[american current source] (7.25,13.25);
    \draw (4.5,13.25) to[short] (4.5,14.5);
    \draw (7.25,13.25) to[short] (7.25,14.5);
    \draw (4.5,14.5) to[R] (7.25,14.5);
    \draw (7.25,13.25) to[R] (7.25,10.5);
    \draw (10,10.5) to[american controlled current source] (10,13.25);
    \draw (1.75,10.5) to[short] (10,10.5);
    \draw (10,13.25) to[short] (7.25,13.25);
    \node at (4.5,13.25) [circ] {};
    \node at (7.25,13.25) [circ] {};
    \node at (4.5,10.5) [circ] {};
    \node at (7.25,10.5) [circ] {};
    \node [font=\normalsize, color={rgb,255:red,128; green,0; blue,128}] at (1.75,13.5) {$1$};
    \node [font=\normalsize, color={rgb,255:red,128; green,0; blue,128}] at (4.25,13.5) {$2$};
    \node [font=\normalsize, color={rgb,255:red,128; green,0; blue,128}] at (7,13.5) {$3$};
    \node [font=\normalsize, color={rgb,255:red,128; green,0; blue,128}] at (0.75,12) {$v_u \cdot U_S$};
    \node [font=\normalsize] at (3,13.75) {$R_1$};
    \node [font=\normalsize] at (4,12) {$R_2$};
    \node [font=\normalsize] at (5.75,12.5) {$I_q$};
    \node [font=\normalsize] at (6,15) {$G_3$};
    \node [font=\normalsize] at (11,11.75) {$V_i \cdot I_s$};
    \node [font=\normalsize] at (6.75,11.75) {$G_4$};
    \draw (2.25,10.5) to (2.25,10.25) node[ground]{};
    \node at (2.25,10.5) [circ] {};
    \draw [ color={rgb,255:red,255; green,0; blue,0}, ->, >=Stealth] (1.75,12.75) -- (1.75,13);
    \draw [ color={rgb,255:red,255; green,0; blue,0}, ->, >=Stealth] (4.5,11) -- (4.5,10.75);
    \node [font=\normalsize, color={rgb,255:red,255; green,0; blue,0}] at (1.5,13) {$I_q$};
    \node [font=\normalsize, color={rgb,255:red,255; green,0; blue,0}] at (4.75,10.75) {$I_s$};
    \draw [ color={rgb,255:red,0; green,128; blue,255}, ->, >=Stealth] (7.5,12.75) .. controls (8,12.5) and (8,11.25) .. (7.5,11) ;
    \node [font=\normalsize, color={rgb,255:red,0; green,128; blue,255}] at (8.25,12) {$U_s$};
\end{circuitikz}
\end{center}

\[
\begin{bmatrix}
    \frac{1}{R_1} & -\frac{1}{R_1} & \cdot & -1 & \cdot \\
    -\frac{1}{R_1} & \cellcolor{red!30}\frac{1}{R_1} + G_3 & - G_3 & \cdot & 1 \\
    \cdot & -G_3 & G_3 + G_4 & \cdot & v_i \\
    -1 & \cdot & -v_u & \cdot & \cdot \\ 
    \cdot & 1 & \cdot & \cdot &\cellcolor{red!30}-R_2\\
\end{bmatrix} 
\begin{pmatrix}
    U_1 \\
    U_2 \\
    U_3 \\
    I_q \\
    I_s \\
\end{pmatrix}
=
\begin{pmatrix}
    \cdot \\
    -I_q \\
    I_q \\
    \cdot \\
    \cdot \\
\end{pmatrix}
\]

An der markierten Stelle würde eigentlich $\frac{1}{R_2}$ Stehen laut NA.
Da aber ein Strom des Widerstandes gesucht wird, ist dieser nach unten rechts gerutscht da er ansonsten doppelt vorkommt. 
Siehe Elementstempel.

        
	\end{multicols*}
 
    \includepdf{formelblatt_ELT1_2023_v1_1_page.pdf}%anhang
\end{document}