\section{Nichtlineare Netzwerke / Leistungsanpassung}



\begin{wrapfigure}{l}{0.1\linewidth}
    \begin{circuitikz}
        \draw (0,0) to[R] (0,1.5);
        \draw [short] (0.5,1.5) -- (0.5,1);
        \draw [short] (0.5,1) -- (-0.5,0.5);
        \draw [short] (-0.5,0.5) -- (-0.5,0);   
    \end{circuitikz}     
\end{wrapfigure}

Viele Netzwerke sind in der Praxis nicht linear.\\
Meistens wird dabei ein nicht linearen Widerstand verwendet, um diesen Effekt zu modellieren.\\
Das links stehende Schemasymbol wird dabei verwendet.\\
Die Änderung des Widerstandswertes durch einen Effekt wie zum Beispiel Temperatur, licht oder mechanische Einwirkung herbeigeführt wird können Symbole diesen Effekt signalisieren. 
Symbole wie z.B. Pfeile oder ein Temperatursymbol.\\ 


\subsection{Leistungsanpassung}

Leistungsanpassung beschreibt das Verfahren ein Widerstand an einer realen Quelle anzupassen, sodass die maximal mögliche Energie aus der Quelle fliesst. 
Im Lineraren Fall entspricht der Lastwiderstand dem Innenwiderstand der Quelle. 

\begin{multicols*}{2}
    
    % LTeX: enabled=false
\begin{circuitikz}
    \tikzstyle{every node}=[font=\normalsize]
    \draw (0,2) to[american voltage source,l={ \normalsize $U_0$}] (0,0);
    \draw (0,2) to[R,l={ \normalsize $R_i$}] (2,2);
    \draw (2,2) to[short, -o] (2.5,2) ;
    \draw (2,0) to[short, -o] (2.5,0) ;
    \draw (2,0) to[short] (0,0);
    \draw [ color={rgb,255:red,115; green,0; blue,230} ](3.25,2) to[R,l={ \normalsize Rl}] (3.25,0);
    \draw[ color={rgb,255:red,115; green,0; blue,230}, ] (3.25,2) to[short] (2.5,2);
    \draw [ color={rgb,255:red,115; green,0; blue,230}, ](2.5,0) to[short] (3.25,0);
    \draw [ color={rgb,255:red,115; green,0; blue,230}, ->, >=Stealth] (2.75,0.5) -- (3.75,1.5);
    \draw [ color={rgb,255:red,0; green,0; blue,255}, short] (2.5,1.75) -- (4.5,1.5);
    \draw [ color={rgb,255:red,0; green,0; blue,255}, short] (2.5,1.75) -- (2.5,0.25);
    \draw [ color={rgb,255:red,0; green,0; blue,255}, short] (2.5,0.25) -- (4.5,0.5);
    \draw [ color={rgb,255:red,0; green,0; blue,255}, short] (4.5,0.5) -- (4.5,0.25);
    \draw [ color={rgb,255:red,0; green,0; blue,255}, short] (4.5,1.5) -- (4.5,1.75);
    \draw [ color={rgb,255:red,0; green,0; blue,255}, short] (4.5,1.75) -- (5.25,1);
    \draw [ color={rgb,255:red,0; green,0; blue,255}, short] (5.25,1) -- (4.5,0.25);
    \node [font=\normalsize, color={rgb,255:red,0; green,0; blue,255}] at (4.75,1) {Pl};
\end{circuitikz}  

    \begin{center}
        \begin{tcolorbox}[colframe=violet , colback=white, arc is curved, hbox]
            $R_l^\ast = R_i$
        \end{tcolorbox}
        Maximale leistung, $\eta=50\%$\\
        Der Stern am $R_l$ signalisiert, dass Leistungsanpassung gemacht wurde. 
    \end{center}


    \newcolumn
    % LTeX: enabled=false
\begin{tikzpicture}
    \draw [->, >=Stealth] (0,0) -- (0,5.5);
    \draw [->, >=Stealth] (0,0) -- (5.5,0);
    \node [font=\normalsize] at (-0.5,3.75) {$\frac{3U_0}{4}$};
    \node [font=\normalsize] at (-0.5,2.5) {$\frac{U_0}{2}$};
    \node [font=\normalsize] at (-0.5,1.25) {$\frac{U_0}{4}$};
    \draw [ color={rgb,255:red,0; green,128; blue,0}, short] (0,0) .. controls (0.75,5) and (4.25,5) .. (5,0);
    \draw [short] (0,2.5) -- (-0.25,2.5);
    \draw [short] (0,1.25) -- (-0.25,1.25);
    \draw [short] (0,3.75) -- (-0.25,3.75);
    \draw [short] (0,5) -- (-0.25,5);
    \draw [short] (1.25,0) -- (1.25,-0.25);
    \draw [short] (2.5,0) -- (2.5,-0.25);
    \draw [short] (3.75,0) -- (3.75,-0.25);
    \draw [short] (5,0) -- (5,-0.25);
    \draw [dashed] (0,5) -- (5,0);
    \node [font=\normalsize] at (2.5,-0.5) {$\frac{I_0}{2}$};
    \node [font=\normalsize] at (1.25,-0.5) {$\frac{I_0}{4}$};
    \node [font=\normalsize] at (3.75,-0.5) {$\frac{3I_0}{4}$};
    \node [font=\normalsize] at (5,-0.5) {$I_0$};
    \node [font=\normalsize, color={rgb,255:red,255; green,0; blue,0}] at (5.75,0) {$I$};
    \node [font=\normalsize, color={rgb,255:red,0; green,0; blue,255}] at (-0.25,5.75) {$U$};
    \node [font=\normalsize] at (-0.5,5) {$U_0$};
    \draw [ color={rgb,255:red,255; green,128; blue,0}, dashed] (0,3.75) -- (1.25,3.75);
    \draw [ color={rgb,255:red,255; green,128; blue,0}, dashed] (1.25,3.75) -- (1.25,0);
    \draw [ color={rgb,255:red,255; green,0; blue,0}, dashed] (0,2.5) -- (2.5,2.5);
    \draw [ color={rgb,255:red,255; green,0; blue,0}, dashed] (2.5,2.5) -- (2.5,0);
    \draw [ color={rgb,255:red,162; green,162; blue,162}, dashed] (0,1.25) -- (3.75,1.25);
    \draw [ color={rgb,255:red,162; green,162; blue,162}, dashed] (3.75,1.25) -- (3.75,0);
    \draw [short] (2.5,2.5) -- (2.5,3.75);
    \node [font=\normalsize, color={rgb,255:red,128; green,0; blue,64}] at (3.5,4) {Maximum!};
    \draw [ color={rgb,255:red,128; green,0; blue,128} , dashed] (2.5,3.75) circle (0.25cm);
\end{tikzpicture}
\end{multicols*}

\subsubsection{Herleitung des optimalen Lastwiderstands einer linearen Stromquelle}

\begin{center}
    % LTeX: enabled=false
\begin{circuitikz}
    \tikzstyle{every node}=[font=\normalsize]
    \draw (2,0) to[R,l={ \normalsize $R_i$}] (2,2);
    \draw (2,2) to[short, -o] (2.5,2) ;
    \draw (2,0) to[short, -o] (2.5,0) ;
    \draw (2,0) to[short] (0,0);
    \draw [ color={rgb,255:red,115; green,0; blue,230} ](3.25,2) to[R,l={ \normalsize Rl}] (3.25,0);
    \draw[ color={rgb,255:red,115; green,0; blue,230}, ] (3.25,2) to[short] (2.5,2);
    \draw [ color={rgb,255:red,115; green,0; blue,230}, ](2.5,0) to[short] (3.25,0);
    \draw [ color={rgb,255:red,115; green,0; blue,230}, ->, >=Stealth] (2.75,0.5) -- (3.75,1.5);
    \draw (0,0) to[american current source, l = $I$] (0,2);
    \draw (0,2) to[short] (2,2);
    \node at (2,0) [circ] {};
    \node at (2,2) [circ] {};
\end{circuitikz}
\end{center}

\begin{align*}
    U_{RL} & = \frac{R_i R_l I}{R_i + R_l}\\
    P_{R_l} & = \frac{1}{R_L}(U(R_l))^2\\
    0 & = \frac{d P(R_l)}{d R_l}\\
    0 & = R_l' (U(R_l))^2 + 2\frac{1}{R_L}(U(R_l)')\\
    0 & = -\frac{1}{R_l^2}\frac{R_i^2 R_l^2 I^2}{(R_i + R_l)^2 } + 2\frac{1}{R_l} \frac{R_i R_l I}{R_i + R_l}\left(\frac{R_i I(R_i + R_L)- R_i R_lI}{(R_i + R_l)^2}\right)\\
    0 & = 2 \frac{R_i^3 I^2}{(R_i + R_l)^3} - \frac{R_i^2  I^2}{(R_i + R_l)^2 }\\
    \frac{R_i^2  I^2}{(R_i + R_l)^2 } & = 2 \frac{R_i^3 I^2}{(R_i + R_l)^3}\\
    R_i^2 I^2 (R_i + R_l) & = 2 R_i^3 I^2\\
    2R_I & = R_i + R_l\\
    R_i &= R_l
\end{align*}

Im falle eines nicht linearen Widerstands muss eine Formel für die Leistung über dem Lastwiderstand gefunden werden. 
Die gefundene Formel muss dann abgeleitet und null gesetzt werden, ähnlich wie in der Herleitung.
