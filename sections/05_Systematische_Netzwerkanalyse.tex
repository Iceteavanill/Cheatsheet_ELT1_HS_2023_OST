\section{Systematische Netzwerkanalyse}

{\color{red}Wichtiger Grundsatz}\\
Das zu analysierende Netzwerk \textbf{MUSS} linear sein. 
Ansonsten kann das Netzwerk nicht mit den folgenden Methoden analysiert werden.
Grundsätzlich solle das zu analysierende Netzwerk immer zuerst maximal vereinfacht werden da dies meist zu weniger Aufwand führt. 
Das schlussendliche Ziel ist es eine Matrixgleichung zu erhalten, welche man zuletzt invertieren muss.

\subsection{Auswahl einer geeigneten Methode}

Meisten ist die Wahl der Methode frei. 
Um die optimale zu wählen, sollte immer der Aufwand von allen Methoden bestimmt werden und aufgrund dessen entschieden werden.

\subsection{Graphentheorie}

Um elektrische Netzwerke zu analysieren wird ein mathematisches Teilgebiet namens Graphentheorie angewendet.
Für systematische Netzwerkanalyse müssen ein paar Grundbegriffe geläufig sein:

\begin{itemize}
    \item \textbf{Knoten} : Ein Punkt in welchem Zweige enden
    \item \textbf{Zweig/Kante} : Verbindet Knoten untereinander
    \item \textbf{Weg} : Eine zusammenhängende Reihe von Knoten und zweigen (keine Kante doppelt)
    \item \textbf{Baum} : Ein Weg welcher nirgends in einem Kreis geschlossen ist (sehr bildlich)
    \item \textbf{Spannbaum} : Ein weg welcher alle Knoten miteinander verbindet
    \item \textbf{Zyklus} : Ein weg welcher im gleichen Knoten beginnt und endet
    \item \textbf{Kreis} : Ein Zyklus welcher keinen Knoten zweimal enthält
    \item \textbf{Masche} : Ein Kreis welche Keine inneren oder äussere Zweige aufweisst.
\end{itemize}

\begin{center}
    % LTeX: enabled=false
\begin{circuitikz}
    \tikzstyle{every node}=[font=\normalsize]
    \draw (6.25,10.75) to[short] (6.25,11.25);
    \node at (1.75,10.75) [circ] {};
    \node at (2.75,10.5) [circ] {};
    \node at (2,10.25) [circ] {};
    \node at (2.5,11.25) [circ] {};
    \node [font=\normalsize] at (2,12) {Knoten};
    \node [font=\normalsize] at (4.5,12) {Zweig /Kante};
    \node [font=\normalsize] at (7,12) {Weg (rot)};
    \node [font=\normalsize] at (1.75,8.5) {};
    \node [font=\normalsize] at (9.5,12) {spannbaum (rot)};
    \draw (4.25,11) to[short] (4.25,9.75);
    \node at (4.25,11) [circ] {};
    \node at (4.25,9.75) [circ] {};
    \draw (7.75,10.75) to[short] (7.75,9.75);
    \draw[ color={rgb,255:red,255; green,0; blue,0}, ] (7.75,9.75) to[short] (6.25,9.75);
    \draw (6.25,10.75) to[short] (6.25,9.75);
    \draw [ color={rgb,255:red,255; green,0; blue,0}, ](6.25,10.75) to[short] (7.75,10.75);
    \draw [ color={rgb,255:red,255; green,0; blue,0}, ](7.75,10.75) to[short] (7.75,9.75);
    \node at (6.25,10.75) [circ, color={rgb,255:red,255; green,0; blue,0}] {};
    \node at (7.75,10.75) [circ, color={rgb,255:red,255; green,0; blue,0}] {};
    \draw (6.25,11.25) to[short] (7.75,11.25);
    \draw (7.75,11.25) to[short] (7.75,10.75);
    \node at (6.25,11.25) [circ] {};
    \draw (6.25,11.25) to[short] (5.75,11.25);
    \draw (5.75,11.25) to[short] (5.75,9.75);
    \draw (5.75,9.75) to[short] (6.25,9.75);
    \node at (6.25,9.75) [circ, color={rgb,255:red,255; green,0; blue,0}] {};
    \draw (10.5,10.75) to[short] (10.5,9.75);
    \draw (10.5,9.75) to[short] (9,9.75);
    \draw [ color={rgb,255:red,255; green,0; blue,0}, ](9,10.75) to[short] (9,9.75);
    \draw [ color={rgb,255:red,255; green,0; blue,0}, ](9,10.75) to[short] (10.5,10.75);
    \draw (10.5,10.75) to[short] (10.5,9.75);
    \node at (9,10.75) [circ, color={rgb,255:red,255; green,0; blue,0}] {};
    \node at (10.5,10.75) [circ, color={rgb,255:red,255; green,0; blue,0}] {};
    \draw [ color={rgb,255:red,255; green,0; blue,0}, ](9,10.75) to[short] (9,11.25);
    \draw (9,11.25) to[short] (10.5,11.25);
    \draw (10.5,11.25) to[short] (10.5,10.75);
    \node at (9,11.25) [circ] {};
    \draw (9,11.25) to[short] (8.5,11.25);
    \draw (8.5,11.25) to[short] (8.5,9.75);
    \draw (8.5,9.75) to[short] (9,9.75);
    \node at (9,9.75) [circ, color={rgb,255:red,255; green,0; blue,0}] {};
\end{circuitikz}  
\end{center}

Anzahl an Elementen wird wie folgt angegeben:
\begin{itemize}
    \item Zyklen : z
    \item Knoten : k
    \item Maschen M = z - (k - 1) = z - k + 1
    \item Ideale (nicht lineare) Spannungsquellen : v
    \item Ideale (nicht lineare) Stromquellen : i
\end{itemize}

\subsection{Zweigstrommethode}

Bei der Zweigstrommethode wird das Ohmsche und die beiden krichhoffschen Gesetze systematisch angewendet. 
3 Gruppen an Gleichungen entstehen.

\subsubsection{Zweiggleichungen}

Anzahl: z - i - v\\
Ohmsches Gesetz anwenden. 
Darauf achten das keine Brüche vorkommen

\subsubsection{Stromgleichungen}

Anzahl: k - 1\\
Kirchhoff current law umsetzen. 
Bekannte quellen nach links nehmen.

\subsubsection{Spannungsgleichungen}

Anzahl: z - k + 1\\
Kirchhoff voltage law anwenden. 
Bekannte Quellen auf die linke Seite nehmen.


\subsubsection{Zusammenführen}

Zweiggleichungen in die Strom- und Spannungsgleichungen einsetzen. 
Gleichungstablo aufstellen mit unbekannten als Vektor.

\subsubsection{Beispiel}

In diesem Beispiel wird die maximale Anzahl Knoten verwendet.
Ein Knoten könnte man sparen, allerdings bleibt der Rechenaufwand gleich.

\begin{center}
    % LTeX: enabled=false
\begin{circuitikz}
    \tikzstyle{every node}=[font=\small]
    \draw (3,11.75) to[american voltage source] (3,9.5);
    \draw (3,11.75) to[R] (5.25,11.75);
    \draw (5.25,11.75) to[R] (5.25,9.5);
    \draw (5.25,11.75) to[R] (7.5,11.75);
    \draw (7.5,9.5) to[american current source] (7.5,11.75);
    \draw (3,9.5) to[short] (7.5,9.5);
    \node at (5.25,11.75) [circ] {};
    \node at (7.5,11.75) [circ] {};
    \node at (5.25,9.5) [circ] {};
    \draw [ color={rgb,255:red,255; green,0; blue,0}, ->, >=Stealth] (3.25,11.75) -- (3.5,11.75);
    \node at (3,11.75) [circ] {};
    \node [font=\small, color={rgb,255:red,255; green,0; blue,0}] at (3.75,9.25) {};
    \node [font=\small, color={rgb,255:red,255; green,0; blue,0}] at (3.75,9.25) {};
    \draw [ color={rgb,255:red,255; green,0; blue,0}, ->, >=Stealth] (4.75,11.75) -- (5,11.75);
    \draw [ color={rgb,255:red,255; green,0; blue,0}, ->, >=Stealth] (5.25,10) -- (5.25,9.75);
    \draw [ color={rgb,255:red,255; green,0; blue,0}, ->, >=Stealth] (5.75,11.75) -- (5.5,11.75);
    \draw [ color={rgb,255:red,0; green,0; blue,255}, ->, >=Stealth] (5,11.25) .. controls (4.75,11) and (4.75,10.5) .. (5,10) ;
    \draw [ color={rgb,255:red,0; green,0; blue,255}, ->, >=Stealth] (3.5,11.5) .. controls (3.75,11.25) and (4.5,11.25) .. (4.75,11.5) ;
    \draw [ color={rgb,255:red,0; green,0; blue,255}, ->, >=Stealth] (7,11.5) .. controls (6.5,11.25) and (6.25,11.25) .. (5.75,11.5) ;
    \node [font=\small, color={rgb,255:red,0; green,0; blue,255}] at (2,10.5) {$U_q$};
    \node [font=\small, color={rgb,255:red,0; green,0; blue,255}] at (4,11) {$U_1$};
    \node [font=\small, color={rgb,255:red,0; green,0; blue,255}] at (4.5,10.75) {$U_2$};
    \node [font=\small, color={rgb,255:red,0; green,0; blue,255}] at (6.5,11) {$U_3$};
    \node [font=\small, color={rgb,255:red,255; green,0; blue,0}] at (3.5,12) {$I_0$};
    \node [font=\small, color={rgb,255:red,255; green,0; blue,0}] at (5,12) {$I_1$};
    \node [font=\small, color={rgb,255:red,255; green,0; blue,0}] at (5.5,12) {$I_3$};
    \node [font=\small, color={rgb,255:red,255; green,0; blue,0}] at (5.5,9.75) {$I_2$};
    \node [font=\small, color={rgb,255:red,255; green,0; blue,0}] at (8.25,10.75) {$I_q$};
    \node [font=\small] at (4.25,12.25) {R1};
    \node [font=\small] at (6.25,12.25) {R3};
    \node [font=\small] at (5.75,10.5) {R2};
    \draw [ color={rgb,255:red,0; green,0; blue,255}, ->, >=Stealth] (2.75,11.25) .. controls (2.25,11) and (2.25,10.25) .. (2.75,10) ;
    \draw [ color={rgb,255:red,0; green,0; blue,255}, ->, >=Stealth] (7.25,11.25) .. controls (7,11) and (7,10.5) .. (7.25,10) ;
    \node [font=\small, color={rgb,255:red,0; green,0; blue,255}] at (6.75,10.5) {$U_4$};
\end{circuitikz}
\end{center}
$U_q$ und $I_q$ gegeben\\
Knoten : 4, Zweige : 5\\

\textbf{Zweiggleichungen:}\\
z - v - i = 5 - 1 - 1 = 3

\[
U_1 = R_1 \cdot I_1\\
U_2 = R_2 \cdot I_2\\
U_3 = R_3 \cdot I_3
\]

\textbf{Stromgleichungen:}\\
k - 1 = 4 - 1 = 3 

\[
0 = I_1 + I3 - I_2\\
I_q = I_3\\
0 = I_1 - I_0
\]

\textbf{Spannungsgleichungen:}\\

z - k + 1 = 5 - 4 + 1 = 2

\[
U_q = U_1 + U_2\\
0 = U_2 + U_3 + U_4
\]

\textbf{Einsetze der Zweiggleichungen:}

\[
U_q = R_1 \cdot I_1 + R_2 \cdot I_2\\
0 = R_2 \cdot I_2 + R_3 \cdot I_3 + - U_4\\
0 = I_1 - I_2 + I_3\\
I_q = I_3\\
0 = I_1 - I_0
\]

\textbf{Tablo erstellen:}

\[
\begin{bmatrix}
    \cdot & R_1 & R_2 & \cdot  & \cdot\\
    \cdot & R_2 & R_3 & \cdot  & -1\\
    \cdot & 1 & -1 & 1 & \cdot \\
    \cdot & \cdot & 1 & \cdot  & \cdot\\
    -1 & 1 & \cdot & \cdot  & \cdot\\
\end{bmatrix}
\begin{pmatrix}
    I_0\\
    I_1\\
    I_2\\
    I_3\\
    U_4\\
\end{pmatrix}
=
\begin{pmatrix}
    U_q\\
    \cdot\\
    \cdot\\
    I_q\\
    \cdot\\
\end{pmatrix}
\]    

\subsection{Maschenstrom-/Kreisstrommethode}

\textbf{Aufwand} : M - 1 - i\\

Bei der Kreisstrommethode werden verschiedene Kreisströme als unbekannte definiert. 
Liegen diese auf Maschen spricht man von der Maschentrommethode. 
KVL wird dabei stark genutzt. 
Falls Ströme oder Spannungen gesucht werden, können diese durch einen einzelnen Kreisstrom am einfachsten bestimmt werden. \\
Das Verfahren kann nur bei Planaren Schaltungen verwendet werden. 
Das darf keine Überschneidungen der Zweige geben.    
\textbf{Achtung!} nicht planar wirkende Schaltungen können planar sein, wenn sie etwas umgezeichnet werden:

\begin{center}
    % LTeX: enabled=false
\begin{circuitikz}
    \tikzstyle{every node}=[font=\small]
    \node at (3,0) [circ, color={rgb,255:red,255; green,128; blue,0}] {};
    \draw (0,3.5) to[crossing] (0,4);
    \draw (0,3) to[short] (0,3.5);
    \draw (0,4) to[short] (0,4.25);
    \draw (0,0) to[short] (-0.5,0);
    \draw (-0.5,0) to[short] (-0.5,3.75);
    \draw (-0.5,3.75) to[short] (3,3.75);
    \draw (3,3.75) to[short] (3,3);
    \draw (3,0) to[short] (3.5,0);
    \draw (3.5,0) to[short] (3.5,4.25);
    \draw (3.5,4.25) to[short] (0,4.25);
    \draw (7,3) to[short] (7,0);
    \draw (10,3) to[short] (10,0);
    \draw [short] (7,3) -- (10,0);
    \draw [short] (10,3) -- (7,0);
    \node at (0,3) [circ, color={rgb,255:red,128; green,0; blue,128}] {};
    \node at (3,3) [circ, color={rgb,255:red,0; green,128; blue,255}] {};
    \node at (0,0) [circ, color={rgb,255:red,0; green,255; blue,0}] {};
    \draw [short] (0,0) -- (3,3);
    \draw [short] (0,3) -- (3,0);
    \node at (1.5,1.5) [circ, color={rgb,255:red,255; green,0; blue,0}] {};
    \node at (10,3) [circ, color={rgb,255:red,128; green,0; blue,128}] {};
    \node at (10,0) [circ, color={rgb,255:red,255; green,128; blue,0}] {};
    \node at (7,0) [circ, color={rgb,255:red,0; green,255; blue,0}] {};
    \node at (7,3) [circ, color={rgb,255:red,0; green,128; blue,255}] {};
    \node at (8.5,1.5) [circ, color={rgb,255:red,255; green,0; blue,0}] {};
    \draw [->, >=Stealth] (4.25,1.5) -- (6.5,1.5);
    \node [font=\small] at (5.25,1.75) {umzeichnen};
\end{circuitikz}
\end{center}

\subsubsection{Vorgehen}

\begin{enumerate}
    \item Knoten definieren
    \item Stammbaum definieren\\
    {\color{red}\textbf{Es darf keine Ideale Stromquelle im Pfad des Stammbaums sein}}
    \item Matrix einfüllen:
\end{enumerate}

\begin{tabular}{ccc}
    Widerstände & Kreisstöme & Quellspannungen\\
    $\begin{bmatrix} 
        \mathcolor{orange}{R + R} & R \mathcolor{violet}{+ v}\\ 
        R & \mathcolor{orange}{R + R}\\ 
    \end{bmatrix}$
    &
    $\begin{bmatrix}
        J_1\\
        J_2\\
    \end{bmatrix}$
    &
    $=
    \begin{bmatrix}
        U_{q_1}\\
        U_{q_2}\\
    \end{bmatrix}$
    \\
\end{tabular}
\\
Die Matrix ist grundsätzlich Symmetrisch an der diagonalen. 
In der Diagonalen sind die Summen aller Widerstände im Pfad der Kreisströme. 
Alle Überlagerungen welche durch verschiedene Kreisströme durch denselben Widerstand entstehen, werden jeweils oberhalb und unterhalb davon notiert. 
Diese sind \textbf{negativ}. 
Entgegengesetzte Richtung \say{-} sonst \say{+}. 
Gesteuerte Quellen brechen diese Symetrie.
Wenn Kreisströme nicht unabhängig definiert sind müssen sie trotzdem als Störfaktoren mit einbezogen werden. 
Diese können in den Quellenvektor genommen werden. 
Falls basierend auf vorherige Grössen können sie in die Matrix genommen werden.

\subsubsection{Eifaches Beispiel ohne gesteuerte Quellen}

\begin{center}
    % LTeX: enabled=false
\begin{circuitikz}
    \tikzstyle{every node}=[font=\normalsize]
    \draw (3.75,13.25) to[R] (10.75,13.25);
    \draw (4.75,10) to[R] (4.75,7.5);
    \draw (3.75,11.5) to[short] (3.75,13.25);
    \draw (2.75,7.5) to[american current source] (2.75,10);
    \draw (2.75,10) to[short] (4.75,10);
    \draw (4.75,7.5) to[short] (2.75,7.5);
    \draw (3.75,10) to[short] (3.75,11.5);
    \draw (3.75,11.75) to[R] (7.25,11.75);
    \draw (7.25,11.75) to[R] (10.75,11.75);
    \draw (3.75,6.75) to[short] (3.75,7.5);
    \draw (10.75,11.75) to[short] (10.75,13.25);
    \draw (10.75,11.75) to[R] (10.75,9.75);
    \draw (9,9.75) to[R] (9,8);
    \draw (10.75,9.75) to[american voltage source] (10.75,8.25);
    \draw (9,9.75) to[short] (9,10.25);
    \draw (9,10.25) to[short] (7.25,10.25);
    \draw (9,8) to[short] (9,7.5);
    \draw (9,7.5) to[short] (10.75,7.5);
    \draw (10.75,8.25) to[short] (10.75,7.5);
    \draw (7.25,6.75) to[R] (10.75,6.75);
    \draw (10.75,7.5) to[short] (10.75,6.75);
    \draw (10.75,11.75) to[short] (12.75,11.75);
    \draw (12.75,11.75) to[R] (12.75,5.5);
    \draw (12.75,5.5) to[short] (3.75,5.5);
    \draw (3.75,5.5) to[short] (3.75,6.75);
    \draw (3.75,6.75) to[R] (7.25,6.75);
    \draw (7.25,10.25) to[american voltage source] (7.25,6.75);
    \draw (7.25,10.25) to[short] (7.25,11.75);
    \node at (3.75,7.5) [circ] {};
    \node at (3.75,6.75) [circ] {};
    \node at (3.75,10) [circ] {};
    \node at (3.75,11.75) [circ] {};
    \node at (7.25,10.25) [circ] {};
    \node at (7.25,11.75) [circ] {};
    \node at (10.75,11.75) [circ] {};
    \node at (7.25,6.75) [circ] {};
    \node at (10.75,7.5) [circ] {};
    \node [font=\normalsize] at (2,8.75) {$I_c$};
    \node [font=\normalsize] at (5.25,8.75) {$R_1$};
    \node [font=\normalsize] at (5.5,12.25) {$R_2$};
    \node [font=\normalsize] at (7.25,13.75) {$R_4$};
    \node [font=\normalsize] at (9,12.25) {$R_7$};
    \node [font=\normalsize] at (11.25,10.75) {$R_{10}$};
    \node [font=\normalsize] at (11.5,9) {$U_{10}$};
    \node [font=\normalsize] at (13.25,8.5) {$R_5$};
    \node [font=\normalsize] at (9.5,9) {$R_8$};
    \node [font=\normalsize] at (6.5,8.5) {$U_6$};
    \node [font=\normalsize] at (9,7.25) {$R_9$};
    \node [font=\normalsize] at (5.5,7.25) {$R_3$};
    \draw [ color={rgb,255:red,0; green,128; blue,255}, ->, >=Stealth] (7,12.25) .. controls (6.25,13) and (8.25,13) .. (7.5,12.25) ;
    \node [font=\normalsize, color={rgb,255:red,0; green,128; blue,255}] at (7.25,12.5) {$J_1$};
    \draw [ color={rgb,255:red,0; green,128; blue,255}, ->, >=Stealth] (9.5,10.5) .. controls (8.75,11.25) and (10.75,11.25) .. (10,10.5) ;
    \node [font=\normalsize, color={rgb,255:red,0; green,128; blue,255}] at (9.75,10.75) {$J_4$};
    \draw [ color={rgb,255:red,0; green,128; blue,255}, ->, >=Stealth] (7.75,7.25) .. controls (7,8) and (9,8) .. (8.25,7.25) ;
    \node [font=\normalsize, color={rgb,255:red,0; green,128; blue,255}] at (8,7.5) {$J_3$};
    \draw [ color={rgb,255:red,0; green,128; blue,255}, ->, >=Stealth] (11.25,6.25) .. controls (10.5,7) and (12.5,7) .. (11.75,6.25) ;
    \node [font=\normalsize, color={rgb,255:red,0; green,128; blue,255}] at (11.5,6.5) {$J_5$};
    \draw [ color={rgb,255:red,0; green,128; blue,255}, ->, >=Stealth] (5.5,10.25) .. controls (4.75,11) and (6.75,11) .. (6,10.25) ;
    \node [font=\normalsize, color={rgb,255:red,0; green,128; blue,255}] at (5.75,10.5) {$J_2$};
\end{circuitikz}
\end{center}

\[
\begin{bmatrix}
    R_2 + R_4 + R_7 & - R_2 & \cdot & -R_7 & \cdot \\
    -R_2 & R_1 + R_2 + R_3 & \cdot & \cdot & - R_3 \\
    \cdot & \cdot & R_8 + R_9 & -R_8 & -R_9 \\
    -R_7 & \cdot & -R_8 & R_7 + R_8 + R_{10} & -R_{10} \\
    \cdot & -R_3 & -R_9 & -R_{10} & R_3 + R_5 + R_9 + R_{10} \\
\end{bmatrix}
\begin{bmatrix}
    J_1\\
    J_2\\
    J_3\\
    J_4\\
    J_5\\
\end{bmatrix}
=
\]
\[
\begin{bmatrix}
    \cdot\\
    R_1\cdot I_c - U_6\\
    U_6\\
    -U_{10}\\
    U_{10}\\
\end{bmatrix}
\]

\subsubsection{2. Beispiel mit gesteuerten Quellen}

\begin{center}
    % LTeX: enabled=false
\begin{circuitikz}
    \tikzstyle{every node}=[font=\normalsize]
    \draw (2.25,14) to[american current source] (2.25,11);
    \draw (5.5,11) to[R] (5.5,14);
    \draw (8.75,11) to[R] (8.75,14);
    \draw (2.25,14) to[short] (8.75,14);
    \draw (2.25,11) to[R] (5.5,11);
    \draw (5.5,8) to[R] (5.5,11);
    \draw (2.25,11) to[american controlled voltage source] (2.25,8);
    \draw (8.75,8) to[american controlled current source] (8.75,11);
    \draw (5.5,11) to[american voltage source] (8.75,11);
    \draw (2.25,8) to[short] (5.5,8);
    \draw (5.5,8) to[short] (8.75,8);
    \node [font=\normalsize] at (1.5,12.5) {$I_q$};
    \node [font=\normalsize] at (4,11.5) {$R_1$};
    \node [font=\normalsize] at (6,12.5) {$R_2$};
    \node [font=\normalsize] at (9.25,12.5) {$R_3$};
    \node [font=\normalsize] at (9.5,9) {$g\cdot U_s$};
    \node [font=\normalsize] at (6,9.5) {$R_4$};
    \node [font=\normalsize] at (1.25,9.5) {$r\cdot I_s$};
    \node [font=\normalsize] at (7.25,11.5) {$U_q$};
    \draw [ color={rgb,255:red,255; green,0; blue,0}, ->, >=Stealth] (7,14) -- (7.25,14);
    \draw [ color={rgb,255:red,0; green,128; blue,255}, ->, >=Stealth] (5.25,13.5) .. controls (4.75,13.25) and (4.75,12) .. (5.25,11.5) ;
    \node [font=\normalsize, color={rgb,255:red,0; green,128; blue,255}] at (4.5,12.5) {$U_s$};
    \node [font=\normalsize, color={rgb,255:red,255; green,0; blue,0}] at (7.25,14.5) {$I_s$};
    \draw [ color={rgb,255:red,217; green,0; blue,108}, ->, >=Stealth] (7,9.25) .. controls (6.25,9.75) and (8,9.75) .. (7.25,9.25) node[pos=0.5, fill=white]{$J_1$};
    \draw [ color={rgb,255:red,217; green,0; blue,108}, ->, >=Stealth] (3.5,12.5) .. controls (2.75,13) and (4.5,13) .. (3.75,12.5) node[pos=0.5, fill=white]{$J_1$};
    \draw [ color={rgb,255:red,217; green,0; blue,108}, ->, >=Stealth] (7,12.5) .. controls (6.25,13) and (8,13) .. (7.25,12.5) node[pos=0.5, fill=white]{$J_1$};
    \draw [ color={rgb,255:red,217; green,0; blue,108}, ->, >=Stealth] (3.5,9.25) .. controls (2.75,9.75) and (4.5,9.75) .. (3.75,9.25) node[pos=0.5, fill=white]{$J_1$};
    \node at (2.25,11) [circ] {};
    \node at (5.5,11) [circ] {};
    \node at (5.5,8) [circ] {};
    \node at (8.75,11) [circ] {};
    \node at (5.5,14) [circ] {};
\end{circuitikz}
\end{center}

Vorbereitungsgleichungen:\\
\begin{align*}
    J_1 &= -I_q\\
    J_2 &= I_s\\
    J_4 &= -g\cdot U_s = -g\cdot R_2 \cdot (J_1 - J_2) = g\cdot R_2\cdot (I_q + J_2)\\ 
\end{align*}


gesucht sind $_J2$ \& $J_3$

\[
\begin{bmatrix}
    R_2 + R_3 & \cdot\\
    \cdot & R_1 + R_4\\
\end{bmatrix}
\begin{pmatrix}
    J_2\\
    J_3\\
\end{pmatrix}
=
\begin{pmatrix}
    U_q + \overbrace{J_1 \cdot R_2}^{-I_q \cdot R_2}  \\
    r\cdot \underbrace{I_s}_{J_2} + \underbrace{J_1 \cdot R_1}_{-I_q\cdot R_1} + \underbrace{J_4 \cdot R_4}_{g \cdot R_2 \cdot R_4 \cdot I_q + g \cdot R_2 \cdot R_4 \cdot J_2} \\
\end{pmatrix}
\]
Vereinfachen:

\[
\begin{bmatrix}
    R_2 + R_3 & \cdot\\
    -g \cdot R_2 \cdot R_4 - r & R_1 + R_4\\
\end{bmatrix}
\begin{pmatrix}
    J_2\\
    J_3\\
\end{pmatrix}
=
\begin{pmatrix}
    U_q - I_q \cdot R_2 \\
    -I_q \cdot R_1 + g \cdot R_2 \cdot R_4 \cdot I_q \\
\end{pmatrix}
\]
Überprüfen ob in der Matrix nur Widerstände stehen und im Quellvektor nur Spannunen. 

\subsection{Knotenpotentialmethode}

\textbf{Aufwand:} k - 1 - v\\

Bei der Knotenpotentialmethode wird implizit KCL verwendet. 
Es werden Knoten definiert, welche in den Unekanntenvektor eingefüllt werden. 
Ein Knoten wird als \say{0} definiert. 
Dieser wird nicht in den Vektor eingefügt.
Die Matrix enthält nur Leitwerte.
Die Diagonale sind die Leitwerte, die an den knoten angeschlossen sind. 
Die Positionen neben der diagonalen sind \textbf{immer Negativ}. 
Sie sind die Verbindungen zwischen den jeweiligen knoten. 
Wenn keine gesteuerte Quellen vorhanden sind, treten Widerstände entweder 1 oder 4-mal in der Matrix auf. 
Bis auf die gesteuerten Quellen ist die Matrix symmetrisch. 
Spannungsquellen sind problematisch da sie weder in die Matrix noch in den Quellvektor eingefügt werden können. 
Sie müssen verschoben werden wie im Kapitel \ref{Quelln verschieben} gezeigt.

\subsubsection{Beispiel}

\begin{center}
    \input{circuits/Beispiel Knotenpotentialmethode Schritt 1.tex}
\end{center}

Vorbereitung:

\begin{align*}
    U_3 &= -U_q\\
    U_4 &= U_1 - r\cdot I_s\\
    I_s &= \frac{1}{R_3} \cdot (U_2 - U_3)\\
    U_s &= U_2\\
\end{align*}

Verschieben der Spannungsquellen:

\begin{center}
    % LTeX: enabled=false
\begin{circuitikz}
    \tikzstyle{every node}=[font=\normalsize]
    \draw (2.25,14) to[american current source] (2.25,11);
    \draw (5.5,11) to[R] (5.5,14);
    \draw (8.75,12.5) to[R] (8.75,14);
    \draw (2.25,14) to[short] (8.75,14);
    \draw (2.25,11) to[R] (5.5,11);
    \draw (5.5,9.5) to[R] (5.5,11);
    \draw (8.75,8) to[american controlled current source] (8.75,9.5);
    \draw (2.25,8) to[short] (5.5,8);
    \node [font=\normalsize] at (1.5,12.5) {$I_q$};
    \node [font=\normalsize] at (4,11.5) {$R_1$};
    \node [font=\normalsize] at (6,12.5) {$R_2$};
    \node [font=\normalsize] at (9.25,13.25) {$R_3$};
    \node [font=\normalsize] at (9.75,8.75) {$g \cdot U_s$};
    \node [font=\normalsize] at (6,10.25) {$R_4$};
    \node [font=\normalsize] at (4.5,8.75) {$r \cdot I_s$};
    \node [font=\normalsize] at (7.25,11.5) {$U_q$};
    \draw [ color={rgb,255:red,255; green,0; blue,0}, ->, >=Stealth] (7,14) -- (7.25,14);
    \draw [ color={rgb,255:red,0; green,128; blue,255}, ->, >=Stealth] (5.25,13.5) .. controls (4.75,13.25) and (4.75,12) .. (5.25,11.5) ;
    \node [font=\normalsize, color={rgb,255:red,0; green,128; blue,255}] at (4.5,12.5) {$U_s$};
    \node [font=\normalsize, color={rgb,255:red,255; green,0; blue,0}] at (7.25,14.5) {$I_s$};
    \node at (2.25,11) [circ] {};
    \node at (5.5,11) [circ] {};
    \node at (5.5,8) [circ] {};
    \node at (8.75,11) [circ] {};
    \node at (5.5,14) [circ] {};
    \node [font=\normalsize, color={rgb,255:red,0; green,128; blue,255}] at (1.75,11) {$U_1$};
    \node [font=\normalsize, color={rgb,255:red,0; green,128; blue,255}] at (5.5,14.5) {$U_2$};
    \node [font=\normalsize, color={rgb,255:red,0; green,128; blue,255}] at (9.25,11) {$U_3$};
    \node [font=\normalsize, color={rgb,255:red,0; green,128; blue,255}] at (5.5,7.5) {$U_4$};
    \draw (4.75,10.25) to (4.75,10) node[ground]{};
    \draw [short] (4.75,10.25) -- (5.5,11);
    \node [font=\normalsize] at (4.25,10.25) {$"0"$};
    \draw (8.75,11) to[american voltage source] (8.75,9.5);
    \draw (5.5,11) to[short] (8.75,11);
    \draw (8.75,11) to[american voltage source] (8.75,12.5);
    \draw (5.5,9.5) to[american controlled voltage source] (5.5,8);
    \draw (8.75,8) to[american controlled voltage source] (5.5,8);
    \node [font=\normalsize] at (7.25,8.75) {$r \cdot I_s$};
    \draw [ color={rgb,255:red,255; green,128; blue,0}, ](6.5,8) to[short] (7.75,8);
    \draw [ color={rgb,255:red,255; green,128; blue,0}, ](8.75,9.75) to[short] (8.75,10.75);
    \draw (2.25,11) to[short] (2.25,8);
\end{circuitikz}
\end{center}

\[
\begin{bmatrix}
    \frac{1}{R_1} + \frac{1}{R_4} & \cdot \\
    \cdot & \frac{1}{R_2} + \frac{1}{R_3} \\
\end{bmatrix}
\begin{pmatrix}
    U_1\\
    U_2\\
\end{pmatrix}
=
\begin{pmatrix}
    I_q - \overbrace{-\frac{r\cdot I_s}{R_4}}^{\frac{r \cdot \frac{1}{R_3}(U_2 - U_3)}{R_3\cdot R_4}} - \overbrace{g\cdot U_s}^{g \cdot U_2}\\
    -I_q - \frac{U_q}{R_3}\\
\end{pmatrix}
\]
vereinfachen:
\[
\begin{bmatrix}
    \frac{1}{R_1} + \frac{1}{R_4} & -\frac{r}{R_3\cdot R_4}-g \\
    \cdot & \frac{1}{R_2} + \frac{1}{R_3} \\
\end{bmatrix}
\begin{pmatrix}
    U_1\\
    U_2\\
\end{pmatrix}
=
\begin{pmatrix}
    I_q\\
    -I_q - \frac{U_q}{R_3}\\
\end{pmatrix}
\]

Kontrolle: Matrix enthält nur Leitwert und Quellvektor nur Ströme.

\subsection{Modifizierte Knotenpotentialmethode}

\textbf{Aufwand:} K - 1 + v + s (s sind gesuchte grössen)\\

Die modifizierte Knotenpotentialmethode ist die allgemeinere Form der kpm. 
Mit dieser kann jedes Netzwerk ohne vorherige Veränderung analysiert werden. 
Die Matrix der kpm erhält lediglich mehr Variablen. 
Zu der aufgestellten Matrix der kpm werden für die Ströme in idealen Spannungsquellen, gesuchte Ströme und Ströme in Spulen eine zusätzliche Variable ermittelt.  
Für diese muss dann auch eine zusätzliche Gleichung aufgestellt werden. 
Für Ideale Spannungsquellen wird der Strom generell hinei ndefiniert.
Im Allgemeinen kann die Matrix auch mit Elementstempel aufgestellt werden.
Diese sind für alle Elemente definiert, die in einer Schaltung vorkommen. 

\subsubsection{elementstempel}

Jedes Bauteil in einer zu analysierenden Schaltung fügt bestimmte Teile in eine Matrix / Quellvektor hinzu. 
Da diese immer gleich sind, können diese nach folgendem Muster 1:1 hinzugefügt werden. 
Für Abhängige Grössen wird jeweils eine weitere Gleichung gebraucht welche mit a (auxiliary) benannt wird.  

\begin{multicols*}{2}

    \subsubsection{Widerstand}

    \begin{center}

        % LTeX: enabled=false
\begin{circuitikz}
    \tikzstyle{every node}=[font=\LARGE]
    \draw (0,0) to[R, l={ \Large R}] (2,0);
    \draw (0,0) to[short, -o] (-0.25,0) ;
    \draw (2,0) to[short, -o] (2.25,0) ;
    \node [font=\LARGE] at (-0.25,0.5) {m};
    \node [font=\LARGE] at (2.25,0.5) {n};
\end{circuitikz}

        \begin{tabular}{c|c|c}
            & m & n \\\hline
            m & $+ \frac{1}{R}$ & $- \frac{1}{R}$\\\hline
            n & $- \frac{1}{R}$ & $+ \frac{1}{R}$\\
        \end{tabular}
        \begin{tabular}{|c|}
            q vec.\\\hline
            $\cdot$\\\hline
            $\cdot$\\
        \end{tabular}
    \end{center}

   \subsubsection{Unabhängige Stromquelle}

   \begin{center}

        \input{circuits/Elementstempel unabhängige Stromquelle.tex}

        \begin{tabular}{c|c|c}
           & m & n \\\hline
            m & $\cdot$ & $\cdot$\\\hline
            n & $\cdot$ & $\cdot$\\
        \end{tabular}
        \begin{tabular}{|c|}
            q vec.\\\hline
            $-I_q$\\\hline
            $+I_q$\\
        \end{tabular}   
    \end{center}

    \subsubsection{Widerstand mit gesuchtem Strom}

    \begin{center}

        % LTeX: enabled=false
\begin{circuitikz}
    \tikzstyle{every node}=[font=\LARGE]
    \draw (0,0) to[R, l={ \Large R}] (2,0);
    \draw (0,0) to[short, -o] (-0.25,0) ;
    \draw (2,0) to[short, -o] (2.25,0) ;
    \node [font=\LARGE] at (-0.25,0.5) {m};
    \node [font=\LARGE] at (2.25,0.5) {n};
    \draw [ color={rgb,255:red,255; green,0; blue,0}, ->, >=Stealth] (0,0) -- (0.25,0);
    \node [font=\large, color={rgb,255:red,255; green,0; blue,0}] at (0.25,-0.5) {$I_s$};
\end{circuitikz}

        \[a: U_m - U_n = -R  I_s = 0\]

        \begin{tabular}{c|c|c|c}
            & m & n & $I_s$\\\hline
            m & $\cdot$ & $\cdot$ & 1 \\\hline
            n & $\cdot$ & $\cdot$ & -1\\\hline
            a & 1 & -1 & -R
        \end{tabular}
        \begin{tabular}{|c|}
            q vec.\\\hline
            $\cdot$\\\hline
            $\cdot$\\\hline
            0
        \end{tabular}
    \end{center}

    \subsubsection{Unabhängigeige Spannungsquelle}

    \begin{center}

        \input{circuits/Elementstempel unabhängige Spannungsquelle.tex}

        \[a: U_m - U_n = U_q \]

        \begin{tabular}{c|c|c|c}
            & m & n & $I_s$\\\hline
            m & $\cdot$ & $\cdot$ & 1 \\\hline
            n & $\cdot$ & $\cdot$ & -1\\\hline
            a & 1 & -1 & $\cdot$
        \end{tabular}
        \begin{tabular}{|c|}
            q vec.\\\hline
            $\cdot$\\\hline
            $\cdot$\\\hline
            $U_q$
        \end{tabular}
    \end{center}
    \newcolumn

    \subsubsection{Spannungsgesteuerte Spannungsquelle}
    \begin{center}

        % LTeX: enabled=false
\begin{circuitikz}
    \tikzstyle{every node}=[font=\large]
    \node [font=\LARGE] at (-0.25,0.5) {k};
    \node [font=\LARGE] at (2.25,0.5) {l};
    \draw (3.75,0) to[short, -o] (3.5,0) ;
    \draw (5.75,0) to[short, -o] (6,0) ;
    \node [font=\LARGE] at (3.5,0.5) {m};
    \node [font=\LARGE] at (6,0.5) {n};
    \node at (-0.25,0) [circ] {};
    \node at (2.25,0) [circ] {};
    \draw (3.75,0) to[american controlled voltage source, l = $v_u U_S$] (5.75,0);
    \node [font=\large, color={rgb,255:red,255; green,0; blue,0}] at (4,-0.5) {$I_{qu}$};
    \draw [ color={rgb,255:red,255; green,0; blue,0}, ->, >=Stealth] (3.75,0) -- (4,0);
    \draw [ color={rgb,255:red,0; green,128; blue,255}, ->, >=Stealth] (0,0.25) .. controls (0.25,0.75) and (1.75,0.75) .. (2,0.25) ;
    \node [font=\large, color={rgb,255:red,0; green,128; blue,255}] at (1,1) {$U_s$};
\end{circuitikz}

        \begin{align*}
            m - n &= v_u U_s\\
            U_s &= k-l\\
            0 &= m - n - k v_u \\            
        \end{align*}

        \begin{tabular}{c|c|c|c|c|c}
              & k & l & m & n & $I_q$ \\\hline
            k & $\cdot$ & $\cdot$ & $\cdot$ & $\cdot$ & $\cdot$ \\\hline
            l & $\cdot$ & $\cdot$ & $\cdot$ & $\cdot$ & $\cdot$ \\\hline
            m & $\cdot$ & $\cdot$ & $\cdot$ & $\cdot$ & 1 \\\hline
            n & $\cdot$ & $\cdot$ & $\cdot$ & $\cdot$ & -1 \\\hline
            a & $-v_u$ & $v_u$ & 1 & -1 & $\cdot$ \\
        \end{tabular}
        \begin{tabular}{|c|}
            q vec. \\\hline
            $\cdot$ \\\hline
            $\cdot$ \\\hline
            $\cdot$ \\\hline
            $\cdot$ \\\hline
            $\cdot$ \\
        \end{tabular}

    \end{center}

    \subsubsection{Spannungsgesteuerte Stromquelle}
    \begin{center}

        % LTeX: enabled=false
\begin{circuitikz}
    \tikzstyle{every node}=[font=\large]
    \node [font=\LARGE] at (-0.25,0.5) {k};
    \node [font=\LARGE] at (2.25,0.5) {l};
    \draw (3.75,0) to[short, -o] (3.5,0) ;
    \draw (5.75,0) to[short, -o] (6,0) ;
    \node [font=\LARGE] at (3.5,0.5) {m};
    \node [font=\LARGE] at (6,0.5) {n};
    \node at (-0.25,0) [circ] {};
    \node at (2.25,0) [circ] {};
    \draw (3.75,0) to[american controlled current source, l = $g U_S$] (5.75,0);
    \draw [ color={rgb,255:red,0; green,128; blue,255}, ->, >=Stealth] (0,0.25) .. controls (0.25,0.75) and (1.75,0.75) .. (2,0.25) ;
    \node [font=\large, color={rgb,255:red,0; green,128; blue,255}] at (1,1) {$U_s$};
\end{circuitikz}        

        \begin{align*}
            U_s = k - l\\
            I_{mn} = g(k-l)\\
            I_{mn} = gk - gl
        \end{align*}

        \begin{tabular}{c|c|c|c|c}
              & k & l & m & n \\\hline
            k & $\cdot$ & $\cdot$ & $\cdot$ & $\cdot$  \\\hline
            l & $\cdot$ & $\cdot$ & $\cdot$ & $\cdot$ \\\hline
            m & g & -g & $\cdot$ & $\cdot$ \\\hline
            n & -g & g & $\cdot$ & $\cdot$ \\
        \end{tabular}
        \begin{tabular}{|c|}
            q vec. \\\hline
            $\cdot$ \\\hline
            $\cdot$ \\\hline
            $\cdot$ \\\hline
            $\cdot$ \\
        \end{tabular}

    \end{center}
\end{multicols*}

\begin{multicols*}{2}
    
    \subsubsection{Stromgesteuerte Spannungsquelle}
    \begin{center}

        % LTeX: enabled=false
\begin{circuitikz}
    \tikzstyle{every node}=[font=\large]
    \node [font=\LARGE] at (-0.25,0.5) {k};
    \node [font=\LARGE] at (2.25,0.5) {l};
    \draw (3.75,0) to[short, -o] (3.5,0) ;
    \draw (5.75,0) to[short, -o] (6,0) ;
    \node [font=\LARGE] at (3.5,0.5) {m};
    \node [font=\LARGE] at (6,0.5) {n};
    \node at (-0.25,0) [circ] {};
    \node at (2.25,0) [circ] {};
    \draw (3.75,0) to[american controlled current source,l={ \large $v_i I_s$}] (5.75,0);
    \draw (-0.25,0) to[short] (2.25,0);
    \node [font=\Large, color={rgb,255:red,255; green,0; blue,0}] at (4.5,-0.5) {};
    \draw [ color={rgb,255:red,255; green,0; blue,0}, ->, >=Stealth] (1,0) -- (1.25,0);
    \node [font=\large, color={rgb,255:red,255; green,0; blue,0}] at (1,0.5) {$I_s$};
\end{circuitikz}

        \[U_k - U_l = 0\]

        \begin{tabular}{c|c|c|c|c|c}
              & k & l & m & n & $I_s$  \\\hline
            k & $\cdot$ & $\cdot$ & $\cdot$ & $\cdot$ & 1 \\\hline
            l & $\cdot$ & $\cdot$ & $\cdot$ & $\cdot$ & -1 \\\hline
            m & $\cdot$ & $\cdot$ & $\cdot$ & $\cdot$ & $v_i$ \\\hline
            n & $\cdot$ & $\cdot$ & $\cdot$ & $\cdot$ & $-v_i$  \\\hline
            $a_1$ & 1 & -1 & $\cdot$ & $\cdot$ &$\cdot$ \\
        \end{tabular}
        \begin{tabular}{|c|}
            q vec. \\\hline
            $\cdot$ \\\hline
            $\cdot$ \\\hline
            $\cdot$ \\\hline
            $\cdot$ \\\hline
            $\cdot$ \\
        \end{tabular}
    \end{center}

    \newcolumn

    \subsubsection{Stromgesteuerte Stromquelle}
    \begin{center}

        \input{circuits/Elementstempel Stromgesteuerte Stromquelle.tex}
        \begin{align*}
            a_1&: 0 = k - l\\
            a_2&: r I_s = m - n\\
        \end{align*}
        \begin{tabular}{c|c|c|c|c|c|c}
              & k & l & m & n & $I_s$ & $I_{qu}$ \\\hline
            k & $\cdot$ & $\cdot$ & $\cdot$ & $\cdot$ & 1 & $\cdot$  \\\hline
            l & $\cdot$ & $\cdot$ & $\cdot$ & $\cdot$ & -1 & $\cdot$ \\\hline
            m & $\cdot$ & $\cdot$ & $\cdot$ & $\cdot$ & $\cdot$ & 1 \\\hline
            n & $\cdot$ & $\cdot$ & $\cdot$ & $\cdot$ & $\cdot$ & -1 \\\hline
            $a_1$ & 1 & -1 & $\cdot$ & $\cdot$ & $\cdot$ & $\cdot$ \\\hline
            $a_2$ & $\cdot$ & $\cdot$ & 1 & -1 & -r & $\cdot$ \\
        \end{tabular}
        \begin{tabular}{|c|}
            q vec. \\\hline
            $\cdot$ \\\hline
            $\cdot$ \\\hline
            $\cdot$ \\\hline
            $\cdot$ \\\hline
            $\cdot$ \\\hline
            $\cdot$ \\
        \end{tabular}
    \end{center}

\end{multicols*}

\subsubsection{Beispiel}

\begin{center}
    % LTeX: enabled=false
\begin{circuitikz}
    \tikzstyle{every node}=[font=\normalsize]
    \draw (1.75,10.5) to[american controlled voltage source] (1.75,13.25);
    \draw (1.75,13.25) to[R] (4.5,13.25);
    \draw (4.5,13.25) to[R] (4.5,10.5);
    \draw (4.5,13.25) to[american current source] (7.25,13.25);
    \draw (4.5,13.25) to[short] (4.5,14.5);
    \draw (7.25,13.25) to[short] (7.25,14.5);
    \draw (4.5,14.5) to[R] (7.25,14.5);
    \draw (7.25,13.25) to[R] (7.25,10.5);
    \draw (10,10.5) to[american controlled current source] (10,13.25);
    \draw (1.75,10.5) to[short] (10,10.5);
    \draw (10,13.25) to[short] (7.25,13.25);
    \node at (4.5,13.25) [circ] {};
    \node at (7.25,13.25) [circ] {};
    \node at (4.5,10.5) [circ] {};
    \node at (7.25,10.5) [circ] {};
    \node [font=\normalsize, color={rgb,255:red,128; green,0; blue,128}] at (1.75,13.5) {$1$};
    \node [font=\normalsize, color={rgb,255:red,128; green,0; blue,128}] at (4.25,13.5) {$2$};
    \node [font=\normalsize, color={rgb,255:red,128; green,0; blue,128}] at (7,13.5) {$3$};
    \node [font=\normalsize, color={rgb,255:red,128; green,0; blue,128}] at (0.75,12) {$v_u \cdot U_S$};
    \node [font=\normalsize] at (3,13.75) {$R_1$};
    \node [font=\normalsize] at (4,12) {$R_2$};
    \node [font=\normalsize] at (5.75,12.5) {$I_q$};
    \node [font=\normalsize] at (6,15) {$G_3$};
    \node [font=\normalsize] at (11,11.75) {$V_i \cdot I_s$};
    \node [font=\normalsize] at (6.75,11.75) {$G_4$};
    \draw (2.25,10.5) to (2.25,10.25) node[ground]{};
    \node at (2.25,10.5) [circ] {};
    \draw [ color={rgb,255:red,255; green,0; blue,0}, ->, >=Stealth] (1.75,12.75) -- (1.75,13);
    \draw [ color={rgb,255:red,255; green,0; blue,0}, ->, >=Stealth] (4.5,11) -- (4.5,10.75);
    \node [font=\normalsize, color={rgb,255:red,255; green,0; blue,0}] at (1.5,13) {$I_q$};
    \node [font=\normalsize, color={rgb,255:red,255; green,0; blue,0}] at (4.75,10.75) {$I_s$};
    \draw [ color={rgb,255:red,0; green,128; blue,255}, ->, >=Stealth] (7.5,12.75) .. controls (8,12.5) and (8,11.25) .. (7.5,11) ;
    \node [font=\normalsize, color={rgb,255:red,0; green,128; blue,255}] at (8.25,12) {$U_s$};
\end{circuitikz}
\end{center}

\[
\begin{bmatrix}
    \frac{1}{R_1} & -\frac{1}{R_1} & \cdot & -1 & \cdot \\
    -\frac{1}{R_1} & \cellcolor{red!30}\frac{1}{R_1} + G_3 & - G_3 & \cdot & 1 \\
    \cdot & -G_3 & G_3 + G_4 & \cdot & v_i \\
    -1 & \cdot & -v_u & \cdot & \cdot \\ 
    \cdot & 1 & \cdot & \cdot &\cellcolor{red!30}-R_2\\
\end{bmatrix} 
\begin{pmatrix}
    U_1 \\
    U_2 \\
    U_3 \\
    I_q \\
    I_s \\
\end{pmatrix}
=
\begin{pmatrix}
    \cdot \\
    -I_q \\
    I_q \\
    \cdot \\
    \cdot \\
\end{pmatrix}
\]

An der markierten Stelle würde eigentlich $\frac{1}{R_2}$ Stehen laut NA.
Da aber ein Strom des Widerstandes gesucht wird, ist dieser nach unten rechts gerutscht da er ansonsten doppelt vorkommt. 
Siehe Elementstempel.
