\section{Gleichstromnetzwerke}

Generell wird im folgenden Abschnitt nur lineare Gleichstromnetzwerke behandelt. 
Ein Netzwerk besteht in der Regel aus mindestens 2 verknüpften Komponenten. 
Alle Verbindungen werden als ideale Verbindung zu betrachten.

\subsection{Ideale Quellen}

\subsubsection{Spannungsquelle}

Eine Ideale Spannungsquelle legt fest das über den Anschlüssen die festgelegte Spannung herrscht. 
Es ist egal, wie gross die Last ist. 
Die Spannung bleibt gleich.\\
Es gibt verschiedene Symbole:\\
\begin{center}
    \input{circuits/spannungsquellen.tex}
    \begin{itemize}
        \item \textbf{Typ 1} stellt eine Batterie dar. Durch das Symbol ist klar welche Polarität herrscht. Dh kann nur eine Spannung ohne Spannungspfeil gezeichnet werden.
        \item \textbf{Typ 2} ist die Europäische Darstellungsart. Sie ist weniger gebräuchlich da zwingend immer ein Pfeil mit der Spannung gezeichnet werden muss.
        \item \textbf{Typ 3} ist die Amerikanische Variante. Sie ist die gebräuchlichste Variante da die Polarität auch klargestellt ist.
    \end{itemize}
\end{center}

\subsubsection{Stromquelle}

Eine Ideale Stromquelle legt fest das durch das Symbol der festgelegte Stom fliesst. 
Es ist egal wie gross die Last ist. 
Der Strom fliesst durch.\\
Es gibt verschiedene Symbole:\\
\begin{center}
    % LTeX: enabled=false
\begin{circuitikz}[european]
    \draw (0,0) to[isource, *-*, i=$I_q$, bipole/is voltage=false, !vi, name=I1] (0,2);
    \iarr{I1};
    \node at(0,-0.5) {Typ 1};
    \draw (2,0) to[ioosource, *-*, i=$I_q$, bipole/is voltage=false, !vi, name=I2] (2,2);
    \iarr{I2};
    \node at(2,-0.5) {Typ 2};
\end{circuitikz}
\begin{circuitikz}
    \node at(0,-0.5) {};
    \draw (1,0) to[isource, *-*, i=$I_q$, bipole/is voltage=false, !vi, name=I1] (1,2);
    \iarr{I1};
    \node at(1,-0.5) {Typ 3};
\end{circuitikz}
\end{center}

\begin{itemize}
    \item \textbf{Typ 1} ist die Europäische Darstellungsart. Sie braucht ein Strompfeil da die Richtung des Stromes nicht definiert ist.
    \item \textbf{Typ 2} braucht auch ein Strompfeil, aber nicht sehr gebräuchlich.
    \item \textbf{Typ 3} ist die amerikanische Art und am gebräuchlichsten da die Stromrichtung schon definiert ist.
\end{itemize}

\subsubsection{Aussage ob Quelle oder keine Quelle}

Die obigen Quellen sagen lediglich aus, ob eine Spannung oder ein Strom an einem Punkt existiert. 
Ob die Quelle schlussendlich wirklich eine Quelle ist, daher Energie ins System liefert, muss durch abgleich mit der korrespondierenden Spannung / Strom über der Quelle (siehe Kapitel \ref{source_sink}).\\

\subsubsection{Widersprüche}

Es gibt Widersprüche welche durch die obigen Quellen entstehen können:\\
\begin{center}
    \input{circuits/Widersprüche.tex}
\end{center}

Die beiden obigen Schaltungen sind Widersprüchlich, hervorgerufen durch (oder fehlenen) roten idealen Verbindungen.\\ 
In Schaltung 1 sind die beiden Pole einer Idealen Spannungsquelle verbunden. 
Allerdings ist sie Spannung grösser als 0. 
Die Spannung kann nicht grösser als 0 sein, aber an beiden Polen gleich. 
Ein Widerspruch.\\
In Schaltung 2 erzeugt die ideale Stromquelle ein Strom der grösser ist als 0, dieser kann allerdings nicht fliessen da keine Verbindung vorhanden ist. 
Ein Widerspruch.\\

\subsection{Gesteuerte Quellen}

In der Elektrotechnik viel gebrauchte Elemente sind sogenannte gesteuerte Quellen. 
Sie haben eine Messgrösse, welcher dann mit dem Steuerparameter multipliziert wird und dann als Ausgangsgrösse ausgegeben wird. 
Die Messgrösse und Ausgangsgrösse kann entweder Spannung oder Strom sein. 
Daher gibt es 4 verschiedene Arten Gesteuerte Quellen.\\
\begin{itemize}
    \item Spannungsgesteuerte Spannungsquelle (VCVS)
    \item Spannungsgesteuerte Stromquelle (VCCS)
    \item Stromgesteuerte Spannungsquelle (CCVS)
    \item Stromgesteuerte Stromquelle (CCCS)
\end{itemize}
Der Steuerparameter der VCVS und CCCS sind einheitenlose Grössen, welche meist mit Symbolen wie $g, v, r  \text{ oder } \gamma$ bezeichnet. 
Die Steuerparameter der VCCS und CCVS müssen die Messgrösse in die Ausgangsgrösse umwandeln und haben dadurch eine Einheit.\\
$\text{VCCS} \xrightarrow{} \unit{\ohm}$\\
$\text{CCVS} \xrightarrow{} \unit{\siemens}$\\
Symbole und dessen Messklemmen:\\

\begin{center}
\begin{tabular}{|c||c|}\hline
    VCVS & VCCS\\\hline
    % LTeX: enabled=false
\begin{circuitikz}
    \draw (0,2) -- (1,2) to[open,o-o] (1,0);
    \draw (0,0) -- (1,0);
    \draw[-Latex,blue] (1,1.8) to[bend left, blue]  node[midway, right] {$U_m$}(1,0.2);
\end{circuitikz} 
\begin{circuitikz}[american]
    \node at(0,2.5){};
    \draw (0.5,2) to[cvsource,o-o,l=$\gamma\cdot I_m$] (0.5,0);
\end{circuitikz} & % LTeX: enabled=false
\begin{circuitikz}
    \draw (0,2) -- (1,2) to[open,o-o] (1,0);
    \draw (0,0) -- (1,0);
    \draw[-Latex,blue] (1,1.8) to[bend left, blue]  node[midway, right] {$U_m$}(1,0.2);
\end{circuitikz} 
\begin{circuitikz}[american]
    \node at(0,2.5){};
    \draw (0.5,0) to[cisource,o-o,l_=$\unit{\siemens}\cdot u_m$] (0.5,2);
\end{circuitikz}
     \\\hline\hline
    CCCS & CCVS\\\hline
    % LTeX: enabled=false
\begin{circuitikz}
    \draw (1,2) to[short,o-o,i=$I_m$, !vi, name=I1] (1,0);
    \iarr{I1};
\end{circuitikz} 
\begin{circuitikz}[american]
    \node at(0,2.5){};
    \draw (0.5,0) to[cisource,o-o,l_=$\unit{\siemens}\cdot I_m$] (0.5,2);
\end{circuitikz} & % LTeX: enabled=false
\begin{circuitikz}
    \draw  (1,2) to[short,o-o,i=$I_m$, !vi, name=I1] (1,0);
     \iarr{I1};
\end{circuitikz} 
\begin{circuitikz}[american]
    \node at(0,2.5){};
    \draw (0.5,2) to[cvsource,o-o,l=$\gamma\cdot I_m$] (0.5,0);
\end{circuitikz} 
     \\\hline
\end{tabular}
\end{center}

\subsection{Lineare Quellen}

Eine lineare Quelle besteht aus einer Idealen Quelle und einem (linearen) Widerstand. 
Die Spannung der Quelle verhält sich linear zu dem Strom, der aus der Quelle fliesst. 
Mann kann eine lineare Quelle mit einer Spannungs- oder Stromquelle realisieren. 
Man unterscheidet dabei zwischen einer Thévenin oder einer Norton Quelle. 
Lineare Quellen haben 3 verschiedene Eigenschaften. 
\begin{itemize}
    \item \textbf{Leerlaufspannung} ist die Spannung welche an der Ausgangsklemme anliegt wenn keine last anliegt
    \item \textbf{Kurzschlussstom} ist der maximale Strom welche durch einen kurzschluss an der Ausgangsklemme angelegt erzeugt werden kann
    \item \textbf{Innenwiderstand} ist der innere Widerstand der Quelle
\end{itemize}

Wenn 2 dieser Eigenschaften bekannt sind, kann die letzte berechnet werden da diese linear abhängig ist.

\subsubsection{Thévenin Quelle}

Eine Thévenin Quelle besteht aus einer Idealen Spannungsquelle und einem Widerstand in der folgenden Konstellation:

\begin{center}
    % LTeX: enabled=false
\begin{circuitikz}
    \draw (0,2) to[vsource,l_=$U_q$] (0,0)
     to[short,-o] (2,0);
    \draw (0,2) to[R,l=$R_i$,-o] (2,2);
    \draw (2,0) to[open,name=opn,v] (2,2);
    \draw[-Latex,blue] (opn-Vto) to[bend left, blue]  node[midway, right] {$U_{\text{out}}$}(opn-Vfrom);
\end{circuitikz}
\end{center}

\subsubsection{Norton Quelle}

Eine Nortonquelle besteht aus einer Idealen Stromquelle und einem Widerstand in der folgenden Konstellation:

\begin{center}
    % LTeX: enabled=false
\begin{circuitikz}
    \draw (0,0) to[I,l=$I_q$] (0,2)
    -- (2,2) to[R,l=$R_i$,*-*] (2,0) -- (0,0);
    \draw (2,2)to[short, -o] (4,2)
    to[open,name=opn,v] (4,0)
    to[short, o-] (2,0);
    \draw[-Latex,blue] (opn-Vfrom) to[bend left, blue]  node[midway, right] {$U_{\text{out}}$}(opn-Vto);
\end{circuitikz}
\end{center}

\subsection{Helmholz, Thévenin, Norton - Theorem}

Das Helmholz- / Thévenin- und das Nortontheorem sagen etwas sehr Ähnliches aus. 
Helmholz / Thévenin stellen fest das alle Kombinationen aus linearen Widerständen und idealen Spannungsquellen auf eine Thévenin Quelle reduziert werden können. 
Norton hat das erweitert das man auch auf eine Norton Quelle reduzieren kann. 
Daher, wenn man eine Thévenin quelle hat, kann man diese zu einer Norton Quelle umwandeln und umgekehrt. 
Dies kann sehr nützlich sein, um lineare Netzwerke zu lösen.

\subsection{Superposition / Überlagerungssatz}

Superposition ermöglicht das schrittweise Erkunden von Schaltungen durch Ausschalten von idealen Quellen bis auf eine und das anschliessende überlagern(summieren der strömen). 
Voraussetzung ist eine Lineare Schaltung. 
Alle passiven Bauteile bleiben immer gleich. 
Es werden nur die Quellen verändert. 
Wenn abhängige Quellen, welche wirklich abhängig sind, im Netzwerk vorhanden sind dann dürfen diese nicht ausgeschaltet werden und müssen immer eingeschalten bleiben. 
Thévenin und Norton Quellen werden verschieden abgeschaltet. 
Eine Thévenin Quellen wird zu einer idealen Verbindung (nicht kurzgeschlossen!). 
Eine Norton Quelle wird zu einem idealen Unterbruch.

\begin{center}
    % LTeX: enabled=false
\begin{circuitikz}
    \draw (1,2) to[V,v,*-*,name=v1, bipole/is voltage=false] (1,0);
    \draw (0.5,0) -- (1.5,0);
    \draw (0.5,2) -- (1.5,2);
    \draw[-Latex,blue] (v1-Vfrom) to[bend left=55, blue]  node[midway, right] {$U_q$}(v1-Vto);

    \draw[-Latex,orange] (2.5,1) -- node[midway, above] {ausschalten} (3.5,1);

    \draw (5,2) to[short,*-*] (5,0);
    \draw (4.5,0) -- (5.5,0);
    \draw (4.5,2) -- (5.5,2);
\end{circuitikz}
\begin{circuitikz}
    \draw (1,0) to[I,i=$I_q$,*-*,name=i1, bipole/is voltage=false] (1,2);
    \draw (0.5,0) -- (1.5,0);
    \draw (0.5,2) -- (1.5,2);
    \iarr{i1};
    \draw[-Latex,orange] (2.5,1) -- node[midway, above] {ausschalten} (3.5,1);

    \draw (5,0.5) to[short,-*] (5,0);
    \draw (5,1.5) to[short,-*](5,2);
    \draw (4.5,0) -- (5.5,0);
    \draw (4.5,2) -- (5.5,2);
\end{circuitikz}
\end{center}

\subsubsection{Leistung bei Superposition}

Leistung darf während dem Superpositionieren nicht separat berechnet werden. 
Es müssen alle Quellen superponiert werden, summiert und dann darf erst Leistung berechnet werden. 
Ansonsten stimmt das Resultat nicht.

\subsubsection{Beispiel}

\begin{center}
    % LTeX: enabled=false
\begin{circuitikz}
    \tikzstyle{every node}=[font=\normalsize]
    \draw (0,0) to[american current source] (0,3);
    \draw (9,0) to[american current source] (9,3);
    \draw (3,3) to[american voltage source] (3,0);
    \draw (3,3) to[R,l={ \normalsize $3\Omega$}] (6,3);
    \draw (3,0) to[R,l={ \normalsize 3$\Omega$}] (6,0);
    \draw (6,0) to[R] (6,3);
    \draw (6,3) to[short] (9,3);
    \draw (0,3) to[short] (3,3);
    \draw (0,0) to[short] (3,0);
    \draw (6,0) to[short] (9,0);
    \node at (3,0) [circ] {};
    \node at (3,3) [circ] {};
    \node at (6,3) [circ] {};
    \node at (6,0) [circ] {};
    \node [font=\normalsize] at (6.5,1.5) {-3$\Omega$};
    \node [font=\normalsize] at (-1,1.5) {1A};
    \node [font=\normalsize] at (3.75,1.5) {3V};
    \draw [ color={rgb,255:red,255; green,0; blue,0}, short] (0.25,2) -- (0.25,2.75);
    \draw [ color={rgb,255:red,255; green,0; blue,0}, short] (1.5,2.75) -- (2.75,2.75);
    \draw [ color={rgb,255:red,255; green,0; blue,0}, short] (2.75,1) -- (2.75,0.25);
    \draw [ color={rgb,255:red,255; green,0; blue,0}, short] (1.25,0.25) -- (0.25,0.25);
    \draw [ color={rgb,255:red,255; green,0; blue,0}, ->, >=Stealth] (2.75,0.25) -- (1.25,0.25);
    \draw [ color={rgb,255:red,255; green,0; blue,0}, ->, >=Stealth] (0.25,0.25) -- (0.25,1);
    \draw [ color={rgb,255:red,255; green,0; blue,0}, ->, >=Stealth] (0.25,2.75) -- (1.5,2.75);
    \draw [ color={rgb,255:red,255; green,0; blue,0}, ->, >=Stealth] (2.75,2.75) -- (2.75,2);
    \node [font=\normalsize, color={rgb,255:red,255; green,0; blue,0}] at (1.25,3.5) {1A};
    \node [font=\normalsize] at (9.75,1.5) {1A};
    \draw [ color={rgb,255:red,185; green,0; blue,185}, short] (3.25,2) -- (3.25,3.25);
    \draw [ color={rgb,255:red,185; green,0; blue,185}, short] (3.25,3.25) -- (3.75,3.25);
    \draw [ color={rgb,255:red,185; green,0; blue,185}, short] (9.25,2) -- (9.25,3.25);
    \draw [ color={rgb,255:red,185; green,0; blue,185}, ->, >=Stealth] (5,3.25) -- (5.75,3.25);
    \draw [ color={rgb,255:red,185; green,0; blue,185}, ->, >=Stealth] (6.25,3.25) -- (6.25,2.25);
    \draw [ color={rgb,255:red,185; green,0; blue,185}, short] (6.25,0.75) -- (6.25,-0.25);
    \draw [ color={rgb,255:red,185; green,0; blue,185}, ->, >=Stealth] (6.25,-0.25) -- (5.25,-0.25);
    \draw [ color={rgb,255:red,185; green,0; blue,185}, short] (3.75,-0.25) -- (3.25,-0.25);
    \draw [ color={rgb,255:red,185; green,0; blue,185}, short] (3.25,-0.25) -- (3.25,1);
    \draw [ color={rgb,255:red,185; green,0; blue,185}, ->, >=Stealth] (6.25,-0.25) -- (7.25,-0.25);
    \draw [ color={rgb,255:red,185; green,0; blue,185}, short] (7.25,-0.25) -- (9.25,-0.25);
    \draw [ color={rgb,255:red,185; green,0; blue,185}, ->, >=Stealth] (9.25,-0.25) -- (9.25,1);
    \draw [ color={rgb,255:red,185; green,0; blue,185}, ->, >=Stealth] (9.25,3.25) -- (7.5,3.25);
    \draw [ color={rgb,255:red,185; green,0; blue,185}, short] (7.5,3.25) -- (5.75,3.25);
    \draw [ color={rgb,255:red,0; green,106; blue,0}, ->, >=Stealth] (3.5,2.75) -- (3.75,2.75);
    \draw [ color={rgb,255:red,0; green,106; blue,0}, ->, >=Stealth] (5.75,2.75) -- (5.75,2.25);
    \draw [ color={rgb,255:red,0; green,106; blue,0}, ->, >=Stealth] (5.75,0.25) -- (5.25,0.25);
    \draw [ color={rgb,255:red,0; green,106; blue,0}, ->, >=Stealth] (3.5,0.25) -- (3.5,1);
    \draw [ color={rgb,255:red,0; green,106; blue,0}, short] (5.75,0.25) -- (5.75,0.75);
    \draw [ color={rgb,255:red,0; green,106; blue,0}, short] (3.75,0.25) -- (3.5,0.25);
    \draw [ color={rgb,255:red,0; green,106; blue,0}, short] (3.5,2.75) -- (3.5,1.75);
    \draw [ color={rgb,255:red,0; green,106; blue,0}, short] (5.75,2.75) -- (5.25,2.75);
    \node [font=\normalsize, color={rgb,255:red,0; green,106; blue,0}] at (4,2.5) {1A};
    \node [font=\normalsize, color={rgb,255:red,0; green,106; blue,0}] at (5.25,0.5) {1A};
    \node [font=\normalsize, color={rgb,255:red,185; green,0; blue,185}] at (5.5,3.75) {1A};
    \node [font=\normalsize, color={rgb,255:red,185; green,0; blue,185}] at (6.75,2.5) {2A};
    \node [font=\normalsize, color={rgb,255:red,185; green,0; blue,185}] at (7.25,3.75) {1A};
    \node [font=\normalsize, color={rgb,255:red,185; green,0; blue,185}] at (5.5,-0.75) {1A};
\end{circuitikz}
\end{center}

In diesem Beispiel wurden alle Quellen nacheinander ausgeschaltet und die Ströme eingezeichnet. 
Nun können die Ströme summiert und die Leistungen bestimmt werden.
\subsection{Reziprozität /Kirchhoffscher Umkehrungssatz}


In einem Netzwerk mit nur einer realen Idealen Quelle kann die Quelle und eine Messgrösse ausgetauscht werden. 
Das Verhältnis bleibt gleich. 
Die Quellgrösse, welche im veränderten Netzwerk angelegt wird, ist aber nicht gleich.

\subsubsection{Herleitung mit Spannungsquelle}

\begin{multicols*}{2}
    \begin{center}
        % LTeX: enabled=false
\resizebox{\columnwidth}{!}{%
\begin{circuitikz}
    \tikzstyle{every node}=[font=\normalsize]
    \draw (0,3) to[american voltage source] (0,0);
    \draw (0,3) to[R] (3,3);
    \draw (3,3) to[R] (3,0);
    \draw (3,3) to[R] (6,3);
    \draw (6,3) to[R] (6,0);
    \draw (6,0) to[short] (0,0);
    \node at (3,3) [circ] {};
    \node at (3,0) [circ] {};
    \node [font=\normalsize] at (1.5,3.5) {$R_1$};
    \node [font=\normalsize] at (0.75,1.5) {$U_q$};
    \node [font=\normalsize] at (3.5,1.5) {$R_2$};
    \node [font=\normalsize] at (4.5,3.5) {$R_3$};
    \node [font=\normalsize] at (6.5,1.5) {$R_4$};
    \draw [ color={rgb,255:red,255; green,0; blue,0}, ->, >=Stealth] (6,0.5) -- (6,0.25);
    \node [font=\normalsize, color={rgb,255:red,255; green,0; blue,0}] at (6.25,0.25) {$I_1$};
\end{circuitikz}
}%
    \end{center}
    \begin{align*}
        I_1 &= \frac{U_q}{R_1 + (R_2 \parallel (R_3 + R_4))} \cdot \frac{R_2}{R_1 + R_3 + R_4}\\
        I_1 &= \frac{U_q R_2}{\left(R_1 + \frac{R_2(R_3 + R_4)}{R_1 + R_3 + R_4}\right)\left(R_2 + R_3 + R_4\right)}\\
        I_1 &= \frac{U_q R_2}{R_1(R_1 + R_3 + R_4) + R_2(R_3 + R_4)}
    \end{align*}
    \newcolumn

    \begin{center}
        % LTeX: enabled=false
\resizebox{\columnwidth}{!}{%
\begin{circuitikz}        
    \tikzstyle{every node}=[font=\LARGE]
    \draw (0,3) to[R] (3,3);
    \draw (3,3) to[R] (3,0);
    \draw (3,3) to[R] (6,3);
    \draw (6,3) to[R] (6,1.5);
    \draw (6,0) to[short] (0,0);
    \node at (3,3) [circ] {};
    \node at (3,0) [circ] {};
    \node [font=\normalsize] at (1.5,3.5) {$R_1$};
    \node [font=\normalsize] at (6.75,0.75) {$U_q$};
    \node [font=\normalsize] at (3.5,1.5) {$R_2$};
    \node [font=\normalsize] at (4.5,3.5) {$R_3$};
    \node [font=\normalsize] at (6.5,2.5) {$R_4$};
    \node [font=\normalsize, color={rgb,255:red,255; green,0; blue,0}] at (0.5,1.75) {$I_2$};
    \draw (0,0) to[short] (0,3);
    \draw [ color={rgb,255:red,255; green,0; blue,0}, ->, >=Stealth] (0,1.5) .. controls (0,1.5) and (0,1.5) .. (0,1.75) ;
    \draw (6,0) to[american voltage source] (6,1.5);
\end{circuitikz}
}%
    \end{center}
    \begin{align*}
        I_2 &= \frac{U_q}{(R_1 \parallel R_2) + R_3 + R_4 }\frac{R_2}{R_1 + R_2}\\
        I_2 &= \frac{U_q R_2}{\left(\frac{R_1 R_2}{R_1 + R_2} + R_3 + R_4 \right)(R_1 + R_2)}\\
        I_2 &= \frac{U_q R_2}{R_1 R_2 + R_3 (R_1 + R_2) + R_4(R_1 + R_2)}
    \end{align*}

\end{multicols*}
\begin{center}
    $R_1 R_2 + R_1 R_3 + R_1 R_4 + R_2 R_3 + R_2 R_4 \mathcolor{red}{=} R_1 R_2 + R_1 R_3 + R_1 R_4 + R_2 R_3 + R_2 R_4$
\end{center}

\subsubsection{Herleitung mit Stromquelle}

\begin{multicols*}{2}
    \begin{center}
        % LTeX: enabled=false
\resizebox{\columnwidth}{!}{%
\begin{circuitikz}
    \tikzstyle{every node}=[font=\normalsize]
    \draw (0,3) to[R] (3,3);
    \draw (3,3) to[R] (3,0);
    \draw (3,3) to[R] (6,3);
    \draw (6,3) to[R] (6,0);
    \draw (6,0) to[short] (0,0);
    \node at (3,3) [circ] {};
    \node at (3,0) [circ] {};
    \node [font=\normalsize] at (1.5,3.5) {$R_1$};
    \node [font=\normalsize] at (0.75,1.5) {$I_q$};
    \node [font=\normalsize] at (3.5,1.5) {$R_2$};
    \node [font=\normalsize] at (4.5,3.5) {$R_3$};
    \node [font=\normalsize] at (6.5,1.5) {$R_4$};
    \draw (0,0) to[american current source] (0,3);
    \draw [ color={rgb,255:red,162; green,162; blue,162}, line width=0.6pt, ->, >=Stealth] (1,2.5) .. controls (1.5,2.25) and (1.5,0.75) .. (1,0.5) node[pos=0.5, fill=white]{$U_q$};
    \draw (6,3) to[short, -o] (6.25,3) ;
    \draw (6,0) to[short, -o] (6.25,0) ;
    \draw [ color={rgb,255:red,0; green,128; blue,255}, line width=0.6pt, ->, >=Stealth] (6.75,2.5) .. controls (7.25,2.25) and (7.25,0.75) .. (6.75,0.5) node[pos=0.5, fill=white]{$U_1$};
\end{circuitikz}
}%
    \end{center}
    \[U_1 = I_q \frac{R_2}{R_2 + R_3 + R_4} R_4\]
    \newcolumn

    % LTeX: enabled=false
\resizebox{\columnwidth}{!}{%
\begin{circuitikz}
    \tikzstyle{every node}=[font=\normalsize]
    \draw (0,3) to[R] (3,3);
    \draw (3,3) to[R] (3,0);
    \draw (3,3) to[R] (6,3);
    \draw (6,3) to[R] (6,0);
    \draw (6,0) to[short] (0,0);
    \node at (3,3) [circ] {};
    \node at (3,0) [circ] {};
    \node [font=\normalsize] at (1.5,3.5) {$R_1$};
    \node [font=\normalsize] at (9.75,1.5) {$I_q$};
    \node [font=\normalsize] at (3.5,1.5) {$R_2$};
    \node [font=\normalsize] at (4.5,3.5) {$R_3$};
    \node [font=\normalsize] at (6.5,1.5) {$R_4$};
    \draw [ color={rgb,255:red,162; green,162; blue,162}, line width=0.6pt, ->, >=Stealth] (10,2.5) .. controls (10.5,2.25) and (10.5,0.75) .. (10,0.5) node[pos=0.5, fill=white]{$U_q'$};
    \draw [ color={rgb,255:red,0; green,128; blue,255}, line width=0.6pt, ->, >=Stealth] (0,0.5) .. controls (-0.5,0.75) and (-0.5,2.25) .. (0,2.5) node[pos=0.5, fill=white]{$U_2$};
    \draw (9,3) to[american current source] (9,0);
    \draw (6,3) to[short] (9,3);
    \draw (9,0) to[short] (6,0);
    \node at (6,0) [circ] {};
    \node at (6,3) [circ] {};
\end{circuitikz}
}%
    \[U_2 = I_q \frac{R_4}{R_2 + R_3 + R_4} R_2\]

\end{multicols*}
\begin{center}
    $U_1 = U_2$\\
    \begin{tcolorbox}[colframe=red , colback=white, arc is curved, hbox]
        $U_q \neq U_q'$
    \end{tcolorbox}
\end{center}


\subsection{Ideale Quellen verschieben}\label{Quelln verschieben}

Ideale Quellen Lassen sich verschieben. 
Für das Systematische lösen von Netzwerken kann das von Vorteil sein. 
Allerdings ist es immer ein tradeof. 
Man verfälscht und verliert immer Potenziale oder Ströme. 
Mann kann sich jedoch zu Beginn eine Formel zurechtlegen, falls diese wieder interessant werden. 
Das Verfahren für Strom und Spannungsquelle ist prinzipiell das gleiche.


\subsubsection{Ideale Stromquellen}

Ein virtueller Strom wird über die vorhandene Quelle gelegt. 
Der Quellstrom wird in der Summe null. 
Da der Strom jedoch nicht einfach so verschwindet müssen neue Stromquellen hinzugefügt werden, welche den Strom im Kreis führen:

\begin{center}
    % LTeX: enabled=false
\begin{circuitikz}[scale=0.75]
    \tikzstyle{every node}=[font=\normalsize]
    \draw (0,-0.75) to[R] (3,-0.75);
    \draw (3,-0.75) to[R] (3,2.25);
    \draw (3,2.25) to[R] (0,2.25);
    \draw (0,2.25) to[short, -o] (-0.25,2.25) ;
    \draw (0,2.25) to[short, -o] (0,2.5) ;
    \draw (3,2.25) to[short, -o] (3,2.5) ;
    \draw (3,2.25) to[short, -o] (3.25,2.25) ;
    \draw (3,-0.75) to[short, -o] (3.25,-0.75) ;
    \draw (3,-1) to[short, o-] (3,-0.75) ;
    \draw (0,-1) to[short, o-] (0,-0.75) ;
    \draw (0,-0.75) to[short, -o] (-0.25,-0.75) ;
    \draw [ color={rgb,255:red,255; green,0; blue,0} ](0,2.25) to[american current source,l={ \normalsize I}] (0,-0.75);
    \draw [->, >=Stealth] (4,0.75) -- (6,0.75);
    \draw (8,-0.75) to[R] (11,-0.75);
    \draw (11,-0.75) to[R] (11,2.25);
    \draw (11,2.25) to[R] (8,2.25);
    \draw (8,2.25) to[short, -o] (7.75,2.25) ;
    \draw (8,2.25) to[short, -o] (8,2.5) ;
    \draw (11,2.25) to[short, -o] (11,2.5) ;
    \draw (11,2.25) to[short, -o] (11.25,2.25) ;
    \draw (11,-0.75) to[short, -o] (11.25,-0.75) ;
    \draw (11,-1) to[short, o-] (11,-0.75) ;
    \draw (8,-1) to[short, o-] (8,-0.75) ;
    \draw (8,-0.75) to[short, -o] (7.75,-0.75) ;
    \draw [ color={rgb,255:red,255; green,0; blue,0} ](8,2.25) to[american current source,l={ \normalsize I}] (8,-0.75);
    \node at (3,2.25) [circ] {};
    \node at (0,2.25) [circ] {};
    \node at (3,-0.75) [circ] {};
    \node at (0,-0.75) [circ] {};
    \node at (8,2.25) [circ] {};
    \node at (11,2.25) [circ] {};
    \node at (11,-0.75) [circ] {};
    \node at (8,-0.75) [circ] {};
    \draw [ color={rgb,255:red,200; green,0; blue,150} ](12,1.75) to[american current source,l={ \normalsize I}] (12,-0.25);
    \draw [ color={rgb,255:red,200; green,0; blue,150} ](10.5,-1.75) to[american current source,l={ \normalsize I}] (8.5,-1.75);
    \draw [ color={rgb,255:red,200; green,0; blue,150} ](7,-0.25) to[american current source,l={ \normalsize I}] (7,1.75);
    \draw [ color={rgb,255:red,200; green,0; blue,150} ](8.5,3.25) to[american current source,l={ \normalsize I}] (10.5,3.25);
    \draw [ color={rgb,255:red,200; green,0; blue,150}, ](7,1.75) to[short] (8,1.75);
    \draw [ color={rgb,255:red,200; green,0; blue,150}, ](8.5,3.25) to[short] (8.5,2.25);
    \draw [ color={rgb,255:red,200; green,0; blue,150}, ](10.5,3.25) to[short] (10.5,2.25);
    \draw[ color={rgb,255:red,200; green,0; blue,150}, ] (12,1.75) to[short] (11,1.75);
    \draw[ color={rgb,255:red,200; green,0; blue,150}, ] (12,-0.25) to[short] (11,-0.25);
    \draw [ color={rgb,255:red,200; green,0; blue,150}, ](10.5,-1.75) to[short] (10.5,-0.75);
    \draw [ color={rgb,255:red,200; green,0; blue,150}, ](8.5,-1.75) to[short] (8.5,-0.75);
    \draw [ color={rgb,255:red,200; green,0; blue,150}, ](7,-0.25) to[short] (8,-0.25);
    \node at (8,1.75) [circ] {};
    \node at (8.5,2.25) [circ] {};
    \node at (10.5,2.25) [circ] {};
    \node at (11,1.75) [circ] {};
    \node at (10.5,-0.75) [circ] {};
    \node at (8.5,-0.75) [circ] {};
    \node at (8,-0.25) [circ] {};
    \draw [ color={rgb,255:red,255; green,128; blue,64} , dashed] (6,2) rectangle  (9,-0.5);
    \node [font=\normalsize] at (6.5,2.5) {$\sum = 0$};
    \node [font=\normalsize] at (4.75,1) {Verschieben};
\end{circuitikz}
\end{center}


Das Netzwerk hat nun eine möglicherweise mühsame Stromquelle weniger, allerdings generell mehr. 
Am besten schiebt man diese auf eine ideale Spannungsquelle, um diese ebenso zu eliminieren.


\subsubsection{Ideale Spannungsquellen}

Bei Idealen Spannungsquellen gilt dasselbe wie für ideale Stromquellen. 
Sie werden über einen Knoten hinweg geschoben. 
Mann kann sich das auch als zwingenden Ausgleich durch das Entfernen der ersten Spannungsquelle vorstellen.

\begin{center}
    % LTeX: enabled=false
\begin{circuitikz}[scale=0.6]
    \tikzstyle{every node}=[font=\LARGE]
    \draw (1,3.25) to[R] (4,3.25);
    \draw (6,3.25) to[R] (9,3.25);
    \draw (5,4.25) to[R] (5,7.25);
    \draw (5,4.25) to[short] (5,2.25);
    \draw (6,3.25) to[short] (4,3.25);
    \draw (1,3.25) to[short, -o] (0.75,3.25) ;
    \draw (5,7.25) to[short, -o] (5,7.5) ;
    \draw (9,3.25) to[short, -o] (9.25,3.25) ;
    \draw [ color={rgb,255:red,0; green,128; blue,0}, ](5,-1) to[short, o-] (5,-0.75) ;
    \draw [ color={rgb,255:red,0; green,0; blue,255} ](5,2.25) to[american voltage source] (5,-0.75);
    \draw [ color={rgb,255:red,0; green,128; blue,0}, ](5,-0.75) to[short] (5,0.25);
    \node at (5,3.25) [circ] {};
    \draw (12.75,3.25) to[R] (14.25,3.25);
    \draw (17.75,3.25) to[R] (19,3.25);
    \draw (16,5) to[R] (16,6.5);
    \draw (16,3.5) to[short] (16,3);
    \draw[ color={rgb,255:red,0; green,128; blue,0}, ] (16.25,3.25) to[short] (15.75,3.25);
    \draw (12,3.25) to[short, -o] (11.75,3.25) ;
    \draw (16,7.25) to[short, -o] (16,7.5) ;
    \draw (20,3.25) to[short, -o] (20.25,3.25) ;
    \draw [ color={rgb,255:red,0; green,128; blue,0}, ](16,-1) to[short, o-] (16,-0.75) ;
    \draw [ color={rgb,255:red,0; green,0; blue,255} ](16,1.5) to[american voltage source] (16,0);
    \draw [ color={rgb,255:red,0; green,128; blue,0}, ](16,-0.75) to[short] (16,0.25);
    \node at (16,3.25) [circ, color={rgb,255:red,0; green,128; blue,0}] {};
    \draw [ color={rgb,255:red,128; green,0; blue,64} ](16,5) to[american voltage source] (16,3.5);
    \draw (16,6.5) to[short] (16,7.25);
    \draw (12,3.25) to[short] (12.75,3.25);
    \draw (19,3.25) to[short] (20,3.25);
    \draw [ color={rgb,255:red,128; green,0; blue,64} ](17.75,3.25) to[american voltage source] (16.25,3.25);
    \draw [ color={rgb,255:red,128; green,0; blue,64} ](14.25,3.25) to[american voltage source] (15.75,3.25);
    \draw [ color={rgb,255:red,128; green,0; blue,64} ](16,1.5) to[american voltage source] (16,3);
    \draw [->, >=Stealth] (9.75,3.25) -- (11.25,3.25);
    \node [font=\normalsize] at (10.5,3.75) {Verschieben};
    \node [font=\normalsize] at (5.25,-0.5) {$\varphi$};
    \node [font=\normalsize] at (16.25,-0.5) {$\varphi$};
    \node [font=\normalsize] at (16.25,3) {$\varphi$};
    \draw [ dashed] (15.5,2.75) rectangle  (16.5,0.25);
    \node [font=\LARGE] at (17.5,2) {$\sum = 0 $};
\end{circuitikz}
\end{center}

Note: Das Potenzial $\varphi$ hat sich nach oben verschoben.

\subsection{Widerstände}
\begin{wrapfigure}{l}{0.1\linewidth}
    \begin{circuitikz}
        \draw(0,0) to[R,o-o,l=R] (0,1.5);
    \end{circuitikz}
    \begin{circuitikz}[american resistors]
        \draw(0,0) to[R,o-o,l=R] (0,1.5);
    \end{circuitikz}
\end{wrapfigure}

Fixe, lineare Widerstände haben typischerweise die links stehenden Schaltungssymbole.\\
Links das Europäische(mehr gebräuchlich), rechts das Amerikanische.\\
Es gibt verschieden elementare Schaltungen welche einfacher separat analysiert werden können.\\

\subsubsection{Serienschaltung}

Wenn Widerstände in Serie zueinander stehen, können sie summiert werden.\\
\begin{center}
    % LTeX: enabled=false
\begin{circuitikz}
    \draw (0,0) to[R,l=R1,o-] (2,0)
    to[R,l=R2] (4,0)
    to[R,l=R3] (6,0)
    to[R,l=R4,-o] (8,0);
\end{circuitikz}
\end{center}
    
\[ \text{R}_\text{tot} = \sum_{\text{i}=1}^{\text{n}} \text{R}_\text{i}\]

\subsubsection{Paralellschaltung}

Wenn Widerstände parallel zueinander stehen, können ihre Leitwerte zusammengezählt werden.\\

\begin{center}
    % LTeX: enabled=false
\begin{circuitikz}
    \draw (0.5,2) to[R,*-*,l=R1] (0.5,0);
    \draw (1.5,2) to[R,*-*,l=R2] (1.5,0);
    \draw (2.5,2) to[R,*-*,l=R3] (2.5,0);
    \draw (3.5,2) to[R,*-*,l=R4] (3.5,0);
    \draw (0,0) to[short,o-o] (4,0);
    \draw (0,2) to[short,o-o] (4,2);
\end{circuitikz}
\end{center}

\[ \text{R}_\text{tot} = \frac{1}{\sum_{\text{i}=1}^{\text{n}} \frac{1}{\text{R}_\text{i}}} \]

Wenn \textbf{nur 2} Widerstände paralell stehen, kann folgende vereinfachte Formel verwendet werden: $\frac{\text{R1}\cdot\text{R2}}{\text{R1}+\text{R2}}$\\

\subsubsection{Stern - dreieck umwandlung}

Wenn Widerstände in eine Dreieckskonstellation sind, kann man sie in einen Stern und umgekehrt umwandeln.\\

\begin{minipage}[b]{0.55\linewidth}
    \centering

    % LTeX: enabled=false
\begin{circuitikz}[scale=0.75]
    \draw (0,0) to[R,l=$\text{R}_\text{B}$,o-*] (1.5,1.299);
    \draw (3,0) to[R,l=$\text{R}_\text{C}$,o-*] (1.5,1.299);
    \draw (1.5,2.598) to[R,l=$\text{R}_\text{A}$,o-*] (1.5,1.299);
    \node at(0,0) [left]{B};
    \node at(1.5,2.598) [above]{A};
    \node at(3,0) [right]{C};
    \draw[-Latex] (3,1.299) -- node[midway, above] {umformen} (4,1.299);
\end{circuitikz}
\begin{circuitikz}[scale=0.75]
    \draw (0,0) to[R,l=$\text{R}_\text{AB}$,o-o] (1.5,2.598);
    \draw (0,0) to[R,l=$\text{R}_\text{BC}$,o-o] (3,0);
    \draw (1.5,2.598) to[R,l=$\text{R}_\text{BC}$,o-o] (3,0);
    \node at(0,0) [left]{B};
    \node at(1.5,2.598) [above]{A};
    \node at(3,0) [right]{C};
\end{circuitikz}

\end{minipage}%%
\begin{minipage}[b]{0.45\linewidth}
    \centering
    \[
    \begin{aligned}
        \text{R}_\text{AB} &= \text{R}_\text{A} + \text{R}_\text{B} + \frac{\text{R}_\text{A}\cdot\text{R}_\text{B}}{\text{R}_\text{C}}\\
        \text{R}_\text{Ac} &= \text{R}_\text{A} + \text{R}_\text{C} + \frac{\text{R}_\text{A}\cdot\text{R}_\text{C}}{\text{R}_\text{B}}\\
        \text{R}_\text{cB} &= \text{R}_\text{B} + \text{R}_\text{C} + \frac{\text{R}_\text{B}\cdot\text{R}_\text{C}}{\text{R}_\text{A}}
    \end{aligned}
    \]

    \vspace{4ex}
\end{minipage} 
\begin{minipage}[b]{0.55\linewidth}
    \centering

    % LTeX: enabled=false
\begin{circuitikz}[scale=0.75]
    \draw (0,0) to[R,l=$\text{R}_\text{AB}$,o-o] (1.5,2.598);
    \draw (0,0) to[R,l=$\text{R}_\text{BC}$,o-o] (3,0);
    \draw (1.5,2.598) to[R,l=$\text{R}_\text{BC}$,o-o] (3,0);
    \node at(0,0) [left]{B};
    \node at(1.5,2.598) [above]{A};
    \node at(3,0) [right]{C};
    \draw[-Latex] (3,1.299) -- node[midway, above] {umformen} (4,1.299);
\end{circuitikz}
\begin{circuitikz}[scale=0.75]
    \draw (0,0) to[R,l=$\text{R}_\text{B}$,o-*] (1.5,1.299);
    \draw (3,0) to[R,l=$\text{R}_\text{C}$,o-*] (1.5,1.299);
    \draw (1.5,2.598) to[R,l=$\text{R}_\text{A}$,o-*] (1.5,1.299);
    \node at(0,0) [left]{B};
    \node at(1.5,2.598) [above]{A};
    \node at(3,0) [right]{C};    
\end{circuitikz}

    \vspace{4ex}
\end{minipage}%% 
\begin{minipage}[b]{0.45\linewidth}
    \centering
    \[
    \begin{aligned}
        \text{R}_\text{A} &= \frac{\text{R}_\text{AB}\cdot\text{R}_\text{AC}}{\text{R}_\text{AB}+\text{R}_\text{BC}+\text{R}_\text{AC}}\\
        \text{R}_\text{B} &= \frac{\text{R}_\text{AB}\cdot\text{R}_\text{BC}}{\text{R}_\text{AB}+\text{R}_\text{BC}+\text{R}_\text{AC}}\\
        \text{R}_\text{C} &= \frac{\text{R}_\text{AC}\cdot\text{R}_\text{BC}}{\text{R}_\text{AB}+\text{R}_\text{BC}+\text{R}_\text{AC}}
    \end{aligned}
    \]

    \vspace{4ex}
\end{minipage} 


\subsection{Symetrien in Widerstandsnetzwerken}

Symmetrien können beim Lösen von Widerstandsnetzwerken nützlich sein da komplette Netzwerke nur einmal zusammengefasst werden müssen. 
Dabei ist das Äquipotentialprinzip sehr wichtig da ein Widerstand, der auf der Symmetrielinie liegt, aufgeteilt werden darf. 
Dabei ist es egal ob in paralell oder in serie. 
Voraussetzung ist das Gleiche Widerstandswerte vorligen.

% LTeX: enabled=false
\begin{circuitikz}[scale=0.75]
    \tikzstyle{every node}=[font=\normalsize]
    \draw (0,2.25) to[R,l={ \normalsize R3}] (0,0.25);
    \draw (4,2.25) to[R,l={ \normalsize R4}] (4,0.25);
    \draw (2,4.25) to[R,l={ \normalsize R5}] (2,2.25);
    \draw (0.5,6.25) to[R,l={ \normalsize R1}] (0.5,4.25);
    \draw (3.5,6.25) to[R,l={ \normalsize R2}] (3.5,4.25);
    \draw (0.5,4.25) to[short] (3.5,4.25);
    \draw (0.5,6.25) to[short] (3.5,6.25);
    \draw (4,2.25) to[short] (0,2.25);
    \draw (0,0.25) to[short] (4,0.25);
    \draw [ color={rgb,255:red,255; green,128; blue,0}, dashed] (2,7.25) -- (2,-0.75);
    \node [font=\large] at (2.5,7.5) {symetrie Linie};
    \node [font=\LARGE] at (2.5,7.5) {};
    \draw (0.5,6.25) to[short, -o] (-0.5,6.25) ;
    \draw (0,0.25) to[short, -o] (-1,0.25) ;
    \node at (0.5,6.25) [circ] {};
    \node at (3.5,6.25) [circ] {};
    \node at (0,0.25) [circ] {};
    \node at (2,2.25) [circ] {};
    \draw [->, >=Stealth] (5.25,3.25) -- (7,3.25);
    \node at (2,4.25) [circ] {};
    \draw (7.5,2.25) to[R,l={ \normalsize R6}] (7.5,0.25);
    \draw (11.5,2.25) to[R,l={ \normalsize R7}] (11.5,0.25);
    \draw [ color={rgb,255:red,255; green,0; blue,0} ](8.5,4.25) to[R,l={ \normalsize R5,5}] (8.5,2.25);
    \draw (8,6.25) to[R,l={ \normalsize R1}] (8,4.25);
    \draw (11,6.25) to[R,l={ \normalsize R2}] (11,4.25);
    \draw (8,4.25) to[short] (11,4.25);
    \draw (8,6.25) to[short] (11,6.25);
    \draw (11.5,2.25) to[short] (7.5,2.25);
    \draw (7.5,0.25) to[short] (11.5,0.25);
    \draw [ color={rgb,255:red,255; green,128; blue,0}, dashed] (9.5,7.25) -- (9.5,-0.75);
    \node [font=\large] at (10,7.5) {symetrie Linie};
    \node [font=\LARGE] at (10,7.5) {};
    \draw (8,6.25) to[short, -o] (7,6.25) ;
    \draw (7.5,0.25) to[short, -o] (6.5,0.25) ;
    \node at (8,6.25) [circ] {};
    \node at (7.5,0.25) [circ] {};
    \node at (8.5,2.25) [circ] {};
    \node at (8.5,4.25) [circ] {};
    \draw [ color={rgb,255:red,255; green,0; blue,0} ](10.5,4.25) to[R,l={ \normalsize R6,5}] (10.5,2.25);
    \node at (10.5,4.25) [circ] {};
    \node at (10.5,2.25) [circ] {};
    \node [font=\normalsize] at (13,3.25) {(2 x R Wert)};
\end{circuitikz}

Nun können beide Seiten separat angeschaut werden. 
Da beide Seiten identisch sind, kann nur eine Seite berechnet und der resultierende Wert halbiert werden. 